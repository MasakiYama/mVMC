% !TEX root = userguide_en.tex
%----------------------------------------------------------
\chapter{\tr{Algorithm}}
\label{Ch:algorithm}
\section{Variational MonteCalro Method}
In Variational MonteCalro (VMC) method, the importance sampling is performed in the system constructed by the Malkov chain toward the appropriate complete system.
Here, we choose the real spatial arrangement $\{| x\rangle\}$ at $S_z = 0$ as the complete system.
\begin{equation}
| x\rangle =  \prod_{n=1}^{N/2} c_{r_{n\uparrow}}^{\dag} \prod_{n=1}^{N/2} c_{r_{n\downarrow}}^{\dag} |0 \rangle,
\end{equation}
where we set $r_{n\sigma}$ as the position of $n$-th electron with $\sigma (=\uparrow \rm{or} \downarrow)$ spin.

\subsection{Importance sampling}
When the importance of Malkov chain is defined by
\begin{equation}
\rho(x)=\frac{|\langle x| \psi \rangle|^2}{\langle \psi | \psi \rangle} \ge 0, \sum{x} \rho(x)=1,
\end{equation}
the expected value of the operator $A$ is given by
\begin{equation}
\langle A \rangle =\frac{\langle \psi| A| \psi \rangle}{\langle \psi | \psi \rangle} 
=\sum_x \frac{\langle \psi| A | x\rangle \langle x| \psi \rangle}{\langle \psi |\psi \rangle} 
=\sum_x \rho(x) \frac{\langle \psi| A | x\rangle }{\langle \psi |x \rangle} .
\end{equation}
In VMC, the summation in terms of $x$ is replaced by the importance sampling.
Then, local Green's function $G_{ij\sigma\sigma'}(x)$ is defined by
\begin{equation}
G_{ij\sigma\sigma'}(x)=\frac{\langle \psi | c_{i\sigma}^{\dag} c_{j\sigma'} | \psi \rangle}{\langle \psi | x \rangle}.
\end{equation}
In mVMC, the Mersenne twister method is used for sampling as the random number generator \cite{Mutsuo2008}. 

\section{Bogoliubov representation}\label{sec_bogoliubov_rep}

In the spin system,
the spin indices in input files of \verb|transfer|, \verb|InterAll|,
and correlation functions are specified as those of the Bogoliubov representation.
Spin operators are written by using creation/annihilation operators as follows:
\begin{align}
  S_{i z} &= \sum_{\sigma = -S}^{S} \sigma c_{i \sigma}^\dagger c_{i \sigma}
  \\
  S_{i}^+ &= \sum_{\sigma = -S}^{S-1} 
  \sqrt{S(S+1) - \sigma(\sigma+1)} 
  c_{i \sigma+1}^\dagger c_{i \sigma}
  \\
  S_{i}^- &= \sum_{\sigma = -S}^{S-1} 
  \sqrt{S(S+1) - \sigma(\sigma+1)} 
  c_{i \sigma}^\dagger c_{i \sigma+1}
\end{align}

\section{Relation between Pfaffian Slater determinant and single Slater determinant}
\label{sec:PuffAndSlater}
In this section, we show relation between Pfaffian Slater determinant and single Slater determinant.
We also discuss meaning of the singular value decomposition of coefficients $f_{ij}$. 
\subsection{Relation between $f_{ij}$ and $\Phi_{in\sigma}$}
Pfaffian Slater determinant [one-body part of the many-variable variational Monte Carlo (mVMC) method]
is defined as
\begin{equation}
|\phi_{\rm Pf}\rangle=\Big(\sum_{i,j=1}^{N_{s}}f_{ij}c_{i\uparrow}^{\dagger}c_{j\downarrow}^{\dagger}\Big)^{N_{\rm e}/2}|0\rangle,
\end{equation}
where $N_{s}$ is number of sites, 
$N_{e}$ is number of total particles,
and $f_{ij}$ are variational parameters.
For simplicity, we assume that $f_{ij}$ are real number.
Single Slater determinant is defined as 
\begin{align}
|\phi_{\rm SL}\rangle&=\Big(\prod_{n=1}^{N_{e}/2}\psi_{n\uparrow}^{\dagger}\Big)
\Big(\prod_{m=1}^{N_{e}/2}\psi_{m\downarrow}^{\dagger}\Big)|0\rangle, \\
\psi_{n\sigma}^{\dagger}&=\sum_{i=1}^{N_{s}}\Phi_{in\sigma}c^{\dagger}_{i\sigma}.
\end{align}
We note that $\Phi$ is the normalized orthogonal basis, i.e, 
\begin{equation}
\sum_{i=1}^{N_{s}}\Phi_{in\sigma}\Phi_{im\sigma}=\delta_{nm},
\end{equation}
where $\delta_{nm}$ is the Kronecker's delta.
Due to this normalized orthogonality, we obtain 
following relation:
\begin{align}
[\psi^{\dagger}_{n\sigma},\psi_{m\sigma}]_{+}&=\delta_{nm},\\
G_{ij\sigma}=\langle c_{i\sigma}^{\dagger}c_{j\sigma}\rangle 
&=\frac{\langle \phi_{\rm SL}| c_{i\sigma}^{\dagger}c_{j\sigma} | \phi_{\rm SL}\rangle}{\langle \phi_{\rm SL}|\phi_{\rm SL}\rangle } \\
&=\sum_{n} \Phi_{in\sigma} \Phi_{jn\sigma}.
\end{align}

Here, we rewrite $\phi_{\rm SL}$ and obtain explicit 
relation between $f_{ij}$ and $\Phi_{in\sigma}$.
By using the commutation relation for $\psi^{\dagger}_{n\sigma}$,
we rewrite $\phi_{\rm SL}$ as 
\begin{align}
|\phi_{\rm SL}\rangle \propto \prod_{n=1}^{N_{e}/2}\Big(\psi_{n\uparrow}^{\dagger}\psi_{\mu(n)\downarrow}^{\dagger}\Big)|0\rangle,
\end{align}
where $\mu(n)$ represents permutation of sequence of $n= 1, 2, \cdots, N_{e}/2$.
For simplicity, we take identity permutation and obtain the relation 
\begin{align}
|\phi_{\rm SL}\rangle &\propto \prod_{n=1}^{N_{e}/2}\Big(\psi_{n\uparrow}^{\dagger}\psi_{n\downarrow}^{\dagger}\Big)|0\rangle
=\prod_{n=1}^{N_{e}/2} K_{n}^{\dagger}|0\rangle \\
&\propto\Big(\sum_{n=1}^{\frac{N_{e}}{2}}K_{n}^{\dagger}\Big)^{\frac{N_{e}}{2}} |0\rangle
=\Big(\sum_{i,j=1}^{N_{s}}\Big[\sum_{n=1}^{\frac{N_{e}}{2}}\Phi_{in\uparrow}\Phi_{jn\downarrow}\Big]
c_{i\uparrow}^{\dagger}c_{j\downarrow}^{\dagger}\Big)|0\rangle,
\end{align}
where $K_{n}^{\dagger}=\psi_{n\uparrow}^{\dagger}\psi_{n\downarrow}^{\dagger}$ and
we use the relation  $K_{n}^{\dagger}K_{m}^{\dagger}=K_{m}^{\dagger}K_{n}^{\dagger}$.
This result shows that $f_{ij}$ can be expressed by the 
coefficients of the single Slater determinant as
\begin{align}
f_{ij}=\sum_{n=1}^{\frac{N_{e}}{2}}\Phi_{in\uparrow}\Phi_{jn\downarrow}.
\end{align}
We note that this is one of expression of $f_{ij}$ for 
single Slater determinant, i.e, $f_{ij}$ depend on
the pairing degrees of freedom (choices  of $\mu(n)$) and
gauge degrees of freedom ( we can arbitrary change
the sign of $\Phi$ as $\Phi_{in\sigma}\rightarrow -\Phi_{in\sigma}$).
This large degrees of freedom is the origin of huge redundancy of
$f_{ij}$.

\subsection{Singular value decomposition of $f_{ij}$}
We define matrices $F$, $\Phi_{\uparrow}$, $\Phi_{\downarrow}$, and $\Sigma$ as
\begin{align}
&(F)_{ij}=f_{ij},~~~ 
(\Phi_{\uparrow})_{in}=\Phi_{in\uparrow},~~~ 
(\Phi_{\downarrow})_{in}=\Phi_{in\downarrow}, \\
&\Sigma={\rm diag}[1,\cdots,1,0,0,0]~~~\text{({\rm \# of 1} = $N_{e}/2$)}.
\end{align}
By using these notations, we can describe the
singular value decomposition of $f_{ij}$ (or equivalently $F$) as 
\begin{align}
F=\Phi_{\uparrow}\Sigma\Phi_{\downarrow}^{t}.
\end{align}
This result indicates that $f_{ij}$ can be 
described by the mean-field solutions
if the number of
nonzero singular values are $N_{e}/2$ and
all the nonzero singular values of $F$ are one.
In other word, the singular values including their numbers
offers the quantitative criterion how the Pfaffian Slater determinant
deviates from the single Slate determinant.


%----------------------------------------------------------
