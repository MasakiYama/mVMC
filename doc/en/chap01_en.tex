% !TEX root = userguide_en.tex
%----------------------------------------------------------
\chapter{What is mVMC?}
\label{Ch:whatismVMC}
%----------------------------------------------------------
%----------------------------------------------------------
%----------------------------------------------------------
%----------------------------------------------------------
\section{Overview of mVMC}
By using mVMC, the following calculation can be done:
\begin{itemize}
\item{The variational wave function which gives the minimum value of the expected value of energy in the range of the degree of freedoms of variational parameters is numerically generated. The calculation limited to the partial space divided by quantum numbers is also possible. }
\item{The expected values of the physical quantities such as correlation functions can be calculated by using the generated variational wave functions.}
%\item{
%In the specific case under the model where the interaction terms of the Hamiltonian is diagonal in the real space and the electrons are all itinerant, 
%the expected values can be calculated by adopting the Power Lanczos (Single Lanczos Step) method.
%}
\end{itemize}

The calculation flow in mVMC is shown as follows:
\begin{enumerate}
\item{Read input files (*.def)}
\item{Optimize variational parameters $\vec{\alpha}$ to minimize $\langle {\cal H} \rangle$}
\item{Calculate one body and two body Green functions}
\item{Output variational parameters and expected values}
\end{enumerate}
In calculation, the simple parallelization for the generation of the real space arrangement $|x\rangle$ and the collecting samples and the calculation result of expected energies is implemented. Following the procedure for each cluster computers, the parallelized calculation using MPI is automatically done by indicating the parallel number. However, mVMC cannot execute under the environment where the MPI job is forbidden such as the front-end of system B at ISSP. In mVMC, we use PFAPACK\cite{PFAPACK} to compute the Pfaffian matrix.

\section{License}
The distribution of the program package and the source codes for mVMC follows GNU General Public License version 3 (GPL v3). 
\section{Copyright}
\begin{quote}
{\it \copyright 2016- The University of Tokyo} {\it  All rights reserved.}
\end{quote}
This software is developed under the support of ``{\it Project for advancement of software usability in materials science }" by The Institute for Solid State Physics, The University of Tokyo. 

\section{Contributors}
\label{subsec:developers}
This software is developed by following contributors.
\begin{itemize}
\item{ver.1.0 (released at \tr{2016/xx/xx})}
\begin{itemize}
\item{Developers}
	\begin{itemize}
	\item{Takahiro Misawa \\(The Institute for Solid State Physics, The University of Tokyo)}
	\item{Satoshi Morita \\(The Institute for Solid State Physics, The University of Tokyo)}
	\item{Takahiro Ogoe \\(Department of Applied Physics, The University of Tokyo)}
	\item{Kota Ido \\(Department of Applied Physics, The University of Tokyo)}
	\item{Masatoshi Imada \\(Department of Applied Physics, The University of Tokyo)}
	\item{Mitsuaki Kawamura \\(The Institute for Solid State Physics, The University of Tokyo)}
	\item{Kazusyohi Yoshimi \\(The Institute for Solid State Physics, The University of Tokyo)}
	\end{itemize}

\item{Project coordinator}
	\begin{itemize}
	\item{Takeo Kato \\(The Institute for Solid State Physics, The University of Tokyo)}
	\end{itemize}

\end{itemize}

\end{itemize}


\section{Operating environment}
mVMC is tested in the following platform:
\begin{itemize}
\item The supercomputer system-B ``sekirei'' and system-C ``maki'' in ISSP
\item K computer
\item OpenMPI + Intel Compiler + MKL
\item MPICH + Intel Compiler + MKL
\item MPICH + GNU Compiler + MKL
\end{itemize}
