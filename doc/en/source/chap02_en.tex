% !TEX root = userguide_en.tex
%----------------------------------------------------------
\chapter{How to use mVMC?}
\label{Ch:HowTo}

\section{Prerequisite}

mVMC requires the following packages:
\begin{itemize}
\item C compiler (intel, Fujitsu, GNU, etc. )
\item MPI library
\item Option: ScaLAPACK library (intel MKL, Fujitsu, ATLAS, etc.)
\end{itemize}

\begin{screen}
\Large 
{\bf Tips}
\normalsize

{\bf E. g. / Settings of intel compiler}

When you use the intel compiler, you can use easily scripts attached to the compiler.
In the case of the bash in 64 bit OS, write the following in your \verb|~/.bashrc|:
\begin{verbatim}
source /opt/intel/bin/compilervars.sh intel64
\end{verbatim}
or
\begin{verbatim}
source /opt/intel/bin/iccvars.sh intel64
source /opt/intel/mkl/bin/mklvars.sh
\end{verbatim}

Please read manuals of your compiler/library for more information.

\end{screen}

\section{Installation}

% !TEX root = userguide_en.tex
%----------------------------------------------------------
You can download mVMC in the following place.\\
\url{https://github.com/issp-center-dev/mVMC/releases}

You can obtain the mVMC directory by typing
\begin{verbatim}
$ tar xzvf mVMC-xxx.tar.gz
\end{verbatim}

There are two kind of procedures to install mVMC.

\subsection{Using \texttt{mVMCconfig.sh}}

Please run \verb|mVMCconfig.sh| script in the mVMC directory as follow
(for ISSP system-B ''sekirei''):
\begin{verbatim}
$ bash mVMCconfig.sh sekirei
\end{verbatim}
Then environmental configuration file \verb|make.sys| is generated in 
\verb|src/| directory.
The command-line argument of \verb|mVMCconfig.sh| is as follows:
\begin{itemize}
\item \verb|sekirei| : ISSP system-B ''sekirei''
\item \verb|kei| : K computer and ISSP system-C ''maki''
\item \verb|intel-openmpi| : Intel compiler + OpenMPI
\item \verb|intel-mpich| : Intel compiler + MPICH2
\item \verb|intel-intelmpi| : Intel compiler + IntelMPI
\item \verb|gcc-openmpi| : GCC + OpenMPI
\item \verb|gcc-mpich-mkl| : GCC + MPICH + MKL
\end{itemize}

\verb|make.sys| is as follows (for ISSP-system-B ''sekirei''):
\begin{verbatim}
CC = mpicc
F90 = mpif90
CFLAGS = -O3 -no-prec-div -xHost -qopenmp -Wno-unknown-pragmas
FFLAGS = -O3 -implicitnone -xHost
LIBS = -L $(MKLROOT)/lib/intel64 -lmkl_scalapack_lp64 -lmkl_intel_lp64 \
       -lmkl_intel_thread -lmkl_core -lmkl_blacs_sgimpt_lp64 -lpthread -lm
SFMTFLAGS = -no-ansi-alias -DHAVE_SSE2
\end{verbatim}
We explain macros of this file as: 
\begin{itemize}
\item \verb|CC| : C compiler (\verb|mpicc|, \verb|mpifccpx|)
\item \verb|F90| : fortran compiler (\verb|ifort|, \verb|frtpx|)
\item \verb|Libs| : Linker option
\item \verb|CFLAGS| : C compile option
\item \verb|FFLAGS| : fortran compile option
\end{itemize}


Then you are ready to compile mVMC.
Please type
\begin{verbatim}
$ make mvmc
\end{verbatim}
and obtain \verb|vmc.out| and \verb|vmcdry.out| in \verb|src/| directory;
you should add this directory to the \verb|$PATH|.

\begin{screen}
\Large 
{\bf Tips}
\normalsize

You can make a PATH to mVMC as follows:
\\
\verb|$ export PATH=${PATH}:|\textit{mVMC\_top\_directory}\verb|/src/|
\\
If you keep this PATH, you should write above in \verb|~/.bashrc|
(for \verb|bash| as a login shell)

\end{screen}

\subsection{Using \texttt{cmake}}

We can compile mVMC as
\begin{verbatim}
cd $HOME/build/mvmc
cmake -DCONFIG=gcc $PathTomvmc
make
\end{verbatim}
Here, we set a path to mVMC as \verb|$PathTomvmc|
and to a build directory as \verb| $HOME/build/mvmc|.
After compilation, \verb|src| folder is constructed below a \verb|$HOME/build/mvmc|
folder and we obtain an executable \verb|vmc.out| in \verb|src/| directory.

In the above example,
we compile mVMC by using a gcc compiler.
We can select a compiler by using the following options:
\begin{itemize}
\item \verb|sekirei| : ISSP system-B ``sekirei''
\item \verb|fujitsu| : Fujitsu compiler 
\item \verb|intel| : Intel compiler + Linux PC
\item \verb|gcc| : GCC compiler + Linux PC.
\end{itemize}
An example of compiling mVMC by using the Intel compiler is shown as follows:
\begin{verbatim}
mkdir ./build
cd ./build
cmake -DCONFIG=intel ../
make
\end{verbatim}
After compilation,
\verb|src| folder is created below the \verb|build| folder and
an execute \verb|vmc.out in the \verb|src| folder.
We can select ScaLAPACK instead of LAPACK for vmc calculation 
by adding the following as the cmake option
\begin{verbatim}
-DUSE_SCALAPACK=ON -DSCALAPACK_LIBRARIES="xxx".
\end{verbatim}
Here, xxx is the libraries to use ScaLAPACK.
Please note that we must delete the \verb|build| folder and
repeat the above operations when we change the compiler.

\begin{screen}
\Large 
{\bf Tips}
\normalsize\\
Before using cmake for sekirei, you must type 
\begin{verbatim}
source /home/issp/materiapps/tool/env.sh
\end{verbatim}

When we type the following in sekirei, 
\begin{verbatim}
cmake -DCONFIG=sekirei ../ -DUSE_SCALAPACK=ON ,
\end{verbatim}
\verb$-DSCALAPACK_LIBRARIES$ is automatically set as
\begin{verbatim}
-DSCALAPACK_LIBRARIES="\${MKLROOT}/lib/intel64 -lmkl_scalapack_lp64 
-lmkl_intel_lp64 -lmkl_intel_thread -lmkl_core
-lmkl_blacs_sgimpt_lp64".
\end{verbatim}
When the path to libraries for ScaLAPACK is different in your circumstance,
please set \verb$-DSCALAPACK_LIBRARIES$ as the correct path.
\end{screen}

\label{Sec:HowToInstall}

\section{Directory structure}
When mVMC-xxx.tar.gz is unzipped, the following directory structure is composed.\\
\\
├──COPYING\\
├──mVMCconfig.sh\\
├──doc/\\
│~~~~~~├──bib/\\
│~~~~~~│~~~~~~├──elsart-num\_mod.bst\\
│~~~~~~│~~~~~~└──userguide.bib\\
│~~~~~~├──figs/\\
│~~~~~~│~~~~~~├──*.pdf\\
│~~~~~~│~~~~~~└──*.xbb\\
│~~~~~~├──fourier/\\
│~~~~~~│~~~~~~├──en/\\
│~~~~~~│~~~~~~├──figs/\\
│~~~~~~│~~~~~~└──ja/\\
│~~~~~~├──jp/\\
│~~~~~~│~~~~~~└──*.tex\\
│~~~~~~└──en/\\
│~~~~~~~~~~~~~└──*.tex\\
├──sample/\\
│~~~~~~└──Standard/\\
│~~~~~~~~~~~~~~~~~~├──Hubbard/\\
│~~~~~~~~~~~~~~~~~~│~~~~~~├─square/\\
│~~~~~~~~~~~~~~~~~~│~~~~~~│~~~~~~├──StdFace.def\\
│~~~~~~~~~~~~~~~~~~│~~~~~~│~~~~~~└──reference/\\
│~~~~~~~~~~~~~~~~~~│~~~~~~│~~~~~~~~~~~~~~~~~└──**.dat\\
│~~~~~~~~~~~~~~~~~~│~~~~~~└─triangular/\\
│~~~~~~~~~~~~~~~~~~│~~~~~~~~~~~~└──$\cdots$\\
│~~~~~~~~~~~~~~~~~~├──Kondo/\\
│~~~~~~~~~~~~~~~~~~│~~~~~~└─chain/\\
│~~~~~~~~~~~~~~~~~~│~~~~~~~~~~~~└──$\cdots$\\
│~~~~~~~~~~~~~~~~~~└──Spin/\\
│~~~~~~~~~~~~~~~~~~~~~~~~~~~~~~├─HeisenbergChain/\\
│~~~~~~~~~~~~~~~~~~~~~~~~~~~~~~│~~~~~~└──$\cdots$\\
│~~~~~~~~~~~~~~~~~~~~~~~~~~~~~~├─HeisenbergSquare/\\
│~~~~~~~~~~~~~~~~~~~~~~~~~~~~~~│~~~~~~└──$\cdots$\\
│~~~~~~~~~~~~~~~~~~~~~~~~~~~~~~└─Kagome/\\
│~~~~~~~~~~~~~~~~~~~~~~~~~~~~~~~~~~~~~└──$\cdots$\\
├──src/\\
│~~~~~~~~~~├──mVMC/\\
│~~~~~~~~~~│~~~~~~├─ **.c\\
│~~~~~~~~~~│~~~~~~└──include/\\
│~~~~~~~~~~│~~~~~~~~~~~~~~└──**.h\\
│~~~~~~~~~~├──ComplexUHF/\\
│~~~~~~~~~~│~~~~~~├─ **.c\\
│~~~~~~~~~~│~~~~~~└──include/\\
│~~~~~~~~~~│~~~~~~~~~~~~~~└──**.h\\
│~~~~~~~~~~├──StdFace/\\
│~~~~~~~~~~│~~~~~~~├──**.c\\
│~~~~~~~~~~│~~~~~~~└──**.h\\
│~~~~~~~~~~├──pfapack/\\
│~~~~~~~~~~│~~~~~~~├──makefile\_pfapack\\
│~~~~~~~~~~│~~~~~~~└──**.f\\
│~~~~~~~~~~└──sfmt/\\
│~~~~~~~~~~~~~~~~~~├──makefkie\_sfmt\\
│~~~~~~~~~~~~~~~~~~├──**.c\\
│~~~~~~~~~~~~~~~~~~└──**.h\\
└──tool/\\
~~~~~~~~~~~├──**.f90\\
~~~~~~~~~~~└──makefile\_tool\\

\newpage
\section{Basic usage}

mVMC works as whether the following two modes:

\begin{itemize}
\item Expert mode

  mVMC supports the arbitrary fermion-/spin-lattice system;
  we can specify the hopping, etc. at each site independently.
  Although this makes us able to specify flexibly the target
  this requires many input-files, and
  the setup of the calculation is complicated.

\item Standard mode

  For some typical models (such as the Heisenberg model on the square lattice),
  we can start calculation with a few parameters (for example, the size of the
  simulation cell, the common coupling parameter).
  In this case, the input-files for Expert mode are automatically generated.
  Although the number of available systems is smaller than that number of Expert mode,
  the setup of the calculation is easier than in Expert mode.

\end{itemize}

We can calculate by using these modes as follows:

\begin{enumerate}

\item  Prepare a minimal input file

You can choose a model (the Heisenberg model, the Hubbard model, etc.) and 
a lattice (the square lattice, the triangular lattice, etc.) from ones provided;
you can specify some parameters (such as the first/second nearest neighbor hopping integrals,
the on-site Coulomb integral, etc.) for them.
The input file format is described in the Sec. \ref{Ch:HowToStandard}.

\item  Run

Run a executable \verb|vmc.out| in terminal by specifying
the name of input file written in previous step
(option \verb|-s| is required).

\verb|$ mpiexec -np |\textit{number\_of\_processes}\verb| |\textit{Path}\verb|/vmc.out -s| \textit{Input\_file\_name}

When you use a queuing system in workstations or super computers, 
sometimes the number of processes is specified as an argument for the job-submitting command.
If you need more information, please refer manuals for your system. 

\item Watch calculation logs

Log files are outputted in the \verb|output/| directory which is automatically made in the directory for a calculation scenario.
The details of output files are shown in Sec. \ref{Sec:outputfile}.

\item Results

  If the calculation is finished normally, the result files are outputted in the \verb|output/| directory.
  The details of output files are shown in Sec. \ref{Sec:outputfile}.

\item Prepare and run Expert mode

  In the above case, the calculation starts as soon as input files
  for Expert mode are generated.
  If we only generate files without starting the calculation,
  we can use a executable \verb|vmcdry.out| as follows
  (MPI is not used in this step):

  \verb|$ |\textit{Path}\verb|/vmcdry.out |\textit{Input\_file\_name}

  Then, we can edit generated files by hand and run a executable \verb|vmc.out| with 
  \verb|namelist.def| as an argument (option \verb|-e| is required) as follows:

  \verb|$ mpiexec -np |\textit{number\_of\_processes}\verb| |\textit{Path}\verb|/vmcdry.out -e namelist.def|

\end{enumerate}

\begin{screen}
\Large 
{\bf Tips}
\normalsize

{\bf The number of threads for OpenMP}

If you specify the number of OpenMP threads for mVMC,
you should set it as follows (in case of 16 threads) before the running:
\begin{verbatim}
export OMP_NUM_THREADS=16
\end{verbatim}

\end{screen}

\section{Printing version ID}

By using \verb|-v| option as follows, 
you can check which version of mVMC you are using.

\begin{verbatim}
$ PATH/vmcdry.out -v
$ PATH/vmc.out -v
\end{verbatim}
