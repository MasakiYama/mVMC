% !TEX root = userguide_en.tex
%----------------------------------------------------------
\chapter{How to use mVMC?}
\label{Ch:HowTo}

\section{Prerequisite}

mVMC requires the following packages:
\begin{itemize}
\item C compiler (intel, Fujitsu, GNU, etc. )
\item ScaLAPACK library (intel MKL, Fujitsu, ATLAS, etc.)
\item MPI library
\end{itemize}

\begin{screen}
\Large 
{\bf Tips}
\normalsize

{\bf E. g. / Settings of intel compiler}

When you use the intel compiler, you can use easily scripts attached to the compiler.
In the case of the bash in 64 bit OS, write the following in your \verb|~/.bashrc|:
\begin{verbatim}
source /opt/intel/bin/compilervars.sh intel64
\end{verbatim}
or
\begin{verbatim}
source /opt/intel/bin/iccvars.sh intel64
source /opt/intel/mkl/bin/mklvars.sh
\end{verbatim}

Please read manuals of your compiler/library for more information.

\end{screen}

\section{Installation}

% !TEX root = userguide_en.tex
%----------------------------------------------------------
You can download mVMC in the following place.\\

You can obtain the mVMC directory by typing
\begin{verbatim}
$ tar xzvf mVMC-xxx.tar.gz
\end{verbatim}

There are two kind of procedures to install mVMC.

\subsection{Using \texttt{config.sh}}

Please run \verb|config.sh| script in the mVMC directory as follow
(for ISSP system-B ''sekirei''):
\begin{verbatim}
$ bash config.sh sekirei
\end{verbatim}
Then environmental configuration file \verb|make.sys| is generated in 
\verb|src/| directory.
The command-line argument of \verb|config.sh| is as follows:
\begin{itemize}
\item \verb|sekirei| : ISSP system-B ''sekirei''
\item \verb|kei| : K computer and ISSP system-C ''maki''
\item \verb|openmpi-intel| : OpenMPI + intel compiler
\item \verb|gnu| : GCC
\end{itemize}

\verb|make.sys| is as follows (for ISSP-system-B ''sekirei''):
\begin{verbatim}
CC = mpicc
LIB = -L$(MKLROOT)/lib/intel64 -lmkl_scalapack_lp64 -lmkl_intel_lp64 \
      -lmkl_intel_thread -lmkl_core -lmkl_blacs_sgimpt_lp64 -lpthread -lm
CFLAGS = -O3 -no-prec-div -xHost -qopenmp -Wno-unknown-pragmas
REPORT = -qopt-report-phase=openmp -qopt-report-phase=par
OPTION = -D_mpi_use
CP = cp -f -v
AR = ar rv
FORT = ifort
FFLAGS = -O3 -implicitnone -xHost
SMFTFLAGS = -O3 -no-ansi-alias -xHost -DMEXP=19937 -DHAVE_SSE2
\end{verbatim}
We explain macros of this file as: 
\begin{itemize}
\item \verb|CC| : The C Compilation command (\verb|mpicc|, \verb|mpifccpx|)
\item \verb|Lib| : Compilation options for ScaLAPACK.
\item \verb|CFLAGS| : Other compilation options.
\item \verb|FORT| : Fortran compilation command (\verb|ifort|, \verb|frtpx|)
\end{itemize}


Then you are ready to compile mVMC.
Please type
\begin{verbatim}
$ make mvmc
\end{verbatim}
and obtain \verb|vmc.out| and \verb|vmcdry.out| in \verb|src/| directory;
you should add this directory to the \verb|$PATH|.

\begin{screen}
\Large 
{\bf Tips}
\normalsize

You can make a PATH to mVMC as follows:
\\
\verb|$ export PATH=${PATH}:|\textit{mVMC\_top\_directory}\verb|/src/|
\\
If you keep this PATH, you should write above in \verb|~/.bashrc|
(for \verb|bash| as a login shell)

\end{screen}


\subsection{Using \texttt{cmake}}

\begin{screen}
\Large 
{\bf Tips}
\normalsize\\
Before using cmake for sekirei, you must type 
\begin{verbatim}
source /home/issp/materiapps/tool/env.sh
\end{verbatim}
while for maki, you must type
\begin{verbatim}
source /global/app/materiapps/tool/env.sh
\end{verbatim}
\end{screen}

We can compile mVMC as
\begin{verbatim}
cd $HOME/build/mvmc
cmake -DCONFIG=gcc $PathTomVMC
make
\end{verbatim}
Here, we set a path to mVMC as \verb| $PathTomVMC| and to a build directory as \verb| $HOME/build/mvmc|. 
After compiling, a src folder is constructed below a \verb| $HOME/build/mvmc |folder 
and obtain executables \verb|vmc.out| and \verb|vmcdry.out| in \verb|src/| directory. 

In the above example, we compile mVMC by using a gcc compiler. 
We can select a compiler by using following options
\begin{itemize}
\item \verb|sekirei| : ISSP system-B ''sekirei''
\item \verb|fujitsu| : Fujitsu compiler (ISSP system-C ''maki'')
\item \verb|intel| : intel compiler + Linux PC
\item \verb|gcc| : GCC compiler + Linux PC.
\end{itemize}
An example for compiling mVMC by an intel compiler is shown as follows, 
\begin{verbatim}
mkdir ./build
cd ./build
cmake -DCONFIG=intel ../
make
\end{verbatim}
After compiling,  a \verb|src| folder is made below the \verb|build| folder and executes
\verb|vmc.out| and \verb|vmcdry.out| are made in the  \verb|src| folder. 
It is noted that  we must delete the  \verb|build| folder and do the above works again 
when we change the compilers.

\label{Sec:HowToInstall}

\section{Directory structure}
When mVMC-xxx.tar.gz is unzipped, the following directory structure is composed.\\
\\
├──COPYING\\
├──config.sh\\
├──doc/\\
│~~~~~~├──bib/\\
│~~~~~~│~~~~~~├──elsart-num\_mod.bst\\
│~~~~~~│~~~~~~└──userguide.bib\\
│~~~~~~├──figs/\\
│~~~~~~│~~~~~~├──*.pdf\\
│~~~~~~│~~~~~~└──*.xbb\\
│~~~~~~├──jp/\\
│~~~~~~│~~~~~~└──*.tex\\
│~~~~~~└──en/\\
│~~~~~~~~~~~~~└──*.tex\\
├──sample/\\
│~~~~~~├──Expert/\\
│~~~~~~│~~~~~~├──Hubbard/\\
│~~~~~~│~~~~~~│~~~~~~├─square/\\
│~~~~~~│~~~~~~│~~~~~~│~~~~~~├──*.def\\
│~~~~~~│~~~~~~│~~~~~~│~~~~~~└──output\_ref/\\
│~~~~~~│~~~~~~│~~~~~~│~~~~~~~~~~~~~~~~~~└──**.dat\\
│~~~~~~│~~~~~~│~~~~~~└─triangular/\\
│~~~~~~│~~~~~~│~~~~~~~~~~~~└──$\cdots$\\
│~~~~~~│~~~~~~├──Kondo/\\
│~~~~~~│~~~~~~│~~~~~~└─chain/\\
│~~~~~~│~~~~~~│~~~~~~~~~~~~└──$\cdots$\\
│~~~~~~│~~~~~~└──Spin/\\
│~~~~~~│~~~~~~~~~~~~~~~├─HeisenbergChain/\\
│~~~~~~│~~~~~~~~~~~~~~~│~~~~~~└──$\cdots$\\
│~~~~~~│~~~~~~~~~~~~~~~├─HeisenbergSquare/\\
│~~~~~~│~~~~~~~~~~~~~~~│~~~~~~└──$\cdots$\\
│~~~~~~│~~~~~~~~~~~~~~~└─Kitaev/\\
│~~~~~~│~~~~~~~~~~~~~~~~~~~~~~└──$\cdots$\\
│~~~~~~└──Standard/\\
│~~~~~~~~~~~~~~~~~~├──Hubbard/\\
│~~~~~~~~~~~~~~~~~~│~~~~~~├─square/\\
│~~~~~~~~~~~~~~~~~~│~~~~~~│~~~~~~├──StdFace.def\\
│~~~~~~~~~~~~~~~~~~│~~~~~~│~~~~~~└──reference/\\
│~~~~~~~~~~~~~~~~~~│~~~~~~│~~~~~~~~~~~~~~~~~└──**.dat\\
│~~~~~~~~~~~~~~~~~~│~~~~~~└─triangular/\\
│~~~~~~~~~~~~~~~~~~│~~~~~~~~~~~~└──$\cdots$\\
│~~~~~~~~~~~~~~~~~~├──Kondo/\\
│~~~~~~~~~~~~~~~~~~│~~~~~~└─chain/\\
│~~~~~~~~~~~~~~~~~~│~~~~~~~~~~~~└──$\cdots$\\
│~~~~~~~~~~~~~~~~~~└──Spin/\\
│~~~~~~~~~~~~~~~~~~~~~~~~~~~~~~├─HeisenbergChain/\\
│~~~~~~~~~~~~~~~~~~~~~~~~~~~~~~│~~~~~~└──$\cdots$\\
│~~~~~~~~~~~~~~~~~~~~~~~~~~~~~~├─HeisenbergSquare/\\
│~~~~~~~~~~~~~~~~~~~~~~~~~~~~~~│~~~~~~└──$\cdots$\\
│~~~~~~~~~~~~~~~~~~~~~~~~~~~~~~└─Kagome/\\
│~~~~~~~~~~~~~~~~~~~~~~~~~~~~~~~~~~~~~└──$\cdots$\\
└──src/\\
~~~~~~~~~~~├──**.c\\
~~~~~~~~~~~├──**.h\\
~~~~~~~~~~~├──makefile\_src\\
~~~~~~~~~~~├──include/\\
~~~~~~~~~~~│~~~~~~~└──**.h\\
~~~~~~~~~~~├──pfapack/\\
~~~~~~~~~~~│~~~~~~~├──makefile\_pfapack\\
~~~~~~~~~~~│~~~~~~~└──**.f\\
~~~~~~~~~~~└──sfmt/\\
~~~~~~~~~~~~~~~~~~~├──makefkie\_sfmt\\
~~~~~~~~~~~~~~~~~~~├──**.c\\
~~~~~~~~~~~~~~~~~~~└──**.h\\

\newpage
\section{Basic usage}
mVMC has two executables;
one generates detailed input files and the other perform the main calculation with these files.
Here, the basic flows of calculations by using these programs.

\begin{enumerate}

\item  Prepare a minimal input file

You can choose a model (the Heisenberg model, the Hubbard model, etc.) and 
a lattice (the square lattice, the triangular lattice, etc.) from ones provided;
you can specify some parameters (such as the first/second nearest neighbor hopping integrals,
the on-site Coulomb integral, etc.) for them.
The input file format is described in the Sec. \ref{Ch:HowToStandard}.

\item  Run

Run a executable \verb|vmcdry.out| in terminal by specifing
the name of input file written in previous step.
MPI is not used in this step.

\verb|$ |\textit{Path}\verb|/vmcdry.out |\textit{Input\_file\_name}

Then, run a executable \verb|vmc.out| with 
the {\it namelist} file generated in the previous step.

\verb|$ mpiexec -np |\textit{number\_of\_processes}\verb| |\textit{Path}\verb|/vmcdry.out namelist.def|

When you use a queuing system in workstations or super computers, 
sometimes the number of processes is specified as an argument for the job-submitting command.
If you need more information, please refer manuals for your system. 

\item Watch calculation logs

Log files are outputted in the ``output" folder which is automatically made in the directory for a calculation scenario.
The details of output files are shown in Sec. \ref{Sec:outputfile}.

\item Results

If the calculation is finished normally, the result files are outputted in  the ``output" folder. The details of output files are shown in Sec. \ref{Sec:outputfile}.

\end{enumerate}

\begin{screen}
\Large 
{\bf Tips}
\normalsize

{\bf The number of threads for OpenMP}

If you specify the number of OpenMP threads for mVMC,
you should set it as follows (in case of 16 threads) before the running:
\begin{verbatim}
export OMP_NUM_THREADS=16
\end{verbatim}

\end{screen}

\section{Printing version ID}

By using \verb|-v| option as follows, 
you can check which version of mVMC you are using.

\begin{verbatim}
$ PATH/vmcdry.out -v
\end{verbatim}
