% !TEX root = userguide_jp.tex
%----------------------------------------------------------
\appendix
\chapter{非制限Hartree-Fock近似プログラム}
\label{Ch:URHF}
mVMCでは補助プログラムとして、多変数変分モンテカルロ法のペア軌道$f_{ij}$の初期値を非制限Hartree-Fock(UHF)近似から与えるためのプログラムを用意しています(対応関係は\ref{sec:PuffAndSlater}を参照)。なお、本プログラムは遍歴電子系を対象としており、スピン系、近藤系では正しく動作しません。

\section{概要}
UHF近似では揺らぎ$\delta A \equiv A-\langle A \rangle$の一次までを考慮することで、二体項を一体項へと近似します。
たとえば、サイト間クーロン相互作用
\begin{equation}
{\cal H}_V = \sum_{i,j, \sigma, \sigma'}V_{ij} n_ {i\sigma}n_{j\sigma'}
\end{equation}
について考えます。簡単化のため、$i\equiv (i, \sigma)$, $j\equiv (j, \sigma')$とすると相互作用の項は揺らぎの二次を落とすことで、
\begin{eqnarray}
n_ {i}n_{j} &=& (\langle n_{i} \rangle +\delta n_i) (\langle n_{j} \rangle +\delta n_j) - \left[ \langle c_{i}^{\dag}c_j \rangle +\delta (c_{i}^{\dag}c_j ) \right] \left[ \langle c_{j}^{\dag}c_i \rangle +\delta (c_{j}^{\dag}c_i )\right] \nonumber\\
&\sim&\langle n_{i} \rangle n_j+\langle n_{j} \rangle  n_i - \langle c_{i}^{\dag}c_j \rangle  c_{j}^{\dag}c_i  -  \langle c_{j}^{\dag}c_i \rangle c_{i}^{\dag}c_j 
-\langle n_{i} \rangle \langle n_j \rangle +  \langle c_{j}^{\dag}c_i \rangle \langle c_{i}^{\dag}c_j \rangle 
\end{eqnarray}
と近似されます。このような形式で、その他の相互作用についても近似を行うことで、一体問題に帰着させることができます。
計算では、上記の各平均値がself-consistentになるまで計算を行います。

\subsection{ソースコード}
ソースコード一式は\verb|src/ComplexUHF/src|以下に入っています。
\subsection{コンパイル方法}
コンパイルはmVMCのコンパイルと同様にmVMCのフォルダ直下で
\begin{verbatim}
$ make mvmc
\end{verbatim}
を実行することで行われます。コンパイルが終了すると、
\verb|src/ComplexUHF/src|に実行ファイル\verb|UHF|が作成されます。

\subsection{必要な入力ファイル}
\subsubsection{入力ファイル指定用ファイル (namelsit.def)}
UHFで指定するファイルは以下のファイルです。
\verb|namelist.def|は\ref{Subsec:InputFileList}で定義されているファイルと同じ様式です。\\
\begin{itemize}
\item{\verb|ModPara|}
\item{\verb|LocSpin|}
\item{\verb|Trans|}
\item{\verb|CoulombIntra|}
\item{\verb|CoulombInter|}
\item{\verb|Hund|}
\item{\verb|PairHop|}
\item{\verb|Exchange|}
\item{\verb|Orbital|/\verb|OrbitalAntiParallel|}
\item{\verb|OrbitalParallel|}
\item{\verb|OrbitalGeneral|}
\item{\verb|Initial|}
\end{itemize}
基本的にはmVMCと同じファイルとなりますが、
 \begin{itemize}
 \item{\verb|ModPara|ファイルで指定されるパラメータ}
 \item{\verb|Initial|ファイルの追加}
 \end{itemize}
がmVMCと異なります。以下、その詳細を記載します。


\subsubsection{ModParaファイルで指定するパラメータ}
UHFで指定するパラメータは以下のパラメータです。
\begin{itemize}
\item{\verb|Nsite|}
\item{\verb|Ne|}
\item{\verb|Mix|}
\item{\verb|EPS|}
\item{\verb|IterationMax|}
\end{itemize}
\verb|Nsite|, \verb|Ne|はmVMCと共通のパラメータで、以下の三つがUHF独特のパラメータです。
\begin{itemize}
\item{\verb|Mix|}\\
linear mixingをdouble型で指定します。mix=1とすると完全に新しいGreen関数に置き換えられます。
\item{\verb|EPS|}\\
収束判定条件をint型で指定します。新しく計算されたGreen関数と一つ前のGreen関数の残差が$10^{-\verb|eps|}$の場合に、計算が打ち切られます。
\item{\verb|IterationMax|}\\
ループの最大数をint型で指定します。
\end{itemize}
なお、mVMCで使用するその他パラメータが存在する場合はWarningが標準出力されます(計算は中断せずに実行されます)。

\subsubsection{Initialファイル}
グリーン関数$G_{ij\sigma_1\sigma_2}\equiv \langle c_{i\sigma_1}^\dag c_{j\sigma_2}\rangle$の初期値を与えます。
ファイル様式は\verb|Trans|ファイルと同じで、$t_{ij\sigma_1\sigma_2}$の代わりに$G_{ij\sigma_1\sigma_2}$の値を記述します。
なお、値を指定しないグリーン関数には0が入ります。

\section{使用方法}
UHF自体はmVMCと同じように
\begin{verbatim}
$ UHF namelist.def
\end{verbatim}
で動きます。計算の流れは以下の通りです。
\begin{enumerate}
\item{ファイル読み込み}
\item{ハミルトニアンの作成}
\item{グリーン関数の計算 (self-consistentになるまで)}
\item{$f_{ij}$、各種ファイルの出力}
\end{enumerate}
計算後に出力されるファイルおよび出力例は以下の通りです。
\begin{itemize}
\item{zvo\_result.dat:}  エネルギーと粒子数が出力されます。\\
\begin{minipage}{13cm}
\begin{screen}
\begin{verbatim}
 energy -15.2265348135
 num    36.0000000000
\end{verbatim}
\end{screen}
\end{minipage}
\item{zvo\_check.dat:} イタレーションのステップ数、グリーン関数の残差の絶対値の平均、収束過程のエネルギー、粒子数を順に出力します。

\begin{minipage}{13cm}
\begin{screen}
\begin{verbatim}
 0  0.004925645652 -544.963484605164 36.000000
 1  0.002481594941 -278.304285708488 36.000000
 2  0.001274395448 -147.247026925130 36.000000
 3  0.000681060599 -82.973664527606 36.000000
...
\end{verbatim}
\end{screen}
\end{minipage}

\item{zvo\_UHF\_cisajs.dat:} 収束した一体グリーン関数$G_{ij\sigma_1\sigma_2}\equiv\langle c_{i\sigma_1}^{\dag}c_{j\sigma_2}\rangle$。\\
全成分について$i, \sigma_1, j, \sigma_2, {\rm Re}\left[G_{ij\sigma_1\sigma_2}\right], {\rm Im}\left[G_{ij\sigma_1\sigma_2}\right]$の順に出力されます。

\begin{minipage}{13cm}
\begin{screen}
\begin{verbatim}
    0    0    0    0 0.5037555283 0.0000000000
    0    0    0    1 0.4610257618 0.0003115503
    0    1    0    0 0.4610257618 -0.0003115503
    0    1    0    1 0.4962444717 0.0000000000
 ...
\end{verbatim}
\end{screen}
\end{minipage}    
    
\item{zvo\_eigen.dat:} 収束したハミルトニアンの固有値が低エネルギー順に出力されます。\\
\begin{minipage}{13cm}
\begin{screen}
\begin{verbatim}
 1  -2.9425069199
 2  -2.9425069198
 3  -1.5005359205 
 ...
\end{verbatim}
\end{screen}
\end{minipage}

\item{zvo\_gap.dat:} 全電子数を$N_{\rm tot}$とした場合に、$\Delta E= E(N_{\rm tot}+1)-E(N_{\rm tot})$が出力されます。

\begin{minipage}{13cm}
\begin{screen}
\begin{verbatim}
  5.2208232631
\end{verbatim}
\end{screen}
\end{minipage}

\item{zvo\_orbital\_opt.dat:} スレータ行列式から生成した$f_{ij}$。\verb|InOrbital|,\verb|InOrbitalAntiParallel|, \verb|InOrbitalParallel|,\verb|InOrbitalAntiGeneral|ファイルと同じ形式のファイルが出力されます。
$f_{ij}$が\verb|Orbital|, \verb|OrbitalAntiParallel|, \verb|OrbitalParallel|,\verb|OrbitalAntiGeneral|ファイルを参照し計算され、同種のパラメータについては平均化した値が採用されます。

\end{itemize}



%----------------------------------------------------------
