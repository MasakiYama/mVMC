% !TEX root = userguide_jp.tex
%----------------------------------------------------------
\newpage
\section{詳細入力ファイル}
\label{Ch:HowToExpert}

\tr{ここではmVMCで使用する詳細入力ファイル(*def)のフォーマットに関して説明します。
入力ファイルの種別は以下の6つで分類されます。なお、キーワードの後にある括弧内に記載されているファイル名はvmcdry.outにより作成されるファイル名を表します。}
\begin{description}
\item[(1)~リスト:]
~\\{\bf キーワード指定なし (namelist.def)}:
使用するinput fileの名前のリストを書きます。なお、ファイル名は任意に指定することができます。
\item[(2)~基本パラメータ:]
~\\{\bf ModPara (modpara.def)}: 計算時に必要な基本的なパラメーター(サイトの数、電子数など)を設定します。
~\\{\bf LocSpin (locspn.def)}: 局在スピンの位置を設定します。
\item[\tr{(3)~ハミルトニアン:}] 
~\\電子系の表式で記載されるハミルトニアン
\begin{eqnarray}
{\cal H}&=&{\cal H}_T+{\cal H}_U+{\cal H}_V+{\cal H}_H+{\cal H}_E+{\cal H}_P+{\cal H}_I,\\
{\cal H}_T&=&\tr{-}\sum_{i, j}\sum_{\sigma_1, \sigma2}t_{ij\sigma_1\sigma_2} c_{i\sigma_1}^{\dag}c_{j\sigma_2},\\
{\cal H}_U&=&\sum_{i} U_i n_ {i \uparrow}n_{i \downarrow},\\
{\cal H}_V&=&\sum_{i,j} V_{ij}n_ {i}n_{j},\\
{\cal H}_H&=&\tr{-}\sum_{i,j}J_{ij}^{\rm Hund} (n_{i\uparrow}n_{j\uparrow}+n_{i\downarrow}n_{j\downarrow}),\\
{\cal H}_E&=&\sum_{i,j}J_{ij}^{\rm Ex} (c_ {i \uparrow}^{\dag}c_{j\uparrow}c_{j \downarrow}^{\dag}c_{i  \downarrow}+c_ {i \downarrow}^{\dag}c_{j\downarrow}c_{j \uparrow}^{\dag}c_{i  \uparrow}),\\
{\cal H}_P&=&\sum_{i,j}J_{ij}^{\rm Pair} c_ {i \uparrow}^{\dag}c_{j\uparrow}c_{i \downarrow}^{\dag}c_{j  \downarrow},\\
{\cal H}_I&=&\sum_{i,j,k,l}\sum_{\sigma_1,\sigma_2, \sigma_3, \sigma_4}
I_{ijkl\sigma_1\sigma_2\sigma_3\sigma_4}c_{i\sigma_1}^{\dagger}c_{j\sigma_2}c_{k\sigma_3}^{\dagger}c_{l\sigma_4}, 
\end{eqnarray}
について設定します。ここで、$n_{i \sigma}=c_{i\sigma}^{\dag}c_{i\sigma}$はスピン$\sigma$を持つサイト$i$の電子密度演算子を、$n_i=n_{i\uparrow}+n_{i\downarrow}$はサイト$i$の電子密度演算子をそれぞれ表します。 各ハミルトニアンのパラメータは以下のファイルで指定します。
~\\{\bf Trans (trans.def)}: 
${\cal H}_T$内の$t_{ij\sigma_1\sigma_2}$を指定します。
~\\{\bf CoulombIntra (coulombintra.def)}: 
${\cal H}_U$内の$U_i$を指定します。
~\\{\bf CoulombInter (coulombinter.def)}: 
${\cal H}_V$内の$V_{ij}$を指定します。
~\\{\bf Hund (hund.def)}: 
${\cal H}_H$内の$J_{ij}^{\rm Hund}$を指定します。
~\\{\bf Exchange (exchange.def)}:
${\cal H}_E$内の$J_{ij}^{\rm Ex}$を指定します。
~\\{\bf PairHop}:  ${\cal H}_P$内の$J_{ij}^{\rm Pair}$を指定します。
~\\{\bf InterAll}: ${\cal H}_I$内の$I_{ijkl\sigma_1\sigma_2\sigma_3\sigma_4}$を指定します。

\item[(4)~最適化対象変分パラメータ:] 
~\\ \tr{最適化する変分パラメータを指定します}。変分波動関数は
\begin{eqnarray}
|\psi \rangle &=& {\cal P}_G{\cal P}_J{\cal P}_{d-h}^{(2)}{\cal P}_{d-h}^{(4)}{\cal L}^S{\cal L}^K{\cal L}^P |\phi_{\rm pair} \rangle,\\
{\cal P}_G&=&\exp\left[ \sum_i g_i n_{i\uparrow} n_{i\downarrow} \right],\\
{\cal P}_J&=&\exp\left[\frac{1}{2} \sum_{i\neq j} v_{ij} (n_i-1)(n_j-1)\right],\\
{\cal P}_{d-h}^{(2)}&=& \exp \left[ \sum_t \sum_{n=0}^2 (\alpha_{2nt}^d \sum_{i}\xi_{i2nt}^d+\alpha_{2nt}^h \sum_{i}\xi_{i2nt}^h)\right],\\
{\cal P}_{d-h}^{(4)}&=& \exp \left[ \sum_t \sum_{n=0}^4 (\alpha_{4nt}^d \sum_{i}\xi_{i4nt}^d+\alpha_{4nt}^h \sum_{i}\xi_{i4nt}^h)\right],\\
{\cal L}_S&=&\frac{2S+1}{8 \pi^2}\int d\Omega P_s(\cos \beta) \hat{R}(\Omega),\\
{\cal L}_K&=&\frac{1}{N_s}\sum_{{\bm R}}e^{i {\bm K} \cdot{\bm R} } \hat{T}_{\bm R},\\
{\cal L}_P&=&\sum_{\alpha}p_{\alpha} \hat{G}_{\alpha},
\end{eqnarray}
で与えられます。ここで、 $\Omega=(\alpha, \beta, \gamma)$はオイラー角、$\hat{R}(\Omega)$は回転演算子、$P_S(x)$は$S$次のルジャンドル多項式、${\bm K}$は全運動量、$\hat{T}_{\bm R}$は並進ベクトル${\bm R}$に対応する並進演算子、$\hat{G}_{\alpha}$は格子の点群演算子、$p_\alpha$はパリティをそれぞれ表します。ダブロン・ホロン相関因子に関する詳細は文献\cite{Tahara2008}の説明を参照してください。また、一体部分は実空間のペア関数
\begin{equation}
|\phi_{\rm pair} \rangle = \left[\sum_{i, j=1}^{N_s} f_{ij}c_{i\uparrow}^{\dag}c_{j\downarrow}^{\dag} \right]^{N/2}|0 \rangle,
\end{equation}
を用いた波動関数で表されます。ここで$N$は全電子数、$N_s$は全サイト数です。
最適化する変分パラメータは以下のファイルを用いて指定します(${\cal L}_S$は{\bf ModPara}ファイルでパラメータの指定をします)。
~\\{\bf Gutzwiller (gutzwilleridx.def)}: ${\cal P}_G$のうち、最適化の対象とする変分パラメータ$g_i$を指定します。
~\\{\bf Jastrow (jastrowidx.def)}: ${\cal P}_J$のうち、最適化の対象とする変分パラメータ$v_{ij}$を指定します。
~\\{\bf DH2}:  ${\cal P}_{d-h}^{(2)}$で表される2サイトのダブロン・ホロン相関因子を指定します。
~\\{\bf DH4}:  ${\cal P}_{d-h}^{(4)}$で表される4サイトのダブロン・ホロン相関因子を指定します。
~\\{\bf Orbital (orbitalidx.def)}: ペア軌道$|\phi_{\rm pair} \rangle$を設定します。
~\\{\bf TransSym (qptransidx.def)}: 運動量射影${\cal L}_K$と格子対称性射影${\cal L}_P$に関する指定を行います。
\item[(5)~変分パラメータ初期値:]
~\\ 変分パラメータに関する初期値を与えます。キーワード指定されない場合には$0$が初期値として設定されます。
~\\{\bf InGutzwiller}: ${\cal P}_G$内の変分パラメータ$g_i$の初期値を設定します。
~\\{\bf InJastrow}: ${\cal P}_J$内の変分パラメータ$v_{ij}$の初期値を設定します。
~\\{\bf InDH2}:  ${\cal P}_{d-h}^{(2)}$内の2サイトのダブロン・ホロン相関因子$\alpha_{2nt}^{d(h)}$の初期値を設定します。
~\\{\bf InDH4}:  ${\cal P}_{d-h}^{(4)}$内の4サイトのダブロン・ホロン相関因子$\alpha_{4nt}^{d(h)}$の初期値を設定します。
~\\{\bf InOrbital}: ペア軌道$|\phi_{\rm pair} \rangle$の$ f_{ij}$に関する初期値を設定します。


\item[(6)~出力:]
~\\{\bf OneBodyG (greenone.def)}:出力する一体Green関数を指定します。

 {\bf TwoBodyG (greentwo.def)}:出力する二体Green関数を指定します。

\end{description}
%%%%%%%%%%%%%%%%%%%%%%
\newpage
~\subsection{入力ファイル指定用ファイル(\tr{namelist.def})}
\label{Subsec:InputFileList}
計算で使用する入力ファイル一式を指定します。ファイル形式に関しては、以下のようなフォーマットをしています。\\
\begin{minipage}{10cm}
\begin{screen}
\begin{verbatim}
ModPara  modpara.def
LocSpin  zlocspn.def
Trans    ztransfer.def
InterAll zinterall.def
Orbital orbitalidx.def
OneBodyG zcisajs.def
TwoBodyG	zcisajscktaltdc.def
\end{verbatim}
\end{screen}
\end{minipage}
\\
\subsubsection{ファイル形式}
[string01]~[string02]
\subsubsection{パラメータ}
 \begin{itemize}
   \item  $[$string01$]$
   
   {\bf 形式 :} string型 (固定)
   
   {\bf 説明 :} キーワードを指定します。
   
   \item  $[$string02$]$
   
    {\bf 形式 :} string型 

   {\bf 説明 :} キーワードにひも付けられるファイル名を指定します(任意)。
 \end{itemize}
\subsubsection{使用ルール}
本ファイルを使用するにあたってのルールは以下の通りです。
\begin{itemize}
\item キーワードを記載後、半角空白(複数可)を開けた後にファイル名を書きます。ファイル名は自由に設定できます。
\item ファイル読込用キーワードはTable\ref{Table:Defs}により指定します。
\item 必ず指定しなければいけないキーワードはModPara, LocSpin, Orbital, \tr{TransSym}です。それ以外のキーワードについては、指定がない場合はデフォルト値が採用されます(変分パラメータについては最適化されず、固定する設定となります)。詳細は各ファイルの説明を参照してください。
\item 各キーワードは順不同に記述できます。
\item 指定したキーワード、ファイルが存在しない場合はエラー終了します。
\item $\#$で始まる行は読み飛ばされます。
\end{itemize}

 \begin{table*}[h!]
\begin{center}
  \begin{tabular}{|ll|} \hline
           Keywords     & 対応するファイルの概要       \\   \hline\hline
           ModPara$^*$       &  計算用のパラメータを指定します。       \\ \hline 
           %%%%%%%%%%%%%%%%%%  
           LocSpin$^*$         &  局在・遍歴スピンを指定します。\\ 
           Trans       &   一般的な一体相互作用を指定します。\\
           InterAll  &   一般的な二体相互作用を指定します。\\  
           CoulombIntra  &   内部クーロン相互作用を指定します。\\  
           CoulombInter  &   サイト間クーロン相互作用を指定します。\\  
           Hund  &   フント結合を指定します。\\  
           PairHop  &  ペアホッピング相互作用を指定します。 \\  
           Exchange  &  交換相互作用を指定します。\\  \hline
           %%%%%%%%%%%%%%%%%%
           Gutzwiller & 最適化するGutzwiller因子を設定します。\\
           Jastrow & 最適化する電荷Jastrow因子を指定します。\\
           DH2 & 最適化する2サイトダブロン・ホロン相関因子を指定します。\\
           DH4 & 最適化する4サイトダブロン・ホロン相関因子を指定します。\\
           Orbital$^*$  & ペア軌道因子を指定します。\\
           TransSym$^*$  & 並進・格子対称演算子を設定します。\\ \hline
           %%%%%%%%%%%%%%%%%%
           InGutzwiller & Gutzwiller因子の初期値を設定します。\\
           InJastrow & 電荷Jastrow因子の初期値を設定します。\\
           InDH2 & 2サイトダブロン・ホロン相関因子の初期値を設定します。\\
           InDH4 & 4サイトダブロン・ホロン相関因子の初期値を設定します。\\
           InOrbital & ペア軌道因子の初期値を設定します。\\ \hline
           %%%%%%%%%%%%%%%%%%
           OneBodyG         &   出力する一体グリーン関数を指定します。\\   
           TwoBodyG &   出力する二体グリーン関数を指定します。\\   \hline
  \end{tabular}
\end{center}
\caption{*.defファイルの一覧。 * が付いているファイルは実行時に必ず必要となります。}
\label{Table:Defs}
\end{table*}%

\newpage
%----------------------------------
\subsection{ModParaファイル (\tr{modpara.def})}
\label{Subsec:modpara}
計算で使用するパラメータを指定します。以下のようなフォーマットをしています。\\
\begin{minipage}{10cm}
\begin{screen}
\begin{verbatim}
--------------------
Model_Parameters 
--------------------
VMC_Cal_Parameters
--------------------
CDataFileHead  zvo
CParaFileHead  zqp
--------------------
NVMCCalMode    0
NLanczosMode   0
--------------------
NDataIdxStart  1
NDataQtySmp    1
--------------------
Nsite          16
Nelectron      8
NSPGaussLeg    1
NSPStot        0
NMPTrans       1
NSROptItrStep  1200
NSROptItrSmp   100
DSROptRedCut   0.001
DSROptStaDel   0.02
DSROptStepDt   0.02
NVMCWarmUp     10
NVMCInterval   1
NVMCSample     1000
NExUpdatePath  0
RndSeed        11272
NSplitSize     1
NStore         1  
\end{verbatim}
\end{screen}
\end{minipage}

\subsubsection{ファイル形式}
以下のように行数に応じ異なる形式をとります。
 \begin{itemize}
   \item  1 - 5行:  ヘッダ(何が書かれても問題ありません)。
   \item  \tr{6行:  [string01]~[string02]}
   \item  \tr{7行:  [string03]~[string04]}
   \item  \tr{8行:} ヘッダ(何が書かれても問題ありません)
   \item  \tr{9行以降: [string05]~[int01] (もしくは[double01])}
  \end{itemize}
各項目の対応関係は以下の通りです。
\begin{itemize}

   \item  \tr{$[$string01$]$}
   
   {\bf 形式 :} string型 (空白不可)

  {\bf 説明 :} アウトプットファイルのヘッダを表すためのキーワード。何を指定しても問題ありません。
   
   \item  $[$string02$]$
   
   {\bf 形式 :} string型 (空白不可)

  {\bf 説明 :} アウトプットファイルのヘッダ。例えば、一体のGreen関数の出力ファイル名が{\bf xxx\_cisajs.dat}として出力されます(xxxに指定した文字が記載)。

   \item  \tr{$[$string03$]$}
   
   {\bf 形式 :} string型 (空白不可)

  {\bf 説明 :} 最適化された変分パラメータの出力ファイル名のヘッダを表すためのキーワード。何を指定しても問題ありません。
   
   \item  \tr{$[$string04$]$}
   
   {\bf 形式 :} string型 (空白不可)

  {\bf 説明 :} 最適化された変分パラメータの出力ファイル名のヘッダ。最適化された変分パラメータが{\bf xxx\_opt.dat}ファイルとして出力されます(xxxに指定した文字が記載)。

   \item  \tr{$[$string05$]$}
   
   {\bf 形式 :} string型 (固定)

  {\bf 説明 :} キーワードの指定を行います。

   \item  \tr{$[$int01$]$ ($[$double01$]$ )}
   
   {\bf 形式 :} int (double)型 (空白不可)

  {\bf 説明 :} キーワードでひも付けられるパラメータを指定します。

  \end{itemize}

\subsubsection{使用ルール}
本ファイルを使用するにあたってのルールは以下の通りです。
\begin{itemize}
\item 9行目以降ではキーワードを記載後、半角空白(複数可)を開けた後に整数値を書きます。
\item \tr{9行目以降では"-"で始まる行は読み込まれません}。
\end{itemize}

~\subsubsection{キーワード}
 \begin{itemize}
 
 \item  \verb|NVMCCalMode|

 {\bf 形式 :} int型 (デフォルト値 = 0)

{\bf 説明 :} [0] 変分パラメータの最適化、[1] 1 体・2 体のグリーン関数の計算。
 
% \item  \verb|NLanczosMode|

% {\bf 形式 :} int型 (デフォルト値 = 0)

%{\bf 説明 :} [0] 何もしない、[1] Single Lanczos Step でエネルギーまで計算、[2] Single Lanczos Step で1 体・2 体のグリーン関数まで計算(条件: 1, 2 は\verb|NVMCCalMode| = 1のみ使用可能。また, pair hopping項, exchange 項がハミルトニアンに含まれる場合は使用できません)。
 
 \item  \verb|NDataIdxStart|

 {\bf 形式 :} int型 (デフォルト値 = 0)

{\bf 説明 :} 出力ファイルの付加番号。\verb|NVMCCalMode|= 0 の場合は\verb|NDataIdxStart|が出力され、 \verb|NVMCCalMode| = 1 の場合は、\verb|NDataIdxStart|から連番で\verb|NDataQtySmp|個のファイルを出力します。
   
 \item  \verb|NDataQtySmp|

 {\bf 形式 :} int型 (デフォルト値 = 1)

{\bf 説明 :} 出力ファイルのセット数。 \verb|NVMCCalMode| = 1 の場合に使用します。

 \item  \verb|Nsite|

{\bf 形式 :} int型 (1以上、必須)

{\bf 説明 :} サイト数を指定する整数。  

\item  \verb|Nelectron|

{\bf 形式 :} {int型 (1以上、必須)}

{\bf 説明 :} $\uparrow$電子の個数。$S_z=0$部分空間で計算するので、$\uparrow$電子と$\downarrow$電子の個数は等しい。

 \item  \verb|NSPGaussLeg|

{\bf 形式 :} {int型 (1以上、デフォルト値 = 8)}

{\bf 説明 :} スピン量子数射影の$\beta$積分($S^y$回転)のGauss-Legendre求積法の分点数。

 \item  \verb|NSPStot|

{\bf 形式 :} int型 (0以上、デフォルト値 = 0)

{\bf 説明 :}  スピン量子数。

 \item  \verb|NMPTrans|

{\bf 形式 :} int型 (デフォルト値 = 1)

{\bf 説明 :} 
\verb|NMPTrans|の絶対値で並進・格子対称性の量子数射影の個数を指定する。負の場合は反周期境界条件を与える。TransSymファイルで指定した重みで上から\verb|NMPTrans|個まで使用する。\tr{射影を行わない場合は1に設定する必要があります。}

 \item  \verb|NSROptItrStep|

{\bf 形式 :} int型 (1以上、デフォルト値 = 1000)

{\bf 説明 :} 
SR 法で最適化する場合の全ステップ数。\verb|NVMCCalMode|=0の場合のみ使用されます。
 
 \item  \verb|NSROptItrSmp|

{\bf 形式 :} int型 (1以上数、デフォルト値 = \verb|NSROptItrStep|/10)

{\bf 説明 :} \verb|NSROptItrStep|ステップ中、最後の\verb|NSROptItrSmp|ステップでの各変分パラメータの平均値を最適値とする。\verb|NVMCCalMode|=0の場合のみ使用されます。

\item   \verb|DSROptRedCut|
   
{\bf 形式 :} double型 (デフォルト値 = 0.001)

{\bf 説明 :} SR 法安定化因子。手法論文\cite{Tahara2008}の$\varepsilon_{\rm wf}$に対応。

 \item  \verb|DSROptStaDel| 
   
 {\bf 形式 :} double型 (デフォルト値 = 0.02)

  {\bf 説明 :} SR 法安定化因子。手法論文\cite{Tahara2008}の$\varepsilon$に対応。
     
\item \verb|DSROptStepDt|

{\bf 形式 :} double型 

{\bf 説明 :} \tr{SR法で使用する刻み幅。手法論文\cite{Tahara2008}の$\Delta t$に対応。}
 
\item \verb|NVMCWarmUp|

{\bf 形式 :} int型 (1以上、デフォルト値=10)

{\bf 説明 :}マルコフ連鎖の空回し回数。

\item \verb|NVMCInterval|

{\bf 形式 :} int型 (1以上、デフォルト値=1)

{\bf 説明 :} サンプル間のステップ間隔。ローカル更新を\verb|Nsite|× \verb|NVMCInterval| 回行います。

\item \verb|NVMCSample|

{\bf 形式 :} int型 (1以上、デフォルト値=1000)

{\bf 説明 :} 期待値計算に使用するサンプル数。

\item \verb|NExUpdatePath|

{\bf 形式 :} int型 (0以上)

{\bf 説明 :} 電子系でローカル更新で2電子交換を[0] 認めない、[1] 認めるの設定をします。スピン系の場合には2に設定する必要があります。

\item \verb|RndSeed|

{\bf 形式 :} int型 

{\bf 説明 :} 乱数の初期seed。MPI 並列では各計算機に\verb|RndSeed|+my rank+1 で初期seed が与えられます。

 \item \verb|NSplitSize|

{\bf 形式 :} int型 (1以上、デフォルト値=1)

{\bf 説明 :} MPI内部並列を行う場合の並列数。

\item \verb|NStore|

{\bf 形式 :} int型 (1以上、デフォルト値=1)

{\bf 説明 :} 期待値$\langle O_k O_l \rangle$を計算するとき行列-行列積にして高速化するオプション(1で機能On、\tr{モンテカルロサンプリング数に応じてメモリの消費が増大します})。
 
 \end{itemize}


\newpage
%----------------------------------
\subsection{LocSpin指定ファイル(locspn.def)}
\label{Subsec:locspn}
局在スピンを指定します。以下のようなフォーマットをしています。\\
\begin{minipage}{10cm}
\begin{screen}
\begin{verbatim}
================================ 
NlocalSpin     6  
================================ 
========i_0LocSpn_1IteElc ====== 
================================ 
    0      1
    1      0
    2      1
    3      0
    4      1
    5      0
    6      1
    7      0
    8      1
    9      0
   10      1
   11      0
\end{verbatim}
\end{screen}
\end{minipage}


\subsubsection{ファイル形式}
以下のように行数に応じ異なる形式をとります。
 \begin{itemize}
   \item  1行:  ヘッダ(何が書かれても問題ありません)。
   \item  2行:   [string01]~[int01]
   \item  3-5行:  ヘッダ(何が書かれても問題ありません)。
   \item  6行以降:  [int02]~[int03]
  \end{itemize}
 \subsubsection{パラメータ}
 \begin{itemize}

 \item  $[$string01$]$

 {\bf 形式 :} string型 (空白不可)

{\bf 説明 :} 局在スピンの総数を示すキーワード(任意)。


  \item  $[$int01$]$

 {\bf 形式 :} int型 (空白不可)

{\bf 説明 :} 局在スピンの総数を指定する整数。

 
  \item  $[$int02$]$

 {\bf 形式 :} int型 (空白不可)

{\bf 説明 :} サイト番号を指定する整数。0以上\verb|Nsite|{未満}で指定します。

 
  \item  $[$int03$]$

 {\bf 形式 :} int型 (空白不可)

{\bf 説明 :} 遍歴電子か局在スピンかを指定する整数(\tr{0: 遍歴電子, 1: 局在スピン})。\\
 \end{itemize}

\subsubsection{使用ルール}
本ファイルを使用するにあたってのルールは以下の通りです。
\begin{itemize}
\item 行数固定で読み込みを行う為、ヘッダの省略はできません。
\item $[$int01$]$と$[$int03$]$で指定される局在電子数の総数が異なる場合はエラー終了します。
\item $[$int02$]$の総数が全サイト数と異なる場合はエラー終了します。
\item $[$int02$]$が全サイト数以上もしくは負の値をとる場合はエラー終了します。
\end{itemize}


\newpage
\subsection{Trans指定ファイル(\tr{trans.def})}
\label{Subsec:Trans}
\tr{一般的な一体相互作用をハミルトニアンに付け加え、ハミルトニアン中の相互作用パラメータ$t_{ij\sigma_1\sigma2}$を指定します。付け加える項は以下で与えられます。}
\begin{align}
{\cal H}_T =-\sum_{ij\sigma_1\sigma2} t_{ij\sigma_1\sigma2}c_{i\sigma_1}^{\dag}c_{j\sigma_2}
\end{align}
\tr{なお、ver. 1.0では$S_z=0$の場合の計算のみ対応しているため、$\sigma_1=\sigma_2$となるようにしてください。}

以下にファイル例を記載します。\\
\begin{minipage}{12.5cm}
\begin{screen}
\begin{verbatim}
======================== 
NTransfer      24  
======================== 
========i_j_s_tijs====== 
======================== 
    0     0     2     0   1.000000  0.000000
    2     0     0     0   1.000000  0.000000
    0     1     2     1   1.000000  0.000000
    2     1     0     1   1.000000  0.000000
    2     0     4     0   1.000000  0.000000
    4     0     2     0   1.000000  0.000000
    2     1     4     1   1.000000  0.000000
    4     1     2     1   1.000000  0.000000
    4     0     6     0   1.000000  0.000000
    6     0     4     0   1.000000  0.000000
    4     1     6     1   1.000000  0.000000
    6     1     4     1   1.000000  0.000000
    6     0     8     0   1.000000  0.000000
    8     0     6     0   1.000000  0.000000
…
\end{verbatim}
\end{screen}
\end{minipage}

\subsubsection{ファイル形式}
以下のように行数に応じ異なる形式をとります。
 \begin{itemize}
   \item  1行:  ヘッダ(何が書かれても問題ありません)。
   \item  2行:   [string01]~[int01]
   \item  3-5行:  ヘッダ(何が書かれても問題ありません)。
   \item  6行以降: [int02]~~[int03]~~[int04]~~[int05]~~[double01]~~[double02] 
  \end{itemize}
\subsubsection{パラメータ}
 \begin{itemize}

   \item  $[$string01$]$
   
    {\bf 形式 :} string型 (空白不可)

   {\bf 説明 :} 定義するパラメータの総数のキーワード名を指定します(任意)。

   \item  $[$int01$]$
   
    {\bf 形式 :} int型 (空白不可)

   {\bf 説明 :} 定義するパラメータの総数を指定します。

  \item  $[$int02$]$, $[$int04$]$

 {\bf 形式 :} int型 (空白不可)

{\bf 説明 :} サイト番号を指定する整数。0以上\verb|Nsite|{未満}で指定します。
 
  \item  $[$int03$]$, $[$int05$]$

 {\bf 形式 :} int型 (空白不可)

{\bf 説明 :} スピンを指定する整数。\\
0: アップスピン\\
1: ダウンスピン\\
を選択することが出来ます。


 \item  $[$double01$]$
   
   {\bf 形式 :} double型 (空白不可)

  {\bf 説明 :}  $t_{ij\sigma_1\sigma_2}$の実部を指定します。

 \item  $[$double02$]$
   
   {\bf 形式 :} double型 (空白不可)

  {\bf 説明 :}  $t_{ij\sigma_1\sigma_2}$の虚部を指定します。
\end{itemize}

\subsubsection{\tr{使用ルール}}
本ファイルを使用するにあたってのルールは以下の通りです。
\begin{itemize}
\item 行数固定で読み込みを行う為、ヘッダの省略はできません。
\item \tr{空行は許されません}。
\item $[$int01$]$と定義されているTrasferの総数が異なる場合はエラー終了します。
\item $[$int02$]$-$[$int05$]$を指定する際、範囲外の整数を指定した場合はエラー終了します。
\item Hamiltonianがエルミートという制限から$t_{ij\sigma_1\sigma_2}=t_{ji\sigma_2\sigma_1}^{\dagger}$の関係を満たす必要があります。
\end{itemize}

\newpage
\subsection{InterAll指定ファイル}
\label{Subsec:interall}
\tr{一般的な二体相互作用をハミルトニアンに付け加え、ハミルトニアン中の相互作用パラメータを指定します。付け加える項は以下で与えられます。}
\begin{equation}
{\cal H}_{I}=\sum_{i,j,k,l}\sum_{\sigma_1,\sigma_2, \sigma_3, \sigma_4}
I_{ijkl\sigma_1\sigma_2\sigma_3\sigma_4}c_{i\sigma_1}^{\dagger}c_{j\sigma_2}c_{k\sigma_3}^{\dagger}c_{l\sigma_4}
\end{equation}
\tr{なお、ver. 1.0では$S_z=0$の場合の計算のみ対応しているため、$\sigma_1=\sigma_2$、$\sigma_3=\sigma_4$となるようにしてください。}
\tr{また、スピン系には未対応です。}

以下にファイル例を記載します。\\
\begin{minipage}{12.5cm}
\begin{screen}
\begin{verbatim}
====================== 
NInterAll      36  
====================== 
========zInterAll===== 
====================== 
0    0    0    1    1    1    1    0   0.50  0.0
0    1    0    0    1    0    1    1   0.50  0.0
0    0    0    0    1    0    1    0   0.25  0.0
0    0    0    0    1    1    1    1  -0.25  0.0
0    1    0    1    1    0    1    0  -0.25  0.0
0    1    0    1    1    1    1    1   0.25  0.0
2    0    2    1    3    1    3    0   0.50  0.0
2    1    2    0    3    0    3    1   0.50  0.0
2    0    2    0    3    0    3    0   0.25  0.0
2    0    2    0    3    1    3    1  -0.25  0.0
2    1    2    1    3    0    3    0  -0.25  0.0
2    1    2    1    3    1    3    1   0.25  0.0
4    0    4    1    5    1    5    0   0.50  0.0
4    1    4    0    5    0    5    1   0.50  0.0
4    0    4    0    5    0    5    0   0.25  0.0
4    0    4    0    5    1    5    1  -0.25  0.0
4    1    4    1    5    0    5    0  -0.25  0.0
4    1    4    1    5    1    5    1   0.25  0.0
…
\end{verbatim}
\end{screen}
\end{minipage}

\subsubsection{ファイル形式}
以下のように行数に応じ異なる形式をとります。
 \begin{itemize}
   \item  1行:  ヘッダ(何が書かれても問題ありません)。
   \item  2行:   [string01]~[int01]
   \item  3-5行:  ヘッダ(何が書かれても問題ありません)。
   \item  6行以降:
   [int02]~[int03]~[int04]~[int05]~[int06]~[int07]~[int08]~[int09]~[double01]~[double02] 
  \end{itemize}
\subsubsection{パラメータ}
 \begin{itemize}

   \item  $[$string01$]$
   
    {\bf 形式 :} string型 (空白不可)

   {\bf 説明 :} 二体相互作用の総数のキーワード名を指定します(任意)。

   \item  $[$int01$]$
   
    {\bf 形式 :} int型 (空白不可)

   {\bf 説明 :} 二体相互作用の総数を指定します。

  \item  $[$int02$]$, $[$int04$]$, $[$int06$]$, $[$int08$]$

 {\bf 形式 :} int型 (空白不可)

{\bf 説明 :} サイト番号を指定する整数。0以上\verb|Nsite|{未満}で指定します。
 
  \item  $[$int03$]$, $[$int05$]$, $[$int07$]$, $[$int09$]$

 {\bf 形式 :} int型 (空白不可)

{\bf 説明 :} スピンを指定する整数。\\
0: アップスピン\\
1: ダウンスピン\\
を選択することが出来ます。


 \item  $[$double01$]$
   
   {\bf 形式 :} double型 (空白不可)

  {\bf 説明 :}  $I_{ijkl\sigma_1\sigma_2\sigma_3\sigma_4}$の実部を指定します。

 \item  $[$double02$]$
   
   {\bf 形式 :} double型 (空白不可)

  {\bf 説明 :}  $I_{ijkl\sigma_1\sigma_2\sigma_3\sigma_4}$の虚部を指定します。
\end{itemize}


\subsubsection{\tr{使用ルール}}
本ファイルを使用するにあたってのルールは以下の通りです。
\begin{itemize}
\item 行数固定で読み込みを行う為、ヘッダの省略はできません。
\item ハミルトニアンがエルミートという制限から$I_{ijkl\sigma_1\sigma_2\sigma_3\sigma_4}=I_{lkji\sigma_4\sigma_3\sigma_2\sigma_1}^{\dag}$の関係を満たす必要があります。
\item $[$int01$]$と定義されているInterAllの総数が異なる場合はエラー終了します。
\item $[$int02$]$-$[$int09$]$を指定する際、範囲外の整数を指定した場合はエラー終了します。
\end{itemize}


\newpage
\subsection{CoulombIntra指定ファイル(\tr{coulombintra.def})}
\label{Subsec:coulombintra}
オンサイトクーロン相互作用をハミルトニアンに付け加えます。付け加える項は以下で与えられます。
\begin{equation}
{\cal H}_U =\sum_{i}U_i n_ {i \uparrow}n_{i \downarrow}
\end{equation}
以下にファイル例を記載します。

\begin{minipage}{12.5cm}
\begin{screen}
\begin{verbatim}
====================== 
NCoulombIntra 6  
====================== 
========i_0LocSpn_1IteElc ====== 
====================== 
   0  4.000000
   1  4.000000
   2  4.000000
   3  4.000000
   4  4.000000
   5  4.000000
\end{verbatim}
\end{screen}
\end{minipage}

\subsubsection{ファイル形式}
以下のように行数に応じ異なる形式をとります。
 \begin{itemize}
   \item  1行:  ヘッダ(何が書かれても問題ありません)。
   \item  2行:   [string01]~[int01]
   \item  3-5行:  ヘッダ(何が書かれても問題ありません)。
   \item  6行以降:
   [int02]~[double01] 
  \end{itemize}
\subsubsection{パラメータ}
 \begin{itemize}

   \item  $[$string01$]$
   
    {\bf 形式 :} string型 (空白不可)

   {\bf 説明 :} オンサイトクーロン相互作用の総数のキーワード名を指定します(任意)。

   \item  $[$int01$]$
   
    {\bf 形式 :} int型 (空白不可)

   {\bf 説明 :} オンサイトクーロン相互作用の総数を指定します。

  \item  $[$int02$]$
  
 {\bf 形式 :} int型 (空白不可)

{\bf 説明 :} サイト番号を指定する整数。0以上\verb|Nsite|{未満}で指定します。
 
 \item  $[$double01$]$
   
   {\bf 形式 :} double型 (空白不可)

  {\bf 説明 :}  $U_i$を指定します。
  
\end{itemize}

\subsubsection{使用ルール}
本ファイルを使用するにあたってのルールは以下の通りです。
\begin{itemize}
\item 行数固定で読み込みを行う為、ヘッダの省略はできません。
\item $[$int01$]$と定義されているオンサイトクーロン相互作用の総数が異なる場合はエラー終了します。
\item $[$int02$]$を指定する際、範囲外の整数を指定した場合はエラー終了します。
\end{itemize}


\newpage
\subsection{CoulombInter指定ファイル(\tr{coulombiter.def})}
サイト間クーロン相互作用をハミルトニアンに付け加えます。付け加える項は以下で与えられます。
\begin{equation}
{\cal H}_V = \sum_{i,j}V_{ij} n_ {i}n_{j}
\end{equation}
以下にファイル例を記載します。

\begin{minipage}{12.5cm}
\begin{screen}
\begin{verbatim}
====================== 
NCoulombInter 12  
====================== 
========CoulombInter ====== 
====================== 
    0     1         1.500000000000000
    0     3         1.500000000000000
    1     2         1.500000000000000
    1     4         1.500000000000000
    2     0         1.500000000000000
    2     5         1.500000000000000
    3     4         1.500000000000000
    3     0         1.500000000000000
    4     5         1.500000000000000
    4     1         1.500000000000000
    5     3         1.500000000000000
    5     2         1.500000000000000
\end{verbatim}
\end{screen}
\end{minipage}

\subsubsection{ファイル形式}
以下のように行数に応じ異なる形式をとります。
 \begin{itemize}
   \item  1行:  ヘッダ(何が書かれても問題ありません)。
   \item  2行:   [string01]~[int01]
   \item  3-5行:  ヘッダ(何が書かれても問題ありません)。
   \item  6行以降:
   [int02]~[int03]~[double01] 
  \end{itemize}
\subsubsection{パラメータ}
 \begin{itemize}

   \item  $[$string01$]$
   
    {\bf 形式 :} string型 (空白不可)

   {\bf 説明 :} サイト間クーロン相互作用の総数のキーワード名を指定します(任意)。

   \item  $[$int01$]$
   
    {\bf 形式 :} int型 (空白不可)

   {\bf 説明 :} サイト間クーロン相互作用の総数を指定します。

  \item  $[$int02$]$, $[$int03$]$
  
 {\bf 形式 :} int型 (空白不可)

{\bf 説明 :} サイト番号を指定する整数。0以上\verb|Nsite|{未満}で指定します。
 
 \item  $[$double01$]$
   
   {\bf 形式 :} double型 (空白不可)

  {\bf 説明 :}  $V_{ij}$を指定します。
  
\end{itemize}

\subsubsection{\tr{使用ルール}}
本ファイルを使用するにあたってのルールは以下の通りです。
\begin{itemize}
\item 行数固定で読み込みを行う為、ヘッダの省略はできません。
\item $[$int01$]$と定義されているオフサイトクーロン相互作用の総数が異なる場合はエラー終了します。
\item $[$int02$]$-$[$int03$]$を指定する際、範囲外の整数を指定した場合はエラー終了します。
\end{itemize}

\newpage
\subsection{Hund指定ファイル(hund.def)}
Hundカップリングをハミルトニアンに付け加えます。付け加える項は以下で与えられます。
\begin{equation}
{\cal H}_H =-\sum_{i,j}J_{ij}^{\rm Hund} (n_{i\uparrow}n_{j\uparrow}+n_{i\downarrow}n_{j\downarrow})
\end{equation}
以下にファイル例を記載します。

\begin{minipage}{12.5cm}
\begin{screen}
\begin{verbatim}
====================== 
NHund 6  
====================== 
========Hund ====== 
====================== 
   0     1 -0.250000
   1     2 -0.250000
   2     3 -0.250000
   3     4 -0.250000
   4     5 -0.250000
   5     0 -0.250000
\end{verbatim}
\end{screen}
\end{minipage}

\subsubsection{ファイル形式}
以下のように行数に応じ異なる形式をとります。
 \begin{itemize}
   \item  1行:  ヘッダ(何が書かれても問題ありません)。
   \item  2行:   [string01]~[int01]
   \item  3-5行:  ヘッダ(何が書かれても問題ありません)。
   \item  6行以降:
   [int02]~[int03]~[double01] 
  \end{itemize}
\subsubsection{パラメータ}
 \begin{itemize}

   \item  $[$string01$]$
   
    {\bf 形式 :} string型 (空白不可)

   {\bf 説明 :} Hundカップリングの総数のキーワード名を指定します(任意)。

   \item  $[$int01$]$
   
    {\bf 形式 :} int型 (空白不可)

   {\bf 説明 :} Hundカップリングの総数を指定します。

  \item  $[$int02$]$, $[$int03$]$
  
 {\bf 形式 :} int型 (空白不可)

{\bf 説明 :} サイト番号を指定する整数。0以上\verb|Nsite|{未満}で指定します。
 
 \item  $[$double01$]$
   
   {\bf 形式 :} double型 (空白不可)

  {\bf 説明 :}  $J_{ij}^{\rm Hund}$を指定します。
  
\end{itemize}

\subsubsection{使用ルール}
本ファイルを使用するにあたってのルールは以下の通りです。
\begin{itemize}
\item 行数固定で読み込みを行う為、ヘッダの省略はできません。
\item $[$int01$]$と定義されているHundカップリングの総数が異なる場合はエラー終了します。
\item $[$int02$]$-$[$int03$]$を指定する際、範囲外の整数を指定した場合はエラー終了します。
\end{itemize}

\newpage
\subsection{PairHop指定ファイル}
PairHopカップリングをハミルトニアンに付け加えます。付け加える項は以下で与えられます。
\begin{equation}
{\cal H}_P=\sum_{i,j}J_{ij}^{\rm Pair} c_ {i \uparrow}^{\dag}c_{j\uparrow}c_{i \downarrow}^{\dag}c_{j  \downarrow}
\end{equation}
以下にファイル例を記載します。

\begin{minipage}{12.5cm}
\begin{screen}
\begin{verbatim}
====================== 
NPairhop 12  
====================== 
========Pairhop ====== 
====================== 
   0     1  0.50000
   1     0  0.50000  
   1     2  0.50000
   2     1  0.50000
   2     3  0.50000
   3     2  0.50000
   3     4  0.50000
   4     3  0.50000
   4     5  0.50000
   5     4  0.50000
   5     0  0.50000
   0     5  0.50000
\end{verbatim}
\end{screen}
\end{minipage}

\subsubsection{ファイル形式}
以下のように行数に応じ異なる形式をとります。
 \begin{itemize}
   \item  1行:  ヘッダ(何が書かれても問題ありません)。
   \item  2行:   [string01]~[int01]
   \item  3-5行:  ヘッダ(何が書かれても問題ありません)。
   \item  6行以降:
   [int02]~[int03]~[double01] 
  \end{itemize}
\subsubsection{パラメータ}
 \begin{itemize}

   \item  $[$string01$]$
   
    {\bf 形式 :} string型 (空白不可)

   {\bf 説明 :} PairHopカップリングの総数のキーワード名を指定します(任意)。

   \item  $[$int01$]$
   
    {\bf 形式 :} int型 (空白不可)

   {\bf 説明 :} PairHopカップリングの総数を指定します。

  \item  $[$int02$]$, $[$int03$]$
  
 {\bf 形式 :} int型 (空白不可)

{\bf 説明 :} サイト番号を指定する整数。0以上\verb|Nsite|{未満}で指定します。
 
 \item  $[$double01$]$
   
   {\bf 形式 :} double型 (空白不可)

  {\bf 説明 :}  $J_{ij}^{\rm Pair}$を指定します。
  
\end{itemize}

\subsubsection{使用ルール}
本ファイルを使用するにあたってのルールは以下の通りです。
\begin{itemize}
\item 行数固定で読み込みを行う為、ヘッダの省略はできません。
\item $[$int01$]$と定義されているPairHopカップリングの総数が異なる場合はエラー終了します。
\item $[$int02$]$-$[$int03$]$を指定する際、範囲外の整数を指定した場合はエラー終了します。
\end{itemize}

\newpage
\subsection{Exchange指定ファイル (\tr{exchange.def})}
Exchangeカップリングをハミルトニアンに付け加えます。
電子系の場合には
\begin{equation}
{\cal H}_E  =\sum_{i,j}J_{ij}^{\rm Ex} (c_ {i \uparrow}^{\dag}c_{j\uparrow}c_{j \downarrow}^{\dag}c_{i  \downarrow}+c_ {i \downarrow}^{\dag}c_{j\downarrow}c_{j \uparrow}^{\dag}c_{i  \uparrow})
\end{equation}
が付け加えられ、スピン系の場合には
\begin{equation}
{\cal H}_E =\sum_{i,j}J_{ij}^{\rm Ex} (S_i^+S_j^-+S_i^-S_j^+)
\end{equation}
が付け加えられます。
以下にファイル例を記載します。

\begin{minipage}{12.5cm}
\begin{screen}
\begin{verbatim}
====================== 
NExchange 6  
====================== 
========Exchange ====== 
====================== 
   0     1  0.50000
   1     2  0.50000
   2     3  0.50000
   3     4  0.50000
   4     5  0.50000
   5     0  0.50000
\end{verbatim}
\end{screen}
\end{minipage}

\subsubsection{ファイル形式}
以下のように行数に応じ異なる形式をとります。
 \begin{itemize}
   \item  1行:  ヘッダ(何が書かれても問題ありません)。
   \item  2行:   [string01]~[int01]
   \item  3-5行:  ヘッダ(何が書かれても問題ありません)。
   \item  6行以降:
   [int02]~[int03]~[double01] 
  \end{itemize}
\subsubsection{パラメータ}
 \begin{itemize}

   \item  $[$string01$]$
   
    {\bf 形式 :} string型 (空白不可)

   {\bf 説明 :} Exchangeカップリングの総数のキーワード名を指定します(任意)。

   \item  $[$int01$]$
   
    {\bf 形式 :} int型 (空白不可)

   {\bf 説明 :} Exchangeカップリングの総数を指定します。

  \item  $[$int02$]$, $[$int03$]$
  
 {\bf 形式 :} int型 (空白不可)

{\bf 説明 :} サイト番号を指定する整数。0以上\verb|Nsite|{未満}で指定します。
 
 \item  $[$double01$]$
   
   {\bf 形式 :} double型 (空白不可)

  {\bf 説明 :}  $J_{ij}^{\rm Ex}$を指定します。
  
\end{itemize}

\subsubsection{使用ルール}
本ファイルを使用するにあたってのルールは以下の通りです。
\begin{itemize}
\item 行数固定で読み込みを行う為、ヘッダの省略はできません。
\item $[$int01$]$と定義されているExchangeカップリングの総数が異なる場合はエラー終了します。
\item $[$int02$]$-$[$int03$]$を指定する際、範囲外の整数を指定した場合はエラー終了します。
\end{itemize}

\newpage
\subsection{Gutzwiller指定ファイル(\tr{gutzwiller.def})}
\label{Subsec:Gutzwiller}

Gutzwiller因子
\begin{equation}
{\cal P}_G=\exp\left[ \sum_i g_i n_{i\uparrow} n_{i\downarrow} \right]
\end{equation}
の設定を行います。指定するパラメータはサイト番号$i$と$g_i$の変分パラメータの番号です。
以下にファイル例を記載します。

\begin{minipage}{12.5cm}
\begin{screen}
\begin{verbatim}
======================
NGutzwillerIdx 2  
ComplexType 0
====================== 
====================== 
   0     0
   1     0
   2     0
   3     1
(continue...)
  12     1
  13     0
  14     0
  15     0
   0     1
   1     0
\end{verbatim}
\end{screen}
\end{minipage}

\subsubsection{ファイル形式}
以下のように行数に応じ異なる形式をとります($N_s$はサイト数、$N_g$は変分パラメータの種類の数)。
 \begin{itemize}
   \item  1行:  ヘッダ(何が書かれても問題ありません)。
   \item  2行:   [string01]~[int01]
   \item  3行:   [string02]~[int02]
   \item  4-5行:  ヘッダ(何が書かれても問題ありません)。
   \item  6 - (5+$N_s$)行: [int03]~[int04]
   \item  (6+$N_s$) - (5+$N_s$+$N_g$)行:[int05]~[int06]
  \end{itemize}
\subsubsection{パラメータ}
 \begin{itemize}

   \item  $[$string01$]$
   
    {\bf 形式 :} string型 (空白不可)

   {\bf 説明 :} $g_i$の変分パラメータの種類の総数のキーワード名を指定します(任意)。

   \item  $[$int01$]$
   
    {\bf 形式 :} int型 (空白不可)

   {\bf 説明 :} $g_i$の変分パラメータの種類の総数を指定します。

   \item  $[$string02$]$
   
    {\bf 形式 :} string型 (空白不可)

   {\bf 説明 :} $g_i$の変分パラメータの型を指定するためのキーワード名を指定します(任意)。

  \item  $[$int02$]$
  
 {\bf 形式 :} int型 (空白不可)

{\bf 説明 :} 変分パラメータの型を指定する整数。0が実数、1が複素数に対応します。

  \item  $[$int03$]$
  
 {\bf 形式 :} int型 (空白不可)

{\bf 説明 :} サイト番号を指定する整数。0以上\verb|Nsite|{未満}で指定します。
 
 \item  $[$int04$]$
   
   {\bf 形式 :} int型 (空白不可)

  {\bf 説明 :} $g_i$の変分パラメータの種類を表します。0以上[int01]{未満}で指定します。

 \item  $[$int05$]$
   
   {\bf 形式 :} int型 (空白不可)

  {\bf 説明 :} $g_i$の変分パラメータの種類を表します(最適化有無の設定用)。0以上[int01]{未満}で指定します。

 \item  $[$int06$]$
   
   {\bf 形式 :} int型 (空白不可)

  {\bf 説明 :} [int05]で指定した$g_i$の変分パラメータの最適化有無を設定します。最適化する場合は1、最適化しない場合は0とします。
  
\end{itemize}

\subsubsection{使用ルール}
本ファイルを使用するにあたってのルールは以下の通りです。
\begin{itemize}
\item 行数固定で読み込みを行う為、ヘッダの省略はできません。
\item $[$int01$]$と定義されている変分パラメータの種類の総数が異なる場合はエラー終了します。
\item $[$int02$]$-$[$int06$]$を指定する際、範囲外の整数を指定した場合はエラー終了します。
\end{itemize}

\newpage
\subsection{Jastrow指定ファイル(\tr{jastrow.def})}
\label{Subsec:Jastrow}
Jastrow因子
\begin{equation}
{\cal P}_J=\exp\left[\frac{1}{2} \sum_{i\neq j} v_{ij} (n_i-1) (n_j-1)\right]
\end{equation}
の設定を行います。指定するパラメータはサイト番号$i, j$と$v_{ij}$の変分パラメータの番号です。以下にファイル例を記載します。

\begin{minipage}{12.5cm}
\begin{screen}
\begin{verbatim}
======================
NJastrowIdx 5  
ComplexType 0
====================== 
======================
   0     1     0 
   0     2     1 
   0     3     0 
 (continue...)
   0    1 
   1    1 
   2    1 
   3    1 
   4    1 
\end{verbatim}
\end{screen}
\end{minipage}

\subsubsection{ファイル形式}
以下のように行数に応じ異なる形式をとります($N_s$はサイト数、$N_j$は変分パラメータの種類の数)。
 \begin{itemize}
   \item  1行:  ヘッダ(何が書かれても問題ありません)。
   \item  2行:   [string01]~[int01]
   \item  3行:   [string02]~[int02]
   \item  4-5行:  ヘッダ(何が書かれても問題ありません)。
   \item  6 - (5+$N_s\times (N_s-1)$) 行: [int03]~[int04]~[int05]
   \item  (6+$N_s\times (N_s-1)$ ) - (5+$N_s\times (N_s-1)$+$N_j$)行:[int06]~[int07]
  \end{itemize}
\subsubsection{パラメータ}
 \begin{itemize}

   \item  $[$string01$]$
   
    {\bf 形式 :} string型 (空白不可)

   {\bf 説明 :} $v_{ij}$の変分パラメータの種類の総数のキーワード名を指定します(任意)。

   \item  $[$int01$]$
   
    {\bf 形式 :} int型 (空白不可)

   {\bf 説明 :} $v_{ij}$の変分パラメータの種類の総数を指定します。

   \item  $[$string02$]$
   
    {\bf 形式 :} string型 (空白不可)

   {\bf 説明 :} $v_{ij}$の変分パラメータの型を指定するためのキーワード名を指定します(任意)。

   \item  $[$int02$]$
   
    {\bf 形式 :} int型 (空白不可)

   {\bf 説明 :} $v_{ij}$の変分パラメータの型を指定します。0が実数、1が複素数に対応します。

  \item  $[$int03$]$, $[$int04$]$
  
 {\bf 形式 :} int型 (空白不可)

{\bf 説明 :} サイト番号を指定する整数。0以上\verb|Nsite|{未満}で指定します。
 
 \item  $[$int05$]$
   
   {\bf 形式 :} int型 (空白不可)

  {\bf 説明 :} $v_{ij}$の変分パラメータの種類を表します。0以上[int01]{未満}で指定します。

 \item  $[$int06$]$
   
   {\bf 形式 :} int型 (空白不可)

  {\bf 説明 :} $v_{ij}$の変分パラメータの種類を表します(最適化有無の設定用)。0以上[int01]{未満}で指定します。

 \item  $[$int07$]$
   
   {\bf 形式 :} int型 (空白不可)

  {\bf 説明 :} [int06]で指定した$v_{ij}$の変分パラメータの最適化有無を設定します。最適化する場合は1、最適化しない場合は0とします。
  
\end{itemize}

\subsubsection{使用ルール}
本ファイルを使用するにあたってのルールは以下の通りです。
\begin{itemize}
\item 行数固定で読み込みを行う為、ヘッダの省略はできません。
\item $[$int01$]$と定義されている変分パラメータの種類の総数が異なる場合はエラー終了します。
\item $[$int02$]$-$[$int07$]$を指定する際、範囲外の整数を指定した場合はエラー終了します。
\end{itemize}

\newpage
\subsection{DH2指定ファイル}
\label{Subsec:DH2}

\begin{equation}
{\cal P}_{d-h}^{(2)}= \exp \left[ \sum_t \sum_{n=0}^2 (\alpha_{2nt}^d \sum_{i}\xi_{i2nt}^d+\alpha_{2nt}^h \sum_{i}\xi_{i2nt}^h)\right]
\end{equation}
で表される2サイトのdoublon-holon相関因子の設定を行います。指定するパラメータはサイト番号$i$とその周囲2サイト、$\alpha_{2nt}^{d,h}$の変分パラメータの番号で、変分パラメータは各サイト毎に$t$種類設定します。\tr{各パラメータ、演算子に関する詳細は文献\cite{Tahara2008}をご覧ください。}以下にファイル例を記載します。

\begin{minipage}{12.5cm}
\begin{screen}
\begin{verbatim}
====================================
NDoublonHolon2siteIdx 2  
ComplexType 0
====================================
====================================
   0     5   15    0
   0    13    7    1
   1     6   12    0
   1    14    4    1
 (continue...)
  15     0   10    0
  15     8    2    1
   0     1 
   1     1 
   2     1 
(continue...)
  10     1 
  11     1 
\end{verbatim}
\end{screen}
\end{minipage}

\subsubsection{ファイル形式}
以下のように行数に応じ異なる形式をとります($N_s$はサイト数、$N_{\rm dh2}$は変分パラメータの種類の数)。
 \begin{itemize}
   \item  1行:  ヘッダ(何が書かれても問題ありません)。
   \item  2行:   [string01]~[int01]
   \item  3行:   [string02]~[int02]
   \item  4-5行:  ヘッダ(何が書かれても問題ありません)。
   \item  6 - (5+$N_s\times N_{\rm dh2}$)行: [int03]~[int04]~[int05]~[int06]
   \item  (6+$N_s\times N_{\rm dh2}$) - (5+$(N_s+6) \times N_{\rm dh2}$)行:[int07]~[int08]
  \end{itemize}
\subsubsection{パラメータ}
 \begin{itemize}

   \item  $[$string01$]$
   
    {\bf 形式 :} string型 (空白不可)

   {\bf 説明 :} 変分パラメータのセット総数のキーワード名を指定します(任意)。

   \item  $[$int01$]$
   
    {\bf 形式 :} int型 (空白不可)

   {\bf 説明 :} 変分パラメータのセット総数を指定します。

  \item  $[$string02$]$
   
    {\bf 形式 :} string型 (空白不可)

   {\bf 説明 :}変分パラメータの型を指定するためのキーワード名を指定します(任意)。
   
   \item  $[$int02$]$
   
    {\bf 形式 :} int型 (空白不可)

   {\bf 説明 :} 変分パラメータの型を指定します。0が実数、1が複素数に対応します。

  \item  $[$int03$]$,  $[$int04$]$, $[$int05$]$
   
 {\bf 形式 :} int型 (空白不可)

{\bf 説明 :} サイト番号を指定する整数。0以上\verb|Nsite|{未満}で指定します。
 
 \item  $[$int06$]$
   
   {\bf 形式 :} int型 (空白不可)

  {\bf 説明 :} 変分パラメータの種類を表します。0以上[int01]{未満}で指定します。

 \item  $[$int07$]$
   
   {\bf 形式 :} int型 (空白不可)

  {\bf 説明 :} 変分パラメータの種類を表します(最適化有無の設定用)。値は
  \begin{itemize}
  \item{$n$}: 周囲のdoublon(holon)数 (0, 1, 2)  \\
  \item{$s$}: 中心がdoublonの場合$0$, 中心がholonの場合$1$ \\
  \item{$t$}: 変分パラメータのセット番号(0, $\cdots$ [int1]-1)
  \end{itemize}  
  として、$(2n+s)\times$[int01]$+t$を設定します。
  
 \item  $[$int08$]$
   
   {\bf 形式 :} int型 (空白不可)

  {\bf 説明 :} [int07]で指定した変分パラメータの最適化有無を設定します。最適化する場合は1、最適化しない場合は0とします。
  
  
\end{itemize}

\subsubsection{使用ルール}
本ファイルを使用するにあたってのルールは以下の通りです。
\begin{itemize}
\item 行数固定で読み込みを行う為、ヘッダの省略はできません。
\item $[$int01$]$と定義されている変分パラメータの種類の総数が異なる場合はエラー終了します。
\item $[$int02$]$-$[$int08$]$を指定する際、範囲外の整数を指定した場合はエラー終了します。
\end{itemize}


\newpage
\subsection{DH4指定ファイル}
\label{Subsec:DH4}

\begin{equation}
{\cal P}_{d-h}^{(4)}= \exp \left[ \sum_t \sum_{n=0}^4 (\alpha_{4nt}^d \sum_{i}\xi_{i4nt}^d+\alpha_{4nt}^h \sum_{i}\xi_{i4nt}^h)\right]
\end{equation}
で表される4サイトのdoublon-holon相関因子の設定を行います。指定するパラメータはサイト番号$i$とその周囲4サイト、$\alpha_{4nt}^{d,h}$の変分パラメータの番号で、変分パラメータは各サイト毎に$t$種類設定します。\tr{各パラメータ、演算子に関する詳細は文献\cite{Tahara2008}をご覧ください。}以下にファイル例を記載します。

\begin{minipage}{12.5cm}
\begin{screen}
\begin{verbatim}
====================================
NDoublonHolon4siteIdx 1  
ComplexType 0
====================================
====================================
   0     1    3    4   12    0
   1     2    0    5   13    0
   2     3    1    6   14    0
   3     0    2    7   15    0
 (continue...)
  14    15   13    2   10    0
  15    12   14    3   11    0
   0     1 
   1     1 
(continue...)
   8     1 
   9     1 
\end{verbatim}
\end{screen}
\end{minipage}

\subsubsection{ファイル形式}
以下のように行数に応じ異なる形式をとります($N_s$はサイト数、$N_{\rm dh4}$は変分パラメータの種類の数)。
 \begin{itemize}
   \item  1行:  ヘッダ(何が書かれても問題ありません)。
   \item  2行:   [string01]~[int01]
   \item  3行:   [string02]~[int02]
   \item  4-5行:  ヘッダ(何が書かれても問題ありません)。
   \item  6 - (5+$N_s\times N_{\rm dh4}$)行: [int03]~[int04]~[int05]~[int06]~[int07]~[int08]
   \item  (6+$N_s\times N_{\rm dh4}$) - (5+$(N_s+10) \times N_{\rm dh4}$)行:[int09]~[int10]
  \end{itemize}
\subsubsection{パラメータ}
 \begin{itemize}

   \item  $[$string01$]$
   
    {\bf 形式 :} string型 (空白不可)

   {\bf 説明 :} 変分パラメータのセット総数のキーワード名を指定します(任意)。

   \item  $[$int01$]$
   
    {\bf 形式 :} int型 (空白不可)

   {\bf 説明 :} 変分パラメータのセット総数を指定します。

   \item  $[$string02$]$
   
    {\bf 形式 :} string型 (空白不可)

   {\bf 説明 :} 変分パラメータの型を指定するためのキーワード名を指定します(任意)。

   \item  $[$int02$]$
   
    {\bf 形式 :} int型 (空白不可)

   {\bf 説明 :} 変分パラメータの型を指定します。0が実数、1が複素数に対応します。

  \item   $[$int03$]$,  $[$int04$]$, $[$int05$]$, $[$int06$]$, $[$int07$]$
   
 {\bf 形式 :} int型 (空白不可)

{\bf 説明 :} サイト番号を指定する整数。0以上\verb|Nsite|{未満}で指定します。
 
 \item  $[$int08$]$
   
   {\bf 形式 :} int型 (空白不可)

  {\bf 説明 :} 変分パラメータの種類を表します。0以上[int01]{未満}で指定します。

 \item  $[$int09$]$
   
   {\bf 形式 :} int型 (空白不可)

  {\bf 説明 :} 変分パラメータの種類を表します(最適化有無の設定用)。値は
  \begin{itemize}
  \item{$n$}: 周囲のdoublon(holon)数 (0, 1, 2, 3, 4)  \\
  \item{$s$}: 中心がdoublonの場合$0$, 中心がholonの場合$1$ \\
  \item{$t$}: 変分パラメータのセット番号(0, $\cdots$ [int1]-1)
  \end{itemize}  
  として、$(2n+s)\times$[int01]$+t$を設定します。
  
 \item  $[$int10$]$
   
   {\bf 形式 :} int型 (空白不可)

  {\bf 説明 :} [int09]で指定した変分パラメータの最適化有無を設定します。最適化する場合は1、最適化しない場合は0とします。
  
\end{itemize}

\subsubsection{使用ルール}
本ファイルを使用するにあたってのルールは以下の通りです。
\begin{itemize}
\item 行数固定で読み込みを行う為、ヘッダの省略はできません。
\item $[$int01$]$と定義されている変分パラメータの種類の総数が異なる場合はエラー終了します。
\item $[$int02$]$-$[$int10$]$を指定する際、範囲外の整数を指定した場合はエラー終了します。
\end{itemize}

\newpage
\subsection{Orbital指定ファイル(\tr{orbitalidx.def})}
\label{Subsec:Orbital}

\begin{equation}
|\phi_{\rm pair} \rangle = \left[\sum_{i, j=1}^{N_s} f_{ij}c_{i\uparrow}^{\dag}c_{j\downarrow}^{\dag} \right]^{N/2}|0 \rangle
\end{equation}
で表されるペア軌道の設定を行います。指定するパラメータはサイト番号$i, j$と変分パラメータの種類を設定します。以下にファイル例を記載します。

\begin{minipage}{12.5cm}
\begin{screen}
\begin{verbatim}
====================================
NOrbitalIdx 64   
ComplexType 0
====================================
====================================
   0     0     0 
   0     1     1 
   0     2     2 
   0     3     3 
 (continue...)
  15     9    62 
  15    10    63 
   0    1 
   1    1 
(continue...)
  62    1 
  63    1 
\end{verbatim}
\end{screen}
\end{minipage}

\subsubsection{ファイル形式}
以下のように行数に応じ異なる形式をとります($N_s$はサイト数、$N_{\rm o}$は変分パラメータの種類の数)。
 \begin{itemize}
   \item  1行:  ヘッダ(何が書かれても問題ありません)。
   \item  2行:   [string01]~[int01]
   \item  3行:   [string02]~[int02]
   \item  4-5行:  ヘッダ(何が書かれても問題ありません)。
   \item  6 - (5+$N_s^2$)行: [int03]~[int04]~[int05]~[int06]
   \item  (6+$N_s^2$) - (5+$N_s^2+N_{\rm o}$)行:[int07]~[int08]
  \end{itemize}
\subsubsection{パラメータ}
 \begin{itemize}

   \item  $[$string01$]$
   
    {\bf 形式 :} string型 (空白不可)

   {\bf 説明 :} 変分パラメータのセット総数のキーワード名を指定します(任意)。

   \item  $[$int01$]$
   
    {\bf 形式 :} int型 (空白不可)

   {\bf 説明 :} 変分パラメータのセット総数を指定します。

   \item  $[$string02$]$
   
    {\bf 形式 :} string型 (空白不可)

   {\bf 説明 :} 変分パラメータの型を指定するためのキーワード名を指定します(任意)。

   \item  $[$int02$]$
   
    {\bf 形式 :} int型 (空白不可)

   {\bf 説明 :} 変分パラメータの型を指定します。0が実数、1が複素数に対応します。

  \item  $[$int03$]$, $[$int04$]$
   
 {\bf 形式 :} int型 (空白不可)

{\bf 説明 :} サイト番号を指定する整数。0以上\verb|Nsite|{未満}で指定します。
 
 \item  $[$int05$]$
   
   {\bf 形式 :} int型 (空白不可)

  {\bf 説明 :} 変分パラメータの種類を表します。0以上[int01]{未満}で指定します。

 \item  $[$int06$]$
   
   {\bf 形式 :} int型

  {\bf 説明 :} 反周期境界条件モードがON(\verb|ModPara|ファイルで\verb|NMPTrans|が負の場合に有効)の場合、変分パラメータ$f_{ij}$の番号の他に符号を反転するか否かを直接指定する。$[$int06$]$=$\pm1$により符号を指定する。反周期境界条件モードがOFFの場合は省略可能。


 \item  $[$int07$]$
   
   {\bf 形式 :} int型 (空白不可)

  {\bf 説明 :} 変分パラメータの種類を表します(最適化有無の設定用)。0以上[int01]{未満}で指定します。
  
 \item  $[$int08$]$
   
   {\bf 形式 :} int型 (空白不可)

  {\bf 説明 :} [int06]で指定した変分パラメータの最適化有無を設定します。最適化する場合は1、最適化しない場合は0とします。
  
  
\end{itemize}

\subsubsection{使用ルール}
本ファイルを使用するにあたってのルールは以下の通りです。
\begin{itemize}
\item 行数固定で読み込みを行う為、ヘッダの省略はできません。
\item $[$int01$]$と定義されている変分パラメータの種類の総数が異なる場合はエラー終了します。
\item $[$int02$]$-$[$int08$]$を指定する際、範囲外の整数を指定した場合はエラー終了します。
\end{itemize}



\newpage
\subsection{TransSym指定ファイル(\tr{qptransidx.def})}
\label{Subsec:TransSym}

運動量射影${\cal L}_K=\frac{1}{N_s}\sum_{{\bm R}}e^{i {\bm K} \cdot{\bm R} } \hat{T}_{\bm R}$と格子対称性射影${\cal L}_P=\sum_{\alpha}p_{\alpha} \hat{G}_{\alpha}$について、重みとサイト番号に関する指定を行います。射影するパターンは$(\alpha, {\bm R})$で指定されます。射影を行わない場合も重み1.0 で“恒等演算”を指定してください。
以下にファイル例を記載します。

\begin{minipage}{12.5cm}
\begin{screen}
\begin{verbatim}
====================================
NQPTrans 4  
====================================
== TrIdx_TrWeight_and_TrIdx_i_xi  ==
====================================
   0  1.000000
   1  1.000000
   2  1.000000
   3  1.000000
   0     0    0
 (continue...)
   3    12    1
   3    13    2 
\end{verbatim}
\end{screen}
\end{minipage}

\subsubsection{ファイル形式}
以下のように行数に応じ異なる形式をとります($N_s$はサイト数、$N_{\rm TS}$は射影演算子の種類の総数)。
 \begin{itemize}
   \item  1行:  ヘッダ(何が書かれても問題ありません)。
   \item  2行:   [string01]~[int01]
   \item  3-5行:  ヘッダ(何が書かれても問題ありません)。
   \item  6 - (5+$N_{\rm TS}$)行: [int02]~[double01]
   \item  (6+$N_{\rm TS}$) - (5+$(N_s+1) \times N_{\rm TS}$)行:[int03]~[int04]~[int05]~[int06]
  \end{itemize}
\subsubsection{パラメータ}
 \begin{itemize}

   \item  $[$string01$]$
   
    {\bf 形式 :} string型 (空白不可)

   {\bf 説明 :} 射影パターンの総数に関するキーワード名を指定します(任意)。

   \item  $[$int01$]$
   
    {\bf 形式 :} int型 (空白不可)

   {\bf 説明 :} 射影パターンの総数を指定します。

  \item  $[$int02$]$
   
 {\bf 形式 :} int型 (空白不可)

{\bf 説明 :} 射影パターン$(\alpha, {\bm R})$を指定する整数。0以上 $[$int01$]${未満}で指定します。
 
  \item  $[$double01$]$
   
 {\bf 形式 :} double型 (空白不可)

{\bf 説明 :} 射影パターン$(\alpha, {\bm R})$の重み$p_{\alpha}\cos ({\bm K}\cdot {\bm R})$を指定します。
 
 \item  $[$int03$]$
   
   {\bf 形式 :} int型 (空白不可)

  {\bf 説明 :} 射影パターン$(\alpha, {\bm R})$を指定する整数。0以上 $[$int01$]${未満}で指定します。

 \item  $[$int04$]$, $[$int05$]$
   
   {\bf 形式 :} int型 (空白不可)

  {\bf 説明 :} サイト番号を指定する整数。0以上\verb|Nsite|{未満}で指定します。$[$int03$]$で指定した並進・点群移動をサイト番号$[$int04$]$に作用させた場合の行き先が、サイト番号$[$int05$]$となるように設定します。

 \item  $[$int06$]$
   
   {\bf 形式 :} int型 (空白不可)

  {\bf 説明 :} 反周期境界条件モードがON(\verb|ModPara|ファイルで\verb|NMPTrans|が負の場合に有効)の場合、並進演算で生成消滅演算子の符号が反転するか否かを直接指定する。$[$int06$]$=$\pm1$により符号を指定する。反周期境界条件モードがOFFの場合は省略可能。
\end{itemize}

\subsubsection{使用ルール}
本ファイルを使用するにあたってのルールは以下の通りです。
\begin{itemize}
\item 行数固定で読み込みを行う為、ヘッダの省略はできません。
\item $[$int01$]$と定義されている射影パターンの総数が異なる場合はエラー終了します。
\item \tr{$[$int02$]$-$[$int06$]$を指定する際、範囲外の整数を指定した場合はエラー終了します。}
\end{itemize}

\newpage
\subsection{\tr{変分パラメータ初期値指定ファイル}}
\label{Subsec:InputParam}
各変分パラメータの初期値を設定することが可能です。\\
変分パラメータの種類は\tr{入力ファイル指定用ファイル(Sec. \ref{Subsec:InputFileList})において}\\
\verb|InGutzwiller|, \verb|InJastrow|, \verb|InDH2|, \verb|InDH4|, \verb|InOrbital|\\
をキーワードとして指定することで区別します。なお、ファイルフォーマットは全て共通です。
以下、\verb|InJastrow|ファイルの例を記載します。

\begin{minipage}{12.5cm}
\begin{screen}
\begin{verbatim}
======================
NJastrowIdx  28
====================== 
== i_j_JastrowIdx  ===
====================== 
0 -8.909963465082626488e-02  0.000000000000000000e+00
1  5.521681211878626955e-02  0.000000000000000000e+00
(continue...)
27 -9.017586139930480749e-02  0.000000000000000000e+00
\end{verbatim}
\end{screen}
\end{minipage}

\subsubsection{ファイル形式}
以下のように行数に応じ異なる形式をとります($N_v$は変分パラメータの種類の総数)。
 \begin{itemize}
   \item  1行:  ヘッダ(何が書かれても問題ありません)。
   \item  2行:   [string01]~[int01]
   \item  3-5行:  ヘッダ(何が書かれても問題ありません)。
   \item  6 - (5+$N_v$)行: [int02]~[double01]~[double02]
  \end{itemize}
\subsubsection{パラメータ}
 \begin{itemize}

   \item  $[$string01$]$
   
    {\bf 形式 :} string型 (空白不可)

   {\bf 説明 :}変分パラメータ総数のキーワード名を指定します(任意)。

   \item  $[$int01$]$
   
    {\bf 形式 :} int型 (空白不可)

   {\bf 説明 :} 変分パラメータ総数を指定します。

  \item  $[$int02$]$
  
 {\bf 形式 :} int型 (空白不可)

{\bf 説明 :} 変分パラメータの種類を指定する整数。0以上 $[$int01$]$で指定します。
 
 \item  $[$double01$]$
   
   {\bf 形式 :} double型 (空白不可)

  {\bf 説明 :} 変分パラメータの初期値の実部を与えます。
   
   \item  $[$double02$]$
   {\bf 形式 :} double型  \tr{(空白不可)}

  {\bf 説明 :} 変分パラメータの初期値の虚部を与えます。
  
\end{itemize}

\subsubsection{使用ルール}
本ファイルを使用するにあたってのルールは以下の通りです。
\begin{itemize}
\item 行数固定で読み込みを行う為、ヘッダの省略はできません。
\item $[$int01$]$と定義されている変分パラメータの総数が異なる場合はエラー終了します。
\end{itemize}


\newpage
\subsection{OneBodyG指定ファイル(\tr{greenone.def})}
\label{Subsec:onebodyg}
計算・出力する一体グリーン関数$\langle c_{i\sigma_1}^{\dagger}c_{j\sigma_2}\rangle$を指定します。以下にファイル例を記載します。

\begin{minipage}{12.5cm}
\begin{screen}
\begin{verbatim}
===============================
NCisAjs         24
===============================
======== Green functions ======
===============================
    0     0     0     0
    0     1     0     1
    1     0     1     0
    1     1     1     1
    2     0     2     0
    2     1     2     1
    3     0     3     0
    3     1     3     1
    4     0     4     0
    4     1     4     1
    5     0     5     0
    5     1     5     1
    6     0     6     0
    6     1     6     1
    7     0     7     0
    7     1     7     1
    8     0     8     0
    8     1     8     1
    9     0     9     0
    9     1     9     1
   10     0    10     0
   10     1    10     1
   11     0    11     0
   11     1    11     1
\end{verbatim}
\end{screen}
\end{minipage}

\subsubsection{ファイル形式}
以下のように行数に応じ異なる形式をとります。
 \begin{itemize}
   \item  1行:  ヘッダ(何が書かれても問題ありません)。
   \item  2行:   [string01]~[int01]
   \item  3-5行:  ヘッダ(何が書かれても問題ありません)。
   \item  6行以降: [int02]~~[int03]~~[int04]~~[int05]
  \end{itemize}
\subsubsection{パラメータ}
 \begin{itemize}

   \item  $[$string01$]$
   
    {\bf 形式 :} string型 (空白不可)

   {\bf 説明 :} 一体グリーン関数成分総数のキーワード名を指定します(任意)。

   \item  $[$int01$]$
   
    {\bf 形式 :} int型 (空白不可)

   {\bf 説明 :} 一体グリーン関数成分の総数を指定します。

  \item  $[$int02$]$, $[$int04$]$

 {\bf 形式 :} int型 (空白不可)

{\bf 説明 :} サイト番号を指定する整数。0以上\verb|Nsite|{未満}で指定します。
 
  \item  $[$int03$]$, $[$int05$]$

 {\bf 形式 :} int型 (空白不可)

{\bf 説明 :} スピンを指定する整数。\\
0: アップスピン\\
1: ダウンスピン\\
を選択することが出来ます。

\end{itemize}

\subsubsection{使用ルール}
本ファイルを使用するにあたってのルールは以下の通りです。
\begin{itemize}
\item 行数固定で読み込みを行う為、ヘッダの省略はできません。
\item $[$int01$]$と定義されている一体グリーン関数成分の総数が異なる場合はエラー終了します。
\item $[$int02$]$-$[$int05$]$を指定する際、範囲外の整数を指定した場合はエラー終了します。
\end{itemize}

\newpage
\subsection{TwoBodyG指定ファイル(\tr{greentwo.def})}
\label{Subsec:twobodyg}
\tr{計算・出力する}二体グリーン関数$\langle c_{i\sigma_1}^{\dagger}c_{j\sigma_2}c_{k\sigma_3}^{\dagger}c_{l\sigma_4}\rangle$を指定します。
以下にファイル例を記載します。

\begin{minipage}{12.5cm}
\begin{screen}
\begin{verbatim}
=============================================
NCisAjsCktAltDC        576
=============================================
======== Green functions for Sq AND Nq ======
=============================================
    0     0     0     0     0     0     0     0
    0     0     0     0     0     1     0     1
    0     0     0     0     1     0     1     0
    0     0     0     0     1     1     1     1
    0     0     0     0     2     0     2     0
    0     0     0     0     2     1     2     1
    0     0     0     0     3     0     3     0
    0     0     0     0     3     1     3     1
    0     0     0     0     4     0     4     0
    0     0     0     0     4     1     4     1
    0     0     0     0     5     0     5     0
    0     0     0     0     5     1     5     1
    0     0     0     0     6     0     6     0
    0     0     0     0     6     1     6     1
    0     0     0     0     7     0     7     0
    0     0     0     0     7     1     7     1
    0     0     0     0     8     0     8     0
    0     0     0     0     8     1     8     1
    0     0     0     0     9     0     9     0
    0     0     0     0     9     1     9     1
    0     0     0     0    10     0    10     0
    0     0     0     0    10     1    10     1
    0     0     0     0    11     0    11     0
    0     0     0     0    11     1    11     1
    0     1     0     1     0     0     0     0
    …
\end{verbatim}
\end{screen}
\end{minipage}

\subsubsection{ファイル形式}
以下のように行数に応じ異なる形式をとります。
 \begin{itemize}
   \item  1行:  ヘッダ(何が書かれても問題ありません)。
   \item  2行:   [string01]~[int01]
   \item  3-5行:  ヘッダ(何が書かれても問題ありません)。
   \item  6行以降: [int02]~~[int03]~~[int04]~~[int05]~~[int06]~~[int07]~~[int08]~~[int09]
  \end{itemize}
\subsubsection{パラメータ}
 \begin{itemize}

   \item  $[$string01$]$
   
    {\bf 形式 :} string型 (空白不可)

   {\bf 説明 :} 二体グリーン関数成分総数のキーワード名を指定します(任意)。

   \item  $[$int01$]$
   
    {\bf 形式 :} int型 (空白不可)

   {\bf 説明 :} 二体グリーン関数成分の総数を指定します。

  \item  $[$int02$]$, $[$int04$]$, $[$int06$]$, $[$int08$]$

 {\bf 形式 :} int型 (空白不可)

{\bf 説明 :} サイト番号を指定する整数。0以上\verb|Nsite|{未満}で指定します。
 
  \item  $[$int03$]$, $[$int05$]$, $[$int07$]$, $[$int09$]$

 {\bf 形式 :} int型 (空白不可)

{\bf 説明 :} スピンを指定する整数。\\
0: アップスピン\\
1: ダウンスピン\\
を選択することが出来ます。

\end{itemize}

\subsubsection{使用ルール}
本ファイルを使用するにあたってのルールは以下の通りです。
\begin{itemize}
\item 行数固定で読み込みを行う為、ヘッダの省略はできません。
\item $[$int01$]$と定義されている二体グリーン関数成分の総数が異なる場合はエラー終了します。
\item $[$int02$]$-$[$int09$]$を指定する際、範囲外の整数を指定した場合はエラー終了します。
\end{itemize}

\newpage
\section{出力ファイル}
\label{Sec:outputfile}
出力ファイルの一覧は下記の通りです。***には\verb|ModPara|ファイルの\verb|CParaFileHead|で指定されるヘッダが、xxxには\verb|CDataFileHead|で指定されるヘッダが、yyyには\verb|ModPara|ファイルの\verb|NDataIdxStart|, \verb|NDataQtySmp|に従い\verb|NDataIdxStart|$\cdots$\verb|NDataIdxStart|+\verb|NDataQtySmp|の順に記載されます。また、zzzには\verb|ModPara|ファイルの\verb|NDataIdxStart|が記載されます。

 \begin{table*}[h!]
\begin{center}
  \begin{tabular}{|ll|} \hline
           Name     & Details for corresponding files       \\   \hline\hline
           ***\_opt.dat       &  All optimized parameters.       \\ \hline 
           %%%%%%%%%%%%%%%%%%  
           ***\_gutzwiller\_opt.dat       &   Optimized gutzwiller factors.  \\
           ***\_jastrow\_opt.dat        &  Optimized jastrow factors.         \\ 
           ***\_doublonHolon2site\_opt.dat  &   Optimized 2-site doublon-holon correlation factors.  \\  
           ***\_doublonHolon4site\_opt.dat  &   Optimized 4-site doublon-holon correlation factors. \\  
           ***\_orbital\_opt.dat  &   Optimized pair orbital factors. \\  \hline
           xxx\_out\_yyy.dat &  Energy and deviation.\\ 
           xxx\_var\_yyy.dat &  Progress information for optimizing variational parameters.\\  \hline
           xxx\_CalcTimer.dat & Computation time for each processes.\\  
           xxx\_time\_zzz.dat & Progress information for MonteCalro samplings.\\  \hline
           \hline
           %%%%%%%%%%%%%%%%%%
           xxx\_cisajs\_yyy.dat & One body Green's functions.\\
           xxx\_cisajscktalt\_yyy.dat & Correlation functions.\\
\hline
  \end{tabular}
\end{center}
\caption{List of the output files. }
\label{Table:Output}
\end{table*}%

\subsection{\tr{変分パラメータ出力ファイル(***\_opt.dat)}}
SR法で最適化された変分パラメータとエネルギーが一斉出力されます。
\tr{変分パラメータが一斉に読み込めるため、変分パラメータ最適化後の物理量の計算を行う場合に使用します。}
出力されるデータは
\begin{equation}
\langle H \rangle, \langle H^2 \rangle, g_i, v_{ij}, \alpha_{2nt}^{d(h)}, \alpha_{4nt}^{d(h)}, f_{ij} \nonumber
\end{equation}
で、それぞれの平均値と標準偏差が出力されます(平均値は実数、虚数の順に、標準偏差は実数のみ出力)。
なお、全データが1行で出力され、数値の間は半角空白で区切られます。
***には\verb|ModPara|ファイルの\verb|CParaFileHead|で指定されるヘッダが記載されます。

\subsection{\tr{ステップ別変分パラメータ出力ファイル(xxx\_var\_yyy.dat)}}
SR 法の各ステップにおける変分パラメータとエネルギーがzqp\_opt.dat と同形式で“追記しながら” 出力されます(標準偏差はゼロ)が出力されます。
xxxには\verb|CDataFileHead|で指定されるヘッダが、yyyには\verb|ModPara|ファイルの\verb|NDataIdxStart|, \verb|NDataQtySmp|に従い\verb|NDataIdxStart|$\cdots$\verb|NDataIdxStart|+\verb|NDataQtySmp|の順に記載されます。

\subsection{Gutzwiller因子出力ファイル(***\_gutzwiller\_opt.dat)}
SR法で最適化されたGutzwiller因子が出力されます。出力形式はSec. \ref{Subsec:InputParam}のInGutzwiller指定ファイルと同じです。

\subsection{Jastrow因子出力ファイル(***\_jastrow\_opt.dat)}
SR法で最適化されたJastrow因子が出力されます。出力形式はSec. \ref{Subsec:InputParam}のInJastrow指定ファイルと同じです。

\subsection{2サイトダブロンホロン相関因子出力ファイル\\(***\_doublonHolon2site\_opt.dat)}
SR法で最適化された2サイトのdoublon-holon相関因子が出力されます。出力形式はSec. \ref{Subsec:InputParam}のInDH2指定ファイルと同じです。

\subsection{4サイトダブロンホロン相関因子出力ファイル\\(***\_doublonHolon4site\_opt.dat)}
SR法で最適化された4サイトのdoublon-holon相関因子が出力されます。出力形式はSec. \ref{Subsec:InputParam}のInDH4指定ファイルと同じです。

\subsection{ペア軌道出力ファイル(***\_orbital\_opt.dat)}
SR法で最適化されたペア軌道の変分パラメータが出力されます。出力形式はSec. \ref{Subsec:InputParam}のInOrbital指定ファイルと同じです。

\subsection{xxx\_out\_yyy.dat}
ビン毎の計算情報として、
\begin{equation}
\langle H \rangle, \langle H^2 \rangle, \frac{\langle H^2 \rangle- \langle H \rangle^2 }{\langle H \rangle^2} \nonumber
\end{equation}
が順に出力されます。$\langle H \rangle$については実部と虚部がそれぞれ出力され、それ以外は実部のみ出力されます。
xxxには\verb|CDataFileHead|で指定されるヘッダが、yyyには\verb|ModPara|ファイルの\verb|NDataIdxStart|, \verb|NDataQtySmp|に従い\verb|NDataIdxStart|$\cdots$\verb|NDataIdxStart|+\verb|NDataQtySmp|の順に記載されます。以下に出力例を記載します。

\begin{minipage}{13cm}
\begin{screen}
\begin{verbatim}
1.151983765704212992e+01  8.124622418360909482e-01  \
1.619082955438887268e+02  2.019905203939084959e-01 
1.288482613817423150e+01  5.006903733262847433e-01  \ 
1.972000325276957824e+02  1.824505193695792893e-01
1.308897206011880421e+01  5.701244886956570168e-01  \
2.072610167083121837e+02  2.029162857569105916e-01
…
\end{verbatim}
\end{screen}
\end{minipage}

\subsection{xxx\_CalcTimer.dat }
計算終了後に、処理毎の計算処理時間が処理名、処理に割り当てられた識別番号、実行秒数の順に出力されます。出力例は以下の通りです。

\begin{minipage}{15.5cm}
\begin{screen}
\begin{verbatim}
All                         [0]     15.90724
Initialization              [1]      0.04357
  read options             [10]      0.00012
  ReadDefFile              [11]      0.00082
  SetMemory                [12]      0.00002
  InitParameter            [13]      0.03026
VMCParaOpt                  [2]     15.86367
  VMCMakeSample             [3]     12.85650
    makeInitialSample      [30]      0.20219
    make candidate         [31]      0.02553
    hopping update         [32]     12.51967
      UpdateProjCnt        [60]      7.41864
      CalculateNewPfM2     [61]      3.67098
      CalculateLogIP       [62]      0.07599
      UpdateMAll           [63]      1.27466
    exchange update        [33]      0.00000
      UpdateProjCnt        [65]      0.00000
      CalculateNewPfMTwo2  [66]      0.00000
      CalculateLogIP       [67]      0.00000
      UpdateMAllTwo        [68]      0.00000
    recal PfM and InvM     [34]      0.08294
    save electron config   [35]      0.00232
  VMCMainCal                [4]      2.45481
    CalculateMAll          [40]      0.47556
    LocEnergyCal           [41]      0.79754
      CalHamiltonian0      [70]      0.00259
      CalHamiltonian1      [71]      0.18765
      CalHamiltonian2      [72]      0.00107
    ReturnSlaterElmDiff    [42]      0.40035
    calculate OO and HO    [43]      0.68045
  StochasticOpt             [5]      0.30489
    preprocess             [50]      0.02587
    stcOptMain             [51]      0.25471
      initBLACS            [55]      0.06564
      calculate S and g    [56]      0.05603
      DPOSV                [57]      0.09833
      gatherParaChange     [58]      0.02774
    postprocess            [52]      0.02372
  UpdateSlaterElm          [20]      0.02556
  WeightAverage            [21]      0.06676
  outputData               [22]      0.10554
  SyncModifiedParameter    [23]      0.02151\end{verbatim}
\end{screen}
\end{minipage}


\newpage
\subsection{xxx\_time\_zzz.dat }
\label{subsec:time}
計算情報としてビン毎にサンプリング数、hoppingおよびexchangeのアップデートに対するacceptance ratio(acc\_hopp, acc\_ex)、それぞれのアップデートの試行回数(n\_hopp, n\_ex)および実行した際の時間を順に出力します。xxxには\verb|CDataFileHead|で指定されるヘッダが、zzzには\verb|ModPara|ファイルの\verb|NDataIdxStart|が記載されます。
出力例は以下の通りです。\\
\begin{minipage}{15.5cm}
\begin{screen}
\begin{verbatim}
00000  acc_hop acc_ex  n_hop    n_ex     : Mon Jul 25 14:03:29 2016
00001  0.59688 0.00000 320      0        : Mon Jul 25 14:03:30 2016
00002  0.47727 0.00000 176      0        : Mon Jul 25 14:03:30 2016
00003  0.50000 0.00000 176      0        : Mon Jul 25 14:03:30 2016
00004  0.49432 0.00000 176      0        : Mon Jul 25 14:03:30 2016
00005  0.57386 0.00000 176      0        : Mon Jul 25 14:03:30 2016
00006  0.55114 0.00000 176      0        : Mon Jul 25 14:03:30 2016    
…
\end{verbatim}
\end{screen}
\end{minipage}

\subsection{xxx\_cisajs\_yyy.dat }
OneBodyG指定ファイルで指定した順序で一体Green関数の各成分が実部、虚部の順に出力されます。
xxxには\verb|CDataFileHead|で指定されるヘッダが、yyyには\verb|ModPara|ファイルの\verb|NDataIdxStart|, \verb|NDataQtySmp|に従い\verb|NDataIdxStart|$\cdots$\verb|NDataIdxStart|+\verb|NDataQtySmp|の順に記載されます。

\subsection{xxx\_cisajscktalt\_yyy.dat }
TwoBodyG指定ファイルで指定した順序で二体Green関数が出力されます。
xxxには\verb|CDataFileHead|で指定されるヘッダが、yyyには\verb|ModPara|ファイルの\verb|NDataIdxStart|, \verb|NDataQtySmp|に従い\verb|NDataIdxStart|$\cdots$\verb|NDataIdxStart|+\verb|NDataQtySmp|の順に記載されます。

%----------------------------------------------------------
%----------------------------------------------------------
%----------------------------------------------------------
