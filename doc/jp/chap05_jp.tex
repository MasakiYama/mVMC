% !TEX root = userguide_jp.tex
%----------------------------------------------------------
\chapter{アルゴリズム}
\label{Ch:algorithm}
\section{変分モンテカルロ法}
適当な完全系に対するマルコフ連鎖を構成して重みつきサンプリングを行います。
ここでは完全系として$S_z = 0$ の実空間配置$\{| x\rangle\}$ を使用します。
\begin{equation}
| x\rangle =  \prod_{n=1}^{N/2} c_{r_{n\uparrow}}^{\dag} \prod_{n=1}^{N/2} c_{r_{n\downarrow}}^{\dag} |0 \rangle
\end{equation}
ここで、$n$番目の$\sigma$電子の位置を$r_{n\sigma}$としました。
\subsection{Importance sampling}
マルコフ連鎖の重みを
\begin{equation}
\rho(x)=\frac{|\langle x| \psi \rangle|^2}{\langle \psi | \psi \rangle} \ge 0, \sum{x} \rho(x)=1
\end{equation}
とすると演算子$A$の期待値は
\begin{equation}
\langle A \rangle =\frac{\langle \psi| A| \psi \rangle}{\langle \psi | \psi \rangle} 
=\sum_x \frac{\langle \psi| A | x\rangle \langle x| \psi \rangle}{\langle \psi |\psi \rangle} 
=\sum_x \rho(x) \frac{\langle \psi| A | x\rangle }{\langle \psi |x \rangle} 
\end{equation}
となります。$x$に関する和を重み付きサンプリングに置き換えます。また、Local Green's function $G_{ij\sigma\sigma'}(x)$は
\begin{equation}
G_{ij\sigma\sigma'}(x)=\frac{\langle \psi | c_{i\sigma}^{\dag} c_{j\sigma'} | \psi \rangle}{\langle \psi | x \rangle}
\end{equation}
で定義されます。なお、サンプリングに使用する乱数生成については、メルセンヌツイスター法を使用しています\cite{Mutsuo2008}。

\section{Bogoliubov表現}\label{sec_bogoliubov_rep}

スピン系の計算において一体項(\verb|transfer|)、\verb|InterAll|形式での相互作用、
相関関数のインデックスの指定にはBogoliubov表現が使われています。
スピンの演算子は次のように生成$\cdot$消滅演算子で書き換えられます。
\begin{align}
  S_{i z} &= \sum_{\sigma = -S}^{S} \sigma c_{i \sigma}^\dagger c_{i \sigma}
  \\
  S_{i}^+ &= \sum_{\sigma = -S}^{S-1} 
  \sqrt{S(S+1) - \sigma(\sigma+1)} 
  c_{i \sigma+1}^\dagger c_{i \sigma}
  \\
  S_{i}^- &= \sum_{\sigma = -S}^{S-1} 
  \sqrt{S(S+1) - \sigma(\sigma+1)} 
  c_{i \sigma}^\dagger c_{i \sigma+1}
\end{align}

\section{{パフィアン行列式とスレータ行列式の関係}}
\label{sec:PuffAndSlater}
このセクションでは、パフィアン行列式とスレータ行列式の関係および$f_{ij}$の特異値分解の意味について説明します。
\subsection{$f_{ij}$と$\Phi_{in\sigma}$の関係~(スピン反平行の場合)}
パフィアンスレーター行列式(多変数変分モンテカルロ法の一体部分)は
\begin{equation}
|\phi_{\rm Pf}\rangle=\Big(\sum_{i,j=1}^{N_{s}}f_{ij}c_{i\uparrow}^{\dagger}c_{j\downarrow}^{\dagger}\Big)^{N_{\rm e}/2}|0\rangle,
\end{equation}
のように定義されます。ここで、$N_{s}$はサイト数、$N_{e}$は全電子数、$f_{ij}$は変分パラメータとしました。
簡単化のため、以降$f_{ij}$は実数と仮定します。また、シングルスレーター行列式として
\begin{align}
|\phi_{\rm SL}\rangle&=\Big(\prod_{n=1}^{N_{e}/2}\psi_{n\uparrow}^{\dagger}\Big)
\Big(\prod_{m=1}^{N_{e}/2}\psi_{m\downarrow}^{\dagger}\Big)|0\rangle, \\
\psi_{n\sigma}^{\dagger}&=\sum_{i=1}^{N_{s}}\Phi_{in\sigma}c^{\dagger}_{i\sigma}.
\end{align}
を定義します。ただし、$\Phi$は正規直行基底であり、クロネッカーのデルタ$\delta_{nm}$を用い
\begin{equation}
\sum_{i=1}^{N_{s}}\Phi_{in\sigma}\Phi_{im\sigma}=\delta_{nm},
\end{equation}
で表されます。この直交性の関係から、以下の関係式
\begin{align}
[\psi^{\dagger}_{n\sigma},\psi_{m\sigma}]_{+}&=\delta_{nm},\\
G_{ij\sigma}=\langle c_{i\sigma}^{\dagger}c_{j\sigma}\rangle 
&=\frac{\langle \phi_{\rm SL}| c_{i\sigma}^{\dagger}c_{j\sigma} | \phi_{\rm SL}\rangle}{\langle \phi_{\rm SL}|\phi_{\rm SL}\rangle } \\
&=\sum_{n} \Phi_{in\sigma} \Phi_{jn\sigma}.
\end{align}
が導かれます。

次に、$\phi_{\rm SL}$を変形し、$f_{ij}$と$\Phi_{in\sigma}$の関係をあらわにします。
$\psi^{\dagger}_{n\sigma}$の交換関係を用いると、$\phi_{\rm SL}$は
\begin{align}
|\phi_{\rm SL}\rangle \propto \prod_{n=1}^{N_{e}/2}\Big(\psi_{n\uparrow}^{\dagger}\psi_{\mu(n)\downarrow}^{\dagger}\Big)|0\rangle,
\end{align}
と書き換えられます。ここで、$\mu(n)$は$n= 1, 2, \cdots, N_{e}/2$の置換を表します。
ここで議論を簡単にするため、同一のペア$n=\mu(n)$を採用します。
このとき、$K_{n}^{\dagger}=\psi_{n\uparrow}^{\dagger}\psi_{n\downarrow}^{\dagger}$として、
$K_{n}^{\dagger}K_{m}^{\dagger}=K_{m}^{\dagger}K_{n}^{\dagger}$の関係を用いることで、
\begin{align}
|\phi_{\rm SL}\rangle &\propto \prod_{n=1}^{N_{e}/2}\Big(\psi_{n\uparrow}^{\dagger}\psi_{n\downarrow}^{\dagger}\Big)|0\rangle
=\prod_{n=1}^{N_{e}/2} K_{n}^{\dagger}|0\rangle \\
&\propto\Big(\sum_{n=1}^{\frac{N_{e}}{2}}K_{n}^{\dagger}\Big)^{\frac{N_{e}}{2}} |0\rangle
=\Big(\sum_{i,j=1}^{N_{s}}\Big[\sum_{n=1}^{\frac{N_{e}}{2}}\Phi_{in\uparrow}\Phi_{jn\downarrow}\Big]
c_{i\uparrow}^{\dagger}c_{j\downarrow}^{\dagger}\Big)|0\rangle,
\end{align}
の関係が得られます。これより$f_{ij}$はシングルスレーター行列式の係数により
\begin{align}
f_{ij}=\sum_{n=1}^{\frac{N_{e}}{2}}\Phi_{in\uparrow}\Phi_{jn\downarrow}.
\end{align}
として表されることが分かります。なお、この形式はシングルスレーター行列式で与えられる$f_{ij}$の表式の一つであり、
実際にはペアを組む自由度(どの$\mu(n)$を選ぶか)およびゲージの自由度(すなわち$\Phi$の符号の自由度)に依存します。
この自由度の多さが$f_{ij}$の冗長性につながっています。

\subsection{$F_{IJ}$と$\Phi_{In}$の関係~(スピン平行も含めた場合)}
以下の同種スピンのペアリングも考えたパフィアン波動関数と
スレーター波動関数を考えます(ここで$I,J$はスピン自由度も含めたサイトのインデックス).
\begin{align}
|\phi_{\rm Pf}\rangle&=\Big(\sum_{I,J=1}^{2N_{s}}f_{IJ}c_{I}^{\dagger}c_{J}^{\dagger}\Big)^{N_{\rm e}/2}|0\rangle, \\
|\phi_{\rm SL}\rangle&=\Big(\prod_{n=1}^{N_{e}}\psi_{n}^{\dagger}\Big)|0\rangle,~~\psi_{n}^{\dagger}=\sum_{I=1}^{2N_{s}}\Phi_{In}c^{\dagger}_{I}.
\end{align}
スピン反平行の場合とほぼ同様の議論を用いることで、
\begin{align}
F_{IJ}=\sum_{n=1}^{\frac{N_{e}}{2}}\Big(\Phi_{In}\Phi_{Jn+1}-\Phi_{Jn}\Phi_{In+1}\Big).
\end{align}
の関係を示すことができます。
これはスピン反平行の場合にもそのまま適用できるので、
mVMC ver1.0ではこの表式を使用しています。

\subsection{$f_{ij}$の特異値分解}
行列$F$, $\Phi_{\uparrow}$, $\Phi_{\downarrow}$,$\Sigma$を
\begin{align}
&(F)_{ij}=f_{ij},~~~ 
(\Phi_{\uparrow})_{in}=\Phi_{in\uparrow},~~~ 
(\Phi_{\downarrow})_{in}=\Phi_{in\downarrow}, \\
&\Sigma={\rm diag}[1,\cdots,1,0,0,0]~~~\text{({\rm \# of 1} = $N_{e}/2$)}.
\end{align}
として定義します。これらの記法を用いると、$f_{ij}$(すなわち$F$)の特異値分解は
\begin{align}
F=\Phi_{\uparrow}\Sigma\Phi_{\downarrow}^{t}.
\end{align}
として記述することができます。
この結果は、もし非ゼロの特異値が$N_{e}/2$個存在し、
かつ全ての$F$の非ゼロの特異値が$1$であった場合、
$f_{ij}$が平均場近似解として記述できることを示しています。
言い換えると、特異値の非ゼロ成分の数とその値が、
シングルスレータ行列式からパフィアンスレーター行列式がどのようにしてずれるのか、
という点について定量的な基準を与えることを示しています。

\section{Power-Lanczos法}
このセクションでは、Power-Lanczos法での最適化パラメータ$\alpha$の決定方法について述べます。
サンプリング数が有限のため、エネルギーの分散が常に正の値を取るよう計算を行うことが重要になります。
また、ここではシングルステップのLanczos法を適用した後の物理量の計算についても説明します。
\subsection{$\alpha$の決定}
最初に, 変分モンテカルロ法のサンプリングに関して簡単に説明します。
物理量$\hat{A}$は以下の手順で計算されます:
\begin{align}
&\langle \hat{A}\rangle = \frac{\langle \phi| \hat{A}|\phi \rangle}{\langle \phi| \phi \rangle} = \sum_{x} \rho(x) F(x, {\hat{A}}),\\
& \rho(x)=\frac{|\langle \phi|x\rangle|^2}{\langle \phi | \phi \rangle}, ~~~~F(x,  {\hat{A}}) =  \frac{\langle x| \hat{A}|\phi \rangle}{\langle x| \phi \rangle}.
\end{align}
演算子の積$\hat{A}\hat{B}$を計算する場合には、以下の二通りの方法があります。
\begin{align}
&\langle \hat{A} \hat{B}\rangle = \sum_{x} \rho(x) F(x, {\hat{A}\hat{B}}),\\
&\langle \hat{A} \hat{B}\rangle = \sum_{x} \rho(x) F^{\dag}(x, {\hat{A})F(x, \hat{B}}).
\end{align}
後述するように、後者の表記の方が数値的には安定します。
例えば、エネルギーの期待値の分散$\sigma^2=\langle (\hat{H}-\langle \hat{H}\rangle)^2\rangle$を考えてみると、
分散は以下の2通りの方法で計算できます。
\begin{align}
&\sigma^2=\sum_{x} \rho(x) F(x,  (\hat{H}-\langle \hat{H}\rangle)^2) = \sum_{x} \rho(x) F(x,  \hat{H}^2) - \left[ \sum_{x} \rho(x) F(x,  \hat{H})\right]^2 ,\\
&\sigma^2=\sum_{x} \rho(x) F^{\dag}(x,  \hat{H}-\langle \hat{H}\rangle)F(x,  \hat{H}-\langle \hat{H}\rangle) = \sum_{x} \rho(x) F^{\dag}(x,  \hat{H}) F(x, \hat{H})- \left[ \sum_{x} \rho(x) F(x,  \hat{H})\right]^2 
\end{align}
この定義から、後者の方法では常に正の値となることが保証されているのに対して、前者の方法では分散が正の値になることが必ずしも保証されないことが分かります。次に, シングルステップでのpower-Lanczos波動関数$|\phi\rangle =(1+\alpha \hat{H}) |\psi \rangle$に対するエネルギーの期待値とその分散を考えます。エネルギーは以下の式で計算されます:
\begin{align}
E_{LS}(\alpha) =\frac{\langle \phi| \hat{H} |\phi\rangle}{\langle \phi|\phi\rangle}=\frac{h_1 + \alpha(h_{2(20)} + h_{2(11)}) + \alpha^2 h_{3(12)}}{1 + 2\alpha h_1 + \alpha^2 h_{2(11)}},
\end{align}
ここで、$h_1$, $h_{2(11)},~h_{2(20)},$ and $h_{3(12)}$を以下のように定義しました:
\begin{align}
&h_1 =\sum_{x} \rho(x) F^{\dag}(x,  \hat{H}),\\
&h_{2(11)}=\sum_{x} \rho(x) F^{\dag}(x,  \hat{H}) F(x, \hat{H}),\\
&h_{2(20)}=\sum_{x} \rho(x) F^{\dag}(x,  \hat{H}^2),\\
&h_{3(12)}=\sum_{x} \rho(x) F^{\dag}(x,  \hat{H})F(x,  \hat{H}^2).
\end{align}
$\frac{\partial E_{LS}(\alpha)}{\partial \alpha}=0$の条件から$\alpha$の二次方程式が導出され, それを解くことで$\alpha$が決定されます。
分散に関しても同様の手法で計算することが可能です。
\subsection{物理量の計算}
最適化パラメータ$\alpha$を用いることで, 演算子$\hat{A}$の期待値を以下の式から計算することが出来ます: 
\begin{align}
A_{LS}(\alpha) =\frac{\langle \phi| \hat{A} |\phi\rangle}{\langle \phi|\phi\rangle}=\frac{A_0 + \alpha(A_{1(10)} + A_{1(01)}) + \alpha^2 A_{2(11)}}{1 + 2\alpha h_1 + \alpha^2 h_{2(11)}},
\end{align}
ここで、$A_0$, $A_{1(10)},~A_{1(01)},$ and $A_{2(11)}$は
\begin{align}
&A_0 =\sum_{x} \rho(x) F(x,  \hat{A}),\\
&A_{1(10)}=\sum_{x} \rho(x) F^{\dag}(x,  \hat{H}) F(x, \hat{A}),\\
&A_{1(01)}=\sum_{x} \rho(x) F(x, \hat{A}\hat{H}),\\
&A_{2(11)}=\sum_{x} \rho(x) F^{\dag}(x,  \hat{H})F(x,  \hat{A}\hat{H}).
\end{align}
として定義される変数を表します。プログラムでは、この表式に基づき一体グリーン関数および二体グリーン関数の計算を行っています。
%----------------------------------------------------------
