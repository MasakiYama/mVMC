% !TEX root = userguide_jp.tex
%----------------------------------------------------------
\chapter{What is mVMC?}
\label{Ch:whatismVMC}
%----------------------------------------------------------
%----------------------------------------------------------
%----------------------------------------------------------
%----------------------------------------------------------
\section{mVMCとは?}
\subsection{プログラム概要}
このプログラムを利用することで以下の事項が計算可能です。
\begin{itemize}
\item{与えられた変分自由度の範囲でハミルトニアンの期待値が最小 (極小) 値を持つような変分波動関数 を数値的に生成します.   量子数で分割された部分空間に限定して計算することも可能です。}
\item{得られた変分波動関数における各種物理量 (相関関数など) の期待値を計算することができます。}
%\item{
%特定の条件 (ハミルトニアンの相互作用項が実空間で対角的かつ全て遍歴電子で構成された模型) を 持つ場合には Power Lanczos (Single Lanczos %Step) 法 を適用した場合の期待値を計算することがで きます。}
\end{itemize}
mVMCでは以下の流れで計算を行います。
\begin{enumerate}
\item{入力ファイル(*.def)の読込}
\item{$\langle {\cal H} \rangle$を最小化するように変分パラメータ$\vec{\alpha}$を最適化}
\item{一体・二体Green関数の計算}
\item{変分パラメータ・期待値の出力}
\end{enumerate}
計算では「実空間配置 $|x\rangle$の生成からサンプリングまでを並列して行い、期待値を計算する際に一つにまとめる」という単純な並列化を行っています。各計算機クラスターで所定の手続きに従って、 並列数を指定すれば MPI を用いた並列計算を勝手に行われますが、並列を行わない場合 (single core) もMPI を呼び出すため、MPI ジョブを禁止してる環境 (物性研 system B のフロントエンドなど) ではプログラム実行することができません。なお、本プログラムではパフィアンの計算にあたりPFAPACKを利用した計算を行っています\cite{PFAPACK}。

\subsection{ライセンス}
本ソフトウェアのプログラムパッケージおよびソースコード一式はGNU General Public License version 3(GPL v3)に準じて配布されています。

\subsection{コピーライト}
\begin{quote}
{\it \copyright  2016- The University of Tokyo.} {\it All rights reserved.}
\end{quote}
本ソフトウェアは2016年度 東京大学物性研究所 ソフトウェア高度化プロジェクトの支援を受け開発されており、その著作権は東京大学が所持しています。

\subsection{開発貢献者}
\label{subsec:developers}
本ソフトウェアは以下の開発貢献者により開発されています。
\begin{itemize}
\item{ver.1.0 (xxxxリリース)}
\begin{itemize}
\item{開発者}
	\begin{itemize}
	\item{三澤 貴宏 (東京大学 物性研究所)}
	\item{森田 悟史 (東京大学 物性研究所)}
	\item{大越 孝洋 (東京大学 大学院工学系研究科)}
	\item{井戸 康太 (東京大学 大学院工学系研究科)}
	\item{今田 正俊 (東京大学 大学院工学系研究科)}
	\item{河村 光晶 (東京大学 物性研究所)}
	\item{吉見 一慶 (東京大学 物性研究所)}
	\end{itemize}

\item{プロジェクトコーディネーター}
	\begin{itemize}
	\item{加藤 岳生 (東京大学 物性研究所)}
	\end{itemize}

\end{itemize}

\end{itemize}


\section{動作環境}
 以下の環境で動作することを確認しています。\tr{(要確認)}
\begin{itemize}
\item 東京大学物性研究所スーパーコンピューターシステムB「sekirei」
\item 同システムC「maki」(FX10)
\item 京コンピューター
\item OpenMPI + Intel Compiler + MKL
\item MPICH + Intel Compiler + MKL
\item MPICH + GNU Compiler + MKL
\end{itemize}
