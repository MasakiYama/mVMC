% !TEX root = userguide_jp.tex
%----------------------------------------------------------
\chapter{チュートリアル}
\label{Ch:model}

\section{サンプルファイル一覧}

mVMCでは\verb|sample/Standard/|以下に次のサンプルを用意しています。

\begin{itemize}
\item 2次元正方格子Hubbardモデル

  (\verb|sample/Standard/Hubbard/square/|)
\item 2次元三角格子Hubbardモデル

  (\verb|sample/Standard/Hubbard/triangular/|)
\item 1次元近藤格子モデル

  (\verb|sample/Standard/Kondo/chain/|)
\item 1次元反強磁性的Heisenbergモデル

  (\verb|sample/Standard/Spin/HeisenbergChain/|)
\item 2次元正方格子反強磁性的Heisenbergモデル

  (\verb|sample/Standard/Spin/HeisenbergSquare/|)
  
\item 2次元カゴメ格子反強磁性的Heisenbergモデル

  (\verb|sample/Standard/Spin/Kagome/|)

\end{itemize}

これらのチュートリアルの実行方法は全て同じ手順で実行することが可能です。
以下ではHeisenberg模型について説明します。


% !TEX root = userguide_jp.tex
\section{Heisenberg模型}

以下のチュートリアルはディレクトリ
\begin{verbatim}
sample/Standard/Spin/HeisenbergChain/
\end{verbatim}
内で行います。
Heisenberg模型におけるサンプル入力ファイル、参照用出力ディレクトリ、標準出力リダイレクトはそれぞれ
\begin{verbatim}
samples/Standard/Spin/HeisenbergChain/StdFace.def
samples/Standard/Spin/HeisenbergChain/reference/
\end{verbatim}
にあります。
この例では1次元のHeisenberg鎖(最近接サイト間の反強磁性的スピン結合のみを持つ)を考察します。
\begin{align}
  {\hat H} = J \sum_{i=1}^{N_{\rm site}} {\hat {\boldsymbol S}}_i \cdot {\hat {\boldsymbol S}}_{i+1}
\end{align}

インプットファイルの中身は次のとおりです。
\\
\begin{minipage}{10cm}
\begin{screen}
\begin{verbatim}
L = 16
Lsub=4
model = "Spin"
lattice = "chain lattice"
J = 1.0
2Sz = 0
NMPtrans=1
\end{verbatim}
\end{screen}
\end{minipage}
%
\\
この例ではスピン結合$J=1$(任意単位)とし、サイト数は16としました。

\subsubsection{詳細入力ファイル作成}
スタンダードモードでは詳細入力ファイルの作成を最初に行う必要があります。
実行コマンドと標準出力は次のとおりです。

\vspace{1cm}\hspace{-0.7cm}
\verb|$ |\underline{パス}\verb|/vmcdry.out StdFace.def|
\small
\begin{verbatim}
######  Standard Intarface Mode STARTS  ######

  Open Standard-Mode Inputfile StdFace.def

  KEYWORD : l                    | VALUE : 16
  KEYWORD : lsub                 | VALUE : 4
  KEYWORD : model                | VALUE : spin
  KEYWORD : lattice              | VALUE : chain
  KEYWORD : j                    | VALUE : 1.0
  KEYWORD : 2sz                  | VALUE : 0
  KEYWORD : nmptrans             | VALUE : 1

#######  Parameter Summary  #######

  @ Lattice Size & Shape

                L = 16
             Lsub = 4
                L = 16
                W = 1
           phase0 = 1.00000    0.00000     ######  DEFAULT VALUE IS USED  ######

  @ Hamiltonian

               2S = 1           ######  DEFAULT VALUE IS USED  ######
                h = 0.00000     ######  DEFAULT VALUE IS USED  ######
            Gamma = 0.00000     ######  DEFAULT VALUE IS USED  ######
                D = 0.00000     ######  DEFAULT VALUE IS USED  ######
              J0x = 1.00000
              J0y = 1.00000
              J0z = 1.00000

  @ Numerical conditions

             Lsub = 4
             Wsub = 1
      ioutputmode = 1           ######  DEFAULT VALUE IS USED  ######

######  Print Expert input files  ######

    qptransidx.def is written.
         filehead = zvo         ######  DEFAULT VALUE IS USED  ######
         filehead = zqp         ######  DEFAULT VALUE IS USED  ######
      NVMCCalMode = 0           ######  DEFAULT VALUE IS USED  ######
     NLanczosMode = 0           ######  DEFAULT VALUE IS USED  ######
    NDataIdxStart = 1           ######  DEFAULT VALUE IS USED  ######
      NDataQtySmp = 5           ######  DEFAULT VALUE IS USED  ######
      NSPGaussLeg = 8           ######  DEFAULT VALUE IS USED  ######
          NSPStot = 0           ######  DEFAULT VALUE IS USED  ######
         NMPTrans = 1
    NSROptItrStep = 1200        ######  DEFAULT VALUE IS USED  ######
     NSROptItrSmp = 100         ######  DEFAULT VALUE IS USED  ######
     NSROptFixSmp = 1           ######  DEFAULT VALUE IS USED  ######
       NVMCWarmUp = 10          ######  DEFAULT VALUE IS USED  ######
    NVMCIniterval = 1           ######  DEFAULT VALUE IS USED  ######
       NVMCSample = 100         ######  DEFAULT VALUE IS USED  ######
          RndSeed = 123456789   ######  DEFAULT VALUE IS USED  ######
       NSplitSize = 1           ######  DEFAULT VALUE IS USED  ######
           NStore = 0           ######  DEFAULT VALUE IS USED  ######
     DSROptRedCut = 0.00100     ######  DEFAULT VALUE IS USED  ######
     DSROptStaDel = 0.02000     ######  DEFAULT VALUE IS USED  ######
     DSROptStepDt = 0.02000     ######  DEFAULT VALUE IS USED  ######
              2Sz = 0
    locspn.def is written.
    trans.def is written.
    interall.def is written.
    jastrowidx.def is written.
    coulombintra.def is written.
    coulombinter.def is written.
    hund.def is written.
    exchange.def is written.
    orbitalidx.def is written.
    gutzwilleridx.def is written.
    namelist.def is written.
    modpara.def is written.
    greenone.def is written.
    greentwo.def is written.

######  Input files are generated.  ######
\end{verbatim}
\normalsize

この実行では、ハミルトニアンの詳細を記述するファイル
\verb|locspin.def|、\verb|trans.def|、\verb|coulombinter.def|、\verb|coulombintra.def|、
\verb|exchange.def|、\verb|hund.def|、\verb|namelist.def|、\verb|calcmod.def|、\verb|modpara.def|
と、変分パラメータを設定するファイル
\verb|gutzwilleridx.def|、\verb|jastrowidx.def|、\verb|orbitalidx.def|、\verb|qptransidx.def|、
結果として出力する相関関数の要素を指定するファイル
\verb|greenone.def|、\verb|greentwo.def|が生成されます。
各ファイルの詳細についてはSec. \ref{Ch:HowToExpert}をご覧ください。

\subsubsection{計算実行}
作成した詳細入力ファイルを読み込み計算を行います。
実行コマンドと標準出力は次のとおりです。

\vspace{1cm}\hspace{-0.7cm}
\verb|$ mpiexec -np |\underline{プロセス数}\verb| |\underline{パス}\verb|/vmc.out namelist.def|
\small

\begin{verbatim}
-----------
Start: Read *def files.
  Read File namelist.def .
  Read File 'modpara.def' for ModPara.
  Read File 'locspn.def' for LocSpin.
  Read File 'trans.def' for Trans.
  Read File 'coulombintra.def' for CoulombIntra.
  Read File 'coulombinter.def' for CoulombInter.
  Read File 'hund.def' for Hund.
  Read File 'exchange.def' for Exchange.
  Read File 'gutzwilleridx.def' for Gutzwiller.
  Read File 'jastrowidx.def' for Jastrow.
  Read File 'orbitalidx.def' for Orbital.
  Read File 'qptransidx.def' for TransSym.
  Read File 'greenone.def' for OneBodyG.
  Read File 'greentwo.def' for TwoBodyG.
End  : Read *def files.
Start: Read parameters from *def files.
End  : Read parameters from *def files.
Start: Set memories.
End  : Set memories.
Start: Initialize parameters.
End  : Initialize parameters.
Start: Initialize variables for quantum projection.
End  : Initialize variables for quantum projection.
Start: Optimize VMC parameters.
End  : Optimize VMC parameters.
-----------
\end{verbatim}

計算実行中に以下のファイルが情報として出力されます。
\\
\begin{minipage}{12cm}
\begin{screen}
\begin{verbatim}
zvo_SRinfo.dat zvo_cfg_001.dat zvo_out_001.dat
zvo_time_001.dat  zvo_var_001.dat
\end{verbatim}
\end{screen}
\end{minipage}

なお、\verb|zvo_out_001.dat|には、ビン毎の計算情報として、
\begin{equation}
\langle H \rangle, \langle H^2 \rangle, \frac{\langle H^2 \rangle- \langle H \rangle^2 }{\langle H \rangle^2} \nonumber
\end{equation}
が順に出力されますので、収束性の目安として利用することが可能です。各ファイルの詳細についてはSec. \ref{Sec:outputfile}をご覧ください。\\

\subsubsection{計算結果出力}
計算が正常終了すると、エネルギー、エネルギーの分散、変分パラメータおよび計算実行時間を記載したファイルが出力されます。
以下に、このサンプルでの出力ファイルを記載します。\\
\begin{minipage}{12cm}
\begin{screen}
\begin{verbatim}
gutzwiller_opt.dat
jastrow_opt.dat
orbital_opt.dat
zqp_opt.dat
ClacTimer.dat
\end{verbatim}
\end{screen}
\end{minipage}

各ファイルの詳細についてはSec. \ref{Sec:outputfile}をご覧ください。


\subsubsection{Green関数の計算}
\verb|modpara.def|ファイル中の\verb|NVMCCalMode|を0から1に変更の上、以下のコマンドを実行します。
実行時に\verb|namelist.dat|ファイルの後ろに\verb|zqp_opt.dat|を付け加えることで、
一つ前の計算で最適化された変分パラメータを使用した計算が行われます。

\vspace{1cm}\hspace{-0.7cm}
\verb|$ |\underline{パス}\verb|/vmc.out namelist.def zqp_opt.dat|
\small

計算が終了すると以下のファイルが出力されます。
\\
\begin{minipage}{12cm}
\begin{screen}
\begin{verbatim}
zvo1_cisajs_001.dat
zvo1_cisajscktalt_001.dat
\end{verbatim}
\end{screen}
\end{minipage}
\\
各ファイルの詳細についてはSec. \ref{Sec:outputfile}をご覧ください。



\section{エキスパートユーザー向け}
mVMCでは、以下の6つに分類される入力ファイルを読み込み、計算実行を行います。そのため、以下のファイルを直接作成・指定することで、より複雑な計算を行うことが可能です。各ファイルの詳細についてはSec. \ref{Ch:HowToExpert}をご覧ください。
\\
\begin{description}
\item[(1)~List:]
~\\{キーワード指定なし}:
使用するinput fileの名前のリストを書きます。なお、ファイル名は任意に指定することができます。
\item[(2)~Basic parameters:]
~\\{\bf ModPara}: 計算時に必要な基本的なパラメーター(サイトの数、電子数、Lanczosステップを何回やるかなど)を設定します。
~\\{\bf LocSpin}: 局在スピンの位置を設定します(近藤模型でのみ利用)。
\item[(3)~Set Hamiltonian:] 
~\\以下のファイルを用い、Hamiltonianを電子系の表式により指定します。
~\\{\bf Trans}: $c_{i\sigma_1}^{\dag}c_{j\sigma_2}$で表される一体項を指定します。
~\\{\bf InterAll}: $c_ {i \sigma_1}^{\dag}c_{j\sigma_2}c_{k \sigma_3}^{\dag}c_{l \sigma_4}$で表される一般二体相互作用を指定します。\\
~\\なお、使用頻度の高い相互作用に関しては下記のキーワードで指定することも可能です。
~\\{\bf CoulombIntra}: $n_ {i \uparrow}n_{i \downarrow}$で表される相互作用を指定します($n_{i \sigma}=c_{i\sigma}^{\dag}c_{i\sigma}$)。
~\\{\bf CoulombInter}: $n_ {i}n_{j}$で表される相互作用を指定します($n_i=n_{i\uparrow}+n_{i\downarrow}$)。
~\\{\bf Hund}: $n_{i\uparrow}n_{j\uparrow}+n_{i\downarrow}n_{j\downarrow}$で表される相互作用を指定します。
~\\{\bf PairHop}:  $c_ {i \uparrow}^{\dag}c_{j\uparrow}c_{i \downarrow}^{\dag}c_{j  \downarrow}$で表される相互作用を指定します。
~\\{\bf Exchange}: $S_i^+ S_j^-$で表される相互作用を指定します。
\item[(4)~Set condition of variational parameters :] 
~\\変分波動関数は
\begin{equation}
|\psi \rangle = {\cal P}_G{\cal P}_J{\cal P}_{d-h}^{(2)}{\cal P}_{d-h}^{(4)}{\cal L}^S{\cal L}^K{\cal L}^P |\phi_{\rm pair} \rangle,
\end{equation}
で与えられます。ここで、一体部分は実空間のペア関数
\begin{equation}
|\phi_{\rm pair} \rangle = \left[\sum_{i, j=1}^{N_s} f_{ij}c_{i\uparrow}^{\dag}c_{j\downarrow}^{\dag} \right]^{N/2}|0 \rangle,
\end{equation}
を用いた波動関数で表されます。ここで$N$は全電子数、$N_s$は全サイト数です。
変分パラメータの初期値は以下のファイルを用いて指定します。
~\\{\bf Gutzwiller}: ${\cal P}_G=\exp\left[ \sum_i g_i n_{i\uparrow} n_{i\downarrow} \right]$のうち、最適化の対象とする変分パラメータ$g_i$を指定します。
~\\{\bf Jastrow}: ${\cal P}_J=\exp\left[\frac{1}{2} \sum_{i\neq j} v_{ij} n_i n_j\right]$のうち、最適化の対象とする変分パラメータ$v_{ij}$を指定します。
~\\{\bf DH2}:  ${\cal P}_{d-h}^{(2)}= \exp \left[ \sum_t \sum_{n=0}^2 (\alpha_{2nt}^d \sum_{i}\xi_{i2nt}^d+\alpha_{2nt}^h \sum_{i}\xi_{i2nt}^h)\right]$で表される2サイトのdoublon-holon相関因子を指定します。詳細はDH2ファイルの説明を参照してください。
~\\{\bf DH4}:  ${\cal P}_{d-h}^{(4)}= \exp \left[ \sum_t \sum_{n=0}^4 (\alpha_{4nt}^d \sum_{i}\xi_{i4nt}^d+\alpha_{4nt}^h \sum_{i}\xi_{i4nt}^h)\right]$で表される4サイトのdoublon-holon相関因子を指定します。詳細はDH4ファイルの説明を参照してください。
~\\{\bf Orbital}: ペア軌道$|\phi_{\rm pair} \rangle = \left[\sum_{i, j=1}^{N_s} f_{ij}c_{i\uparrow}^{\dag}c_{j\downarrow}^{\dag} \right]^{N/2}|0 \rangle$を設定します。
~\\{\bf TransSym}: 運動量射影${\cal L}_K=\frac{1}{N_s}\sum_{{\bm R}}e^{i {\bm K} \cdot{\bm R} } \hat{T}_{\bm R}$と格子対称性射影${\cal L}_P=\sum_{\alpha}p_{\alpha} \hat{G}_{\alpha}$に関する指定を行います。ここで、${\bm K}$は全運動量、$\hat{T}_{\bm R}$は並進ベクトル${\bm R}$に対応する並進演算子、$\hat{G}_{\alpha}$は格子の点群演算子、$p_\alpha$はパリティをそれぞれ表します。

\item[(5)~Initial variational parameters:]
~\\ 変分パラメータに関する初期値を与えます。キーワード指定されない場合には$0$が初期値として設定されます。
~\\{\bf InGutzwiller}: ${\cal P}_G=\exp\left[ \sum_i g_i n_{i\uparrow} n_{i\downarrow} \right]$のうち、変分パラメータ$g_i$の初期値を指定します。
~\\{\bf InJastrow}: ${\cal P}_J=\exp\left[\frac{1}{2} \sum_{i\neq j} v_{ij} n_i n_j\right]$のうち、変分パラメータ$v_{ij}$の初期値を指定します。
~\\{\bf InDH2}:  ${\cal P}_{d-h}^{(2)}= \exp \left[ \sum_t \sum_{n=0}^2 (\alpha_{2nt}^d \sum_{i}\xi_{i2nt}^d+\alpha_{2nt}^h \sum_{i}\xi_{i2nt}^h)\right]$で表される2サイトのdoublon-holon相関因子$\alpha_{2nt}^{d(h)}$の初期値を指定します。
~\\{\bf InDH4}:  ${\cal P}_{d-h}^{(4)}= \exp \left[ \sum_t \sum_{n=0}^4 (\alpha_{4nt}^d \sum_{i}\xi_{i4nt}^d+\alpha_{4nt}^h \sum_{i}\xi_{i4nt}^h)\right]$で表される4サイトのdoublon-holon相関因子$\alpha_{4nt}^{d(h)}$の初期値を指定します。
~\\{\bf InOrbital}: ペア軌道$|\phi_{\rm pair} \rangle = \left[\sum_{i, j=1}^{N_s} f_{ij}c_{i\uparrow}^{\dag}c_{j\downarrow}^{\dag} \right]^{N/2}|0 \rangle$の$ f_{ij}$に関する初期値を設定します。

\item[(6)~Output:]
~\\{\bf OneBodyG }:出力する一体Green関数を指定します。
 $\langle c^{\dagger}_{i\sigma_1}c_{j\sigma_2}\rangle$が出力されます。

 {\bf TwoBodyG }:出力する二体Green関数を指定します。
 $\langle c^{\dagger}_{i\sigma_1}c_{j\sigma_2}c^{\dagger}_{k \sigma_3}c_{l\sigma_4}\rangle$
が出力されます。
\end{description}

