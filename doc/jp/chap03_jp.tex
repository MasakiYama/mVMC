% !TEX root = userguide_jp.tex
%----------------------------------------------------------
\chapter{チュートリアル}
\label{Ch:model}
\section{スタンダードモード}
\label{Sec:StandardMode}

% !TEX root = userguide_jp.tex
\subsection{Heisenberg模型}

以下のチュートリアルはディレクトリ
\begin{verbatim}
sample/Standard/Spin/HeisenbergChain/
\end{verbatim}
内で行います。
Heisenberg模型におけるサンプル入力ファイル、参照用出力ディレクトリ、標準出力リダイレクトはそれぞれ
\begin{verbatim}
samples/Standard/Spin/HeisenbergChain/StdFace.def
samples/Standard/Spin/HeisenbergChain/reference/
\end{verbatim}
にあります。
この例では1次元のHeisenberg鎖(最近接サイト間の反強磁性的スピン結合のみを持つ)を考察します。
\begin{align}
  {\hat H} = J \sum_{i=1}^{N_{\rm site}} {\hat {\boldsymbol S}}_i \cdot {\hat {\boldsymbol S}}_{i+1}
\end{align}
インプットファイルの中身は次のとおりです。
\\
\begin{minipage}{10cm}
\begin{screen}
\begin{verbatim}
L = 16
Lsub=4
model = "Spin"
lattice = "chain lattice"
J = 1.0
2Sz = 0
NMPtrans=1
\end{verbatim}
\end{screen}
\end{minipage}
%
\\
この例ではスピン結合$J=1$(任意単位)とし、サイト数は16としました。

\subsubsection{詳細入力ファイル作成}
スタンダードモードでは詳細入力ファイルの作成を最初に行う必要があります。
実行コマンドと標準出力は次のとおりです。

\vspace{1cm}\hspace{-0.7cm}
\verb|$ |\underline{パス}\verb|/vmcdry.out StdFace.def|
\small
\begin{verbatim}
######  Standard Intarface Mode STARTS  ######

  Open Standard-Mode Inputfile StdFace.def

  KEYWORD : l                    | VALUE : 16
  KEYWORD : lsub                 | VALUE : 4
  KEYWORD : model                | VALUE : spin
  KEYWORD : lattice              | VALUE : chain
  KEYWORD : j                    | VALUE : 1.0
  KEYWORD : 2sz                  | VALUE : 0
  KEYWORD : nmptrans             | VALUE : 1

#######  Parameter Summary  #######

  @ Lattice Size & Shape

                L = 16
             Lsub = 4
                L = 16
                W = 1
           phase0 = 1.00000    0.00000     ######  DEFAULT VALUE IS USED  ######

  @ Hamiltonian

               2S = 1           ######  DEFAULT VALUE IS USED  ######
                h = 0.00000     ######  DEFAULT VALUE IS USED  ######
            Gamma = 0.00000     ######  DEFAULT VALUE IS USED  ######
                D = 0.00000     ######  DEFAULT VALUE IS USED  ######
              J0x = 1.00000
              J0y = 1.00000
              J0z = 1.00000

  @ Numerical conditions

             Lsub = 4
             Wsub = 1
      ioutputmode = 1           ######  DEFAULT VALUE IS USED  ######

######  Print Expert input files  ######

    qptransidx.def is written.
         filehead = zvo         ######  DEFAULT VALUE IS USED  ######
         filehead = zqp         ######  DEFAULT VALUE IS USED  ######
      NVMCCalMode = 0           ######  DEFAULT VALUE IS USED  ######
     NLanczosMode = 0           ######  DEFAULT VALUE IS USED  ######
    NDataIdxStart = 1           ######  DEFAULT VALUE IS USED  ######
      NDataQtySmp = 5           ######  DEFAULT VALUE IS USED  ######
      NSPGaussLeg = 8           ######  DEFAULT VALUE IS USED  ######
          NSPStot = 0           ######  DEFAULT VALUE IS USED  ######
         NMPTrans = 1
    NSROptItrStep = 1200        ######  DEFAULT VALUE IS USED  ######
     NSROptItrSmp = 100         ######  DEFAULT VALUE IS USED  ######
     NSROptFixSmp = 1           ######  DEFAULT VALUE IS USED  ######
       NVMCWarmUp = 10          ######  DEFAULT VALUE IS USED  ######
    NVMCIniterval = 1           ######  DEFAULT VALUE IS USED  ######
       NVMCSample = 100         ######  DEFAULT VALUE IS USED  ######
          RndSeed = 123456789   ######  DEFAULT VALUE IS USED  ######
       NSplitSize = 1           ######  DEFAULT VALUE IS USED  ######
           NStore = 0           ######  DEFAULT VALUE IS USED  ######
     DSROptRedCut = 0.00100     ######  DEFAULT VALUE IS USED  ######
     DSROptStaDel = 0.02000     ######  DEFAULT VALUE IS USED  ######
     DSROptStepDt = 0.02000     ######  DEFAULT VALUE IS USED  ######
              2Sz = 0
    locspn.def is written.
    trans.def is written.
    interall.def is written.
    jastrowidx.def is written.
    coulombintra.def is written.
    coulombinter.def is written.
    hund.def is written.
    exchange.def is written.
    orbitalidx.def is written.
    gutzwilleridx.def is written.
    namelist.def is written.
    modpara.def is written.
    greenone.def is written.
    greentwo.def is written.

######  Input files are generated.  ######
\end{verbatim}
\normalsize

この実行では、ハミルトニアンの詳細を記述するファイル
\verb|locspin.def|、\verb|trans.def|、\verb|coulombinter.def|、\verb|coulombintra.def|、
\verb|exchange.def|、\verb|hund.def|、\verb|namelist.def|、\verb|calcmod.def|、\verb|modpara.def|
と、変分パラメータを設定するファイル
\verb|gutzwilleridx.def|、\verb|jastrowidx.def|、\verb|orbitalidx.def|、\verb|qptransidx.def|、
結果として出力する相関関数の要素を指定するファイル
\verb|greenone.def|、\verb|greentwo.def|が生成されます。
これらのファイルはエキスパートモードと共通です。

\subsubsection{計算実行}
作成した詳細入力ファイルを読み込み計算を行います。
実行コマンドと標準出力は次のとおりです。

\vspace{1cm}\hspace{-0.7cm}
\verb|$ mpiexec -np |\underline{プロセス数}\verb| |\underline{パス}\verb|/vmc.out namelist.def|
\small

計算実行を行うと以下のログが出力されます。

\subsubsection{計算結果出力}
\tr{T.B.D. }
計算が正常終了すると、エネルギー、エネルギーの分散、変分パラメータおよび一体グリーン関数、二体グリーン関数が計算され、ファイル出力されます。
以下に、このサンプルでの出力ファイル例を記載します。\\
\begin{minipage}{12cm}
\begin{screen}
\begin{verbatim}
zvo_energy.dat zvo_cisajs.dat 
zvo_cisajscktalt.dat  
\end{verbatim}
\end{screen}
\end{minipage}

スタンダードモードの場合は、\verb|greenone.def|、\verb|greentwo.def|に基づき、一体グリーン関数には$\langle n_{i\sigma} \rangle$、二体グリーン関数には$\langle n_{i\sigma} n_{j\sigma'} \rangle$が自動出力されます。なお、Lanczos法で求めた固有ベクトルが十分な精度を持つ場合にはその固有ベクトルで計算されます。一方、Lanczos法で求めた固有ベクトルが十分な精度を持たない場合には、ログ出力に「Accuracy of Lanczos vetor is not enough」が表示され、CG法で固有ベクトルが求められます。各ファイルの詳細は\ref{subsec:energy.dat}, \ref{Subsec:cgcisajs}, \ref{Subsec:cisajscktalt}に記載がありますので、ご参照ください。


\subsection{その他の系でのチュートリアル}

\verb|samples/Standard/|以下には次のチュートリアルが含まれています。

\begin{itemize}
\item 2次元正方格子Hubbardモデル

  (\verb|samples/Standard/Hubbard/square/|)
\item 2次元三角格子Hubbardモデル

  (\verb|samples/Standard/Hubbard/triangular/|)
\item 1次元近藤格子モデル

  (\verb|samples/Standard/Kondo/chain/|)
\item 1次元反強磁性的Heisenbergモデル

  (\verb|samples/Standard/Spin/HeisenbergChain/HeisenbergChain/|)
\item 2次元正方格子反強磁性的Heisenbergモデル

  (\verb|samples/Standard/Spin/HeisenbergSquare/|)
\item Kitaevモデル(Honeycomb格子、2$\times$3サイト)

  (\verb|samples/Standard/Spin/Kitaev/|)

\end{itemize}

これらのチュートリアルの実行方法は前節と同じです。

%----------------------------------------------------------
\newpage
\section{エキスパートモード}
エキスパートモードでは、入力ファイルとして
\begin{enumerate}
\item 入力ファイルリスト
\item 基本パラメータ用ファイル
\item Hamiltonian作成用ファイル
\item 出力結果指定用ファイル
\end{enumerate}
を用意した後、計算を行います。計算開始後に関しては、スタンダードモードと同様です。ここでは"{\it InstallDir}/samples/Expert/Spin/HeisenbergChain"にある16サイトの一次元鎖量子ハイゼンベルグ模型
\begin{equation}
H=\sum_{i=0}^{15} J {\bm S}_i\cdot {\bm S}_{i+1}
\end{equation}
を例に入力ファイルの作成に関する説明を行います(ただし、$J=2$, ${\bm S}_{16}={\bm S}_{0}$)。なお、サンプルディレクトリの中には下記の入力ファイルが用意されています。\\
\begin{minipage}{15cm}
\begin{screen}
\begin{verbatim}
calcmod.def   greentwo.def  namelist.def  zTrans.def
greenone.def  modpara.def   zInterAll.def zlocspn.def
\end{verbatim}
\end{screen}
\end{minipage}

\subsection{入力ファイルリストファイル}
入力ファイルの種類と名前を指定するファイルnamelist.defには、下記の内容が記載されています。各入力ファイルリストファイルでは、行毎にKeywordとファイル名を記載し、ファイルの種類の区別を行います。詳細は\ref{Subsec:InputFileList}をご覧ください。
\\
\begin{minipage}{15cm}
\begin{screen}
\begin{verbatim}
CalcMod calcmod.def
ModPara modpara.def
LocSpin zlocspn.def
Trans zTrans.def
InterAll zInterAll.def
OneBodyG greenone.def
TwoBodyG greentwo.def
\end{verbatim}
\end{screen}
\end{minipage}

\subsection{基本パラメータの指定}
計算モード、計算用パラメータ、局在スピンの位置を以下のファイルで指定します。
\begin{description}
\item {\bf 計算モードの指定}\\
CalcModでひも付けられるファイル(ここではcalcmod.def)で計算モードを指定します。ファイルの中身は下記の通りです。\\
\begin{minipage}{15cm}
\begin{screen}
\begin{verbatim}
#CalcType = 0:Lanczos, 1:TPQCalc, 2:FullDiag
#CalcMod = 0:Hubbard, 1:Spin, 2:Kondo, 3:HubbardGC, 
4:SpinGC, 5:KondoGC 
CalcType   0
CalcModel   1
\end{verbatim}
\end{screen}
\end{minipage}
~\\
CalcTypeで計算手法の選択、CalcModelで対象モデルの選択を行います。ここでは、計算手法としてLanczos法、対象モデルとしてスピン系(カノニカル)を選択しています。{CalcModファイルでは固有ベクトルの入出力機能も指定することができます。}CalcModファイルの詳細は\ref{Subsec:calcmod}をご覧ください。\\

\item {\bf 計算用パラメータの指定}\\
ModParaでひも付けられるファイル(ここではmodpara.def)で計算用パラメータを指定します。ファイルの中身は下記の通りです。\\
\begin{minipage}{15cm}
\begin{screen}
\begin{verbatim}
--------------------
Model_Parameters   0
--------------------
VMC_Cal_Parameters
--------------------
CDataFileHead  zvo
CParaFileHead  zqp
--------------------
Nsite          16   
Ncond          16    
2Sz            0    
Lanczos_max    1000 
initial_iv     12   
nvec           1    
exct           1    
LanczosEps     14   
LanczosTarget  2    
LargeValue     12   
NumAve         5    
ExpecInterval  20 
\end{verbatim}
\end{screen}
\end{minipage}
~\\
このファイルでは、サイト数、{伝導電子の数、トータル$S_z$}やLanczosステップの最大数などを指定します。ModParaファイルの詳細は\ref{Subsec:modpara}をご覧ください。\\

\item {\bf 局在スピンの位置の指定}\\
LocSpinでひも付けられるファイル(ここではzlocspn.def)で局在スピンの位置と$S$の値を指定します。ファイルの中身は下記の通りです。\\
\begin{minipage}{15cm}
\begin{screen}
\begin{verbatim}
================================ 
NlocalSpin    16  
================================ 
========i_0LocSpn_1IteElc ====== 
================================ 
    0      1
    1      1
    2      1
    3      1
    4      1
    5      1
…
\end{verbatim}
\end{screen}
\end{minipage}
~\\
LocSpinファイルの詳細は\ref{Subsec:locspn}をご覧ください。
\end{description}

\subsection{Hamiltonianの指定}
基本パラメータを設定した後は、Hamiltonianを構築するためのファイルを作成します。$\HPhi$では、電子系の表現で計算を行うため、スピン系では以下の関係式
\begin{align}
S_z^{(i)}&=(c_{i\uparrow}^{\dag}c_{i\uparrow}-c_{i\downarrow}^{\dag}c_{i\downarrow})/2,\\
S_+^{(i)}&=c_{i\uparrow}^{\dag}c_{i\downarrow},\\
S_-^{(i)}&=c_{i\downarrow}^{\dag}c_{i\uparrow}
\end{align}
を用い、電子系の演算子に変換しHamiltonianの作成をする必要があります。
\begin{description}
\item {\bf Transfer部の指定}\\
Transでひも付けられるファイル(ここではzTrans.def)で電子系のTransferに相当するHamiltonian
\begin{align}
H +=-\sum_{ij\sigma_1\sigma2} t_{ij\sigma_1\sigma2}c_{i\sigma_1}^{\dag}c_{j\sigma_2}
\end{align}
を指定します。ファイルの中身は下記の通りです。\\
\begin{minipage}{15cm}
\begin{screen}
\begin{verbatim}
======================== 
NTransfer       0  
======================== 
========i_j_s_tijs====== 
======================== 
\end{verbatim}
\end{screen}
\end{minipage}
~\\
スピン系では外場を掛ける場合などに使用することができます。
例えば、サイト1に$-0.5 S_z^{(1)}${($S=1/2$)}の外場を掛けたい場合には、
電子系の表現$-0.5/2(c_{1\uparrow}^{\dag}c_{1\uparrow}-c_{1\downarrow}^{\dag}c_{1\downarrow})$
に書き換えた以下のファイルを作成することで計算することが出来ます。\\
\begin{minipage}{15cm}
\begin{screen}
\begin{verbatim}
======================== 
NTransfer      1   
======================== 
========i_j_s_tijs====== 
======================== 
1 0 1 0 -0.25 0
1 1 1 1 0.25 0
\end{verbatim}
\end{screen}
\end{minipage}
~\\
Transファイルの詳細は\ref{Subsec:Trans}をご覧ください。

\item {\bf 二体相互作用部の指定}\\
InterAllでひも付けられるファイル(ここではzInterAll.def)で電子系の二体相互作用部に相当するHamiltonian
\begin{equation}
H+=\sum_{i,j,k,l}\sum_{\sigma_1,\sigma_2, \sigma_3, \sigma_4}
I_{ijkl\sigma_1\sigma_2\sigma_3\sigma_4}c_{i\sigma_1}^{\dagger}c_{j\sigma_2}c_{k\sigma_3}^{\dagger}c_{l\sigma_4}
\end{equation}
を指定します。ファイルの中身は下記の通りです。\\
\begin{minipage}{16cm}
\begin{screen}
\begin{verbatim}
====================== 
NInterAll      96  
====================== 
========zInterAll===== 
====================== 
    0     0     0     0     1     0     1     0   0.500000  0.000000
    0     0     0     0     1     1     1     1  -0.500000  0.000000
    0     1     0     1     1     0     1     0  -0.500000  0.000000
    0     1     0     1     1     1     1     1   0.500000  0.000000
    0     0     0     1     1     1     1     0   1.000000  0.000000
    0     1     0     0     1     0     1     1   1.000000  0.000000
…
\end{verbatim}
\end{screen}
\end{minipage}
~\\
ここでは、簡単のためサイトiとサイトi+1間の相互作用に着目して説明します。{$S=1/2$の場合、}相互作用の項をフェルミオン演算子で書き換えると、
\begin{align}
H_{i,i+1}&=J(S_x^{(i)}S_x^{(i+1)}+S_y^{(i)}S_y^{(i+1)}+S_z^{(i)}S_z^{(i+1)}) \nonumber\\
&=J \left( \frac{1}{2}S_+^{(i)}S_-^{(i+1)}+\frac{1}{2}S_-^{(i)}S_+^{(i+1)}+S_z^{(i)}S_z^{(i+1)} \right) \nonumber\\
&=J \left[ \frac{1}{2}c_{i\uparrow}^{\dag}c_{i\downarrow}c_{i+1\downarrow}^{\dag}c_{i+1\uparrow}+\frac{1}{2}c_{i\downarrow}^{\dag}c_{i\uparrow}c_{i+1\uparrow}^{\dag}c_{i+1\downarrow}+\frac{1}{4}(c_{i\uparrow}^{\dag}c_{i\uparrow}-c_{i\downarrow}^{\dag}c_{i\downarrow})(c_{i+1\uparrow}^{\dag}c_{i+1\uparrow}-c_{i+1\downarrow}^{\dag}c_{i+1\downarrow}) \right] \nonumber 
\end{align}
となります。したがって、$J=2$に対してInterAllファイルのフォーマットを参考に相互作用を記載すると、$S_z^{(i)}S_z^{(i+1)}$の相互作用は\\
\begin{minipage}{16cm}
\begin{screen}
\begin{verbatim}
    i     0     i     0    i+1     0    i+1     0   0.500000  0.000000
    i     0     i     0    i+1     1    i+1     1  -0.500000  0.000000
    i     1     i     1    i+1     0    i+1     0  -0.500000  0.000000
    i     1     i     1    i+1     1    i+1     1   0.500000  0.000000
\end{verbatim}
\end{screen}
\end{minipage}
~\\
となり、それ以外の項は\\
\begin{minipage}{16cm}
\begin{screen}
\begin{verbatim}
    i     0     i     1    i+1     1    i+1     0   1.000000  0.000000
    i     1     i     0    i+1     0    i+1     1   1.000000  0.000000
\end{verbatim}
\end{screen}
\end{minipage}
~\\
と表せばよいことがわかります。なお、InterAll以外にも、Hamiltonianを簡易的に記載するための
下記のファイル形式に対応しています。
~\\{\bf CoulombIntra}: $n_ {i \uparrow}n_{i \downarrow}$で表される相互作用を指定します
($n_{i \sigma}=c_{i\sigma}^{\dag}c_{i\sigma}$)。
~\\{\bf CoulombInter}: $n_ {i}n_{j}$で表される相互作用を指定します($n_i=n_{i\uparrow}+n_{i\downarrow}$)。
~\\{\bf Hund}: $n_{i\uparrow}n_{j\uparrow}+n_{i\downarrow}n_{j\downarrow}$で表される相互作用を指定します。
~\\{\bf PairHop}:  $c_ {i \uparrow}^{\dag}c_{j\uparrow}c_{i \downarrow}^{\dag}c_{j  \downarrow}$
で表される相互作用を指定します。
~\\{\bf Exchange}: $S_i^+ S_j^-$で表される相互作用を指定します。
~\\{\bf Ising}: $S_i^z S_j^z$で表される相互作用を指定します。
~\\{\bf PairLift}: $c_ {i \uparrow}^{\dag}c_{i\downarrow}c_{j \uparrow}^{\dag}c_{j \downarrow}$
で表される相互作用を指定します

二体相互作用に関するファイル入力形式の詳細は\ref{Subsec:interall}-\ref{Subsec:pairlift}をご覧ください。

\end{description}

\subsection{出力ファイルの指定}
一体Green関数および二体Green関数の計算する成分を、それぞれOneBodyG, TwoBodyGでひも付けられるファイルで指定します。
\begin{description}
\item {\bf 一体Green関数の計算対象の指定}\\
OneBodyGでひも付けられるファイル(ここではgreenone.def)で計算する一体Green関数$\langle c_{i\sigma_1}^{\dag}c_{j\sigma_2} \rangle$の成分を指定します。ファイルの中身は下記の通りです。\\
\begin{minipage}{15cm}
\begin{screen}
\begin{verbatim}
===============================
NCisAjs         32
===============================
======== Green functions ======
===============================
    0     0     0     0
    0     1     0     1
    1     0     1     0
    1     1     1     1
    2     0     2     0
…
\end{verbatim}
\end{screen}
\end{minipage}
~\\
一体Green関数計算対象成分の指定に関するファイル入力形式の詳細は\ref{Subsec:onebodyg}をご覧ください。
\item {\bf 二体Green関数の計算対象の指定}\\
TwoBodyGでひも付けられるファイル(ここではgreentwo.def)で計算する二体Green関数$\langle c_{i\sigma_1}^{\dag}c_{j\sigma_2}c_{k\sigma_3}^{\dag}c_{l\sigma_4} \rangle$の成分を指定します。ファイルの中身は下記の通りです。\\
\begin{minipage}{15cm}
\begin{screen}
\begin{verbatim}
=============================================
NCisAjsCktAltDC       1024
=============================================
======== Green functions for Sq AND Nq ======
=============================================
    0     0     0     0     0     0     0     0
    0     0     0     0     0     1     0     1
    0     0     0     0     1     0     1     0
    0     0     0     0     1     1     1     1
    0     0     0     0     2     0     2     0
    …
\end{verbatim}
\end{screen}
\end{minipage}
~\\
二体Green関数計算対象成分の指定に関するファイル入力形式の詳細は\ref{Subsec:twobodyg}をご覧ください。
\end{description}

\subsection{計算の実行}
全ての入力ファイルが準備できた後、計算実行します。実行時はエキスパートモードを指定する"-e"をオプションとして指定の上、入力ファイルリストファイル(ここではnamelist.def)を引数とし、ターミナルから$\HPhi$を実行します。\\
\verb|$ |\underline{パス}\verb|/HPhi -e namelist.def|
~\\
計算開始後のプロセスは、スタンダードモードと同様になります。

\subsection{その他の系でのチュートリアル}

\verb|samples/Expert/|以下には次のチュートリアルが含まれています。

\begin{itemize}
\item 2次元正方格子Hubbardモデル

  (\verb|samples/Expert/Hubbard/square/|)
\item 2次元三角格子Hubbardモデル

  (\verb|samples/Expert/Hubbard/triangular/|)
\item 1次元近藤格子モデル

  (\verb|samples/Expert/Kondo/chain/|)
\item 1次元反強磁性的Heisenbergモデル

  (\verb|samples/Expert/Spin/HeisenbergChain/HeisenbergChain/|)
\item 2次元正方格子反強磁性的Heisenbergモデル

  (\verb|samples/Expert/Spin/HeisenbergSquare/|)
\item Kitaevモデル(Honeycomb格子、2$\times$3サイト)

  (\verb|samples/Expert/Spin/Kitaev/|)

\end{itemize}

これらのチュートリアルの実行方法は前節と同じです。
\newpage
\section{エキスパート用入力ファイル作成モード}
スタンダードモードで定義したモデルを対象に、エキスパート用入力ファイルを作成するモードです。使用方法は下記の通りです。
\begin{enumerate}
\item{\ref{Sec:StandardMode}に従い、スタンダードモデルで入力ファイルを作成します。}
\item{オプションとして"-sdry"を指定し、入力ファイル(ここではStdFace.def)を記入した上でHPhiを実行します。}\\
\verb|$ |\underline{パス}\verb|/HPhi -sdry StdFace.def|

このときMPIによる並列実行(\verb|mpirun|, \verb|mpiexec|など)は行わないでください。
~\\
\item{実行したディレクトリ内に、エキスパート用として下記のdefファイルが自動生成されます。}\\
\begin{minipage}{12cm}
\begin{screen}
\begin{verbatim}
calcmod.def   greentwo.def  namelist.def  zTrans.def
greenone.def  modpara.def   zInterAll.def zlocspn.def
\end{verbatim}
\end{screen}
\end{minipage}
\end{enumerate}