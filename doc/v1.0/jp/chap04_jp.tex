% !TEX root = userguide_jp.tex
%----------------------------------------------------------
\chapter{ファイル仕様}

%----------------------------------------------------------
\section{スタンダードモード用入力ファイル}
\label{Ch:HowToStandard}
T.B.D

\newpage
\section{エキスパートモード用入力ファイル}
\label{Ch:HowToExpert}

mVMCのエキスパートモードで使用する入力ファイル(*def)に関して説明します。
入力ファイルの種別は以下の4つで分類されます。
\begin{description}
\item[(1)~List:]
~\\{キーワード指定なし}:
使用するinput fileの名前のリストを書きます。なお、ファイル名は任意に指定することができます。
\item[(2)~Basic parameters:]
~\\{\bf ModPara}: 計算時に必要な基本的なパラメーター(サイトの数、電子数、Lanczosステップを何回やるかなど)を設定します。
~\\{\bf LocSpin}: 局在スピンの位置を設定します(近藤模型でのみ利用)。
\item[(3)~Set Hamiltonian:] 
~\\以下のファイルを用い、Hamiltonianを電子系の表式により指定します。
~\\{\bf Trans}: $c_{i\sigma_1}^{\dag}c_{j\sigma_2}$で表される一体項を指定します。
~\\{\bf InterAll}: $c_ {i \sigma_1}^{\dag}c_{j\sigma_2}c_{k \sigma_3}^{\dag}c_{l \sigma_4}$で表される一般二体相互作用を指定します。\\
~\\なお、使用頻度の高い相互作用に関しては下記のキーワードで指定することも可能です。
~\\{\bf CoulombIntra}: $n_ {i \uparrow}n_{i \downarrow}$で表される相互作用を指定します($n_{i \sigma}=c_{i\sigma}^{\dag}c_{i\sigma}$)。
~\\{\bf CoulombInter}: $n_ {i}n_{j}$で表される相互作用を指定します($n_i=n_{i\uparrow}+n_{i\downarrow}$)。
~\\{\bf Hund}: $n_{i\uparrow}n_{j\uparrow}+n_{i\downarrow}n_{j\downarrow}$で表される相互作用を指定します。
~\\{\bf PairHop}:  $c_ {i \uparrow}^{\dag}c_{j\uparrow}c_{i \downarrow}^{\dag}c_{j  \downarrow}$で表される相互作用を指定します。
~\\{\bf Exchange}: $S_i^+ S_j^-$で表される相互作用を指定します。
\item[(4)~Set condition of variational parameters :] 
~\\変分波動関数は
\begin{equation}
|\psi \rangle = {\cal P}_G{\cal P}_J{\cal P}_{d-h}^{(2)}{\cal P}_{d-h}^{(4)}{\cal L}^S{\cal L}^K{\cal L}^P |\phi_{\rm pair} \rangle,
\end{equation}
で与えられます。ここで、一体部分は実空間のペア関数
\begin{equation}
|\phi_{\rm pair} \rangle = \left[\sum_{i, j=1}^{N_s} f_{ij}c_{i\uparrow}^{\dag}c_{j\downarrow}^{\dag} \right]^{N/2}|0 \rangle,
\end{equation}
を用いた波動関数で表されます。ここで$N$は全電子数、$N_s$は全サイト数です。
変分パラメータの初期値は以下のファイルを用いて指定します。
~\\{\bf Gutzwiller}: ${\cal P}_G=\exp\left[ \sum_i g_i n_{i\uparrow} n_{i\downarrow} \right]$のうち、最適化の対象とする変分パラメータ$g_i$を指定します。
~\\{\bf Jastrow}: ${\cal P}_J=\exp\left[\frac{1}{2} \sum_{i\neq j} v_{ij} n_i n_j\right]$のうち、最適化の対象とする変分パラメータ$v_{ij}$を指定します。
~\\{\bf DH2}:  ${\cal P}_{d-h}^{(2)}= \exp \left[ \sum_t \sum_{n=0}^2 (\alpha_{2nt}^d \sum_{i}\xi_{i2nt}^d+\alpha_{2nt}^h \sum_{i}\xi_{i2nt}^h)\right]$で表される2サイトのdoublon-holon相関因子を指定します。詳細はDH2ファイルの説明を参照してください。
~\\{\bf DH4}:  ${\cal P}_{d-h}^{(4)}= \exp \left[ \sum_t \sum_{n=0}^4 (\alpha_{4nt}^d \sum_{i}\xi_{i4nt}^d+\alpha_{4nt}^h \sum_{i}\xi_{i4nt}^h)\right]$で表される4サイトのdoublon-holon相関因子を指定します。詳細はDH4ファイルの説明を参照してください。
~\\{\bf Orbital}: ペア軌道$|\phi_{\rm pair} \rangle = \left[\sum_{i, j=1}^{N_s} f_{ij}c_{i\uparrow}^{\dag}c_{j\downarrow}^{\dag} \right]^{N/2}|0 \rangle$を設定します。
~\\{\bf TransSym}: 運動量射影${\cal L}_K=\frac{1}{N_s}\sum_{{\bm R}}e^{i {\bm K} \cdot{\bm R} } \hat{T}_{\bm R}$と格子対称性射影${\cal L}_P=\sum_{\alpha}p_{\alpha} \hat{G}_{\alpha}$に関する指定を行います。ここで、${\bm K}$は全運動量、$\hat{T}_{\bm R}$は並進ベクトル${\bm R}$に対応する並進演算子、$\hat{G}_{\alpha}$は格子の点群演算子、$p_\alpha$はパリティをそれぞれ表します。

\item[(5)~Initial variational parameters:]
~\\ 変分パラメータに関する初期値を与えます。キーワード指定されない場合には$0$が初期値として設定されます。
~\\{\bf InGutzwiller}: ${\cal P}_G=\exp\left[ \sum_i g_i n_{i\uparrow} n_{i\downarrow} \right]$のうち、変分パラメータ$g_i$の初期値を指定します。
~\\{\bf InJastrow}: ${\cal P}_J=\exp\left[\frac{1}{2} \sum_{i\neq j} v_{ij} n_i n_j\right]$のうち、変分パラメータ$v_{ij}$の初期値を指定します。
~\\{\bf InDH2}:  ${\cal P}_{d-h}^{(2)}= \exp \left[ \sum_t \sum_{n=0}^2 (\alpha_{2nt}^d \sum_{i}\xi_{i2nt}^d+\alpha_{2nt}^h \sum_{i}\xi_{i2nt}^h)\right]$で表される2サイトのdoublon-holon相関因子$\alpha_{2nt}^{d(h)}$の初期値を指定します。
~\\{\bf InDH4}:  ${\cal P}_{d-h}^{(4)}= \exp \left[ \sum_t \sum_{n=0}^4 (\alpha_{4nt}^d \sum_{i}\xi_{i4nt}^d+\alpha_{4nt}^h \sum_{i}\xi_{i4nt}^h)\right]$で表される4サイトのdoublon-holon相関因子$\alpha_{4nt}^{d(h)}$の初期値を指定します。
~\\{\bf InOrbital}: ペア軌道$|\phi_{\rm pair} \rangle = \left[\sum_{i, j=1}^{N_s} f_{ij}c_{i\uparrow}^{\dag}c_{j\downarrow}^{\dag} \right]^{N/2}|0 \rangle$の$ f_{ij}$に関する初期値を設定します。


\item[(6)~Output:]
~\\{\bf OneBodyG }:出力する一体Green関数を指定します。
 $\langle c^{\dagger}_{i\sigma_1}c_{j\sigma_2}\rangle$が出力されます。

 {\bf TwoBodyG }:出力する二体Green関数を指定します。
 $\langle c^{\dagger}_{i\sigma_1}c_{j\sigma_2}c^{\dagger}_{k \sigma_3}c_{l\sigma_4}\rangle$
が出力されます。
\end{description}
%%%%%%%%%%%%%%%%%%%%%%
\newpage
~\subsection{入力ファイル指定用ファイル}
\label{Subsec:InputFileList}
計算で使用する入力ファイル一式を指定します。ファイル形式に関しては、以下のようなフォーマットをしています。\\
\begin{minipage}{10cm}
\begin{screen}
\begin{verbatim}
ModPara  modpara.def
LocSpin  zlocspn.def
Trans    ztransfer.def
InterAll zinterall.def
Orbital orbitalidx.def
OneBodyG zcisajs.def
TwoBodyG	zcisajscktaltdc.def
\end{verbatim}
\end{screen}
\end{minipage}
\\
\subsubsection{ファイル形式}
[string01]~[string02]
\subsubsection{パラメータ}
 \begin{itemize}
   \item  $[$string01$]$
   
   {\bf 形式 :} string型 (固定)
   
   {\bf 説明 :} キーワードを指定します。
   
   \item  $[$string02$]$
   
    {\bf 形式 :} string型 

   {\bf 説明 :} キーワードにひも付けられるファイル名を指定します(任意)。
 \end{itemize}
\subsubsection{使用ルール}
本ファイルを使用するにあたってのルールは以下の通りです。
\begin{itemize}
\item キーワードを記載後、半角空白を開けた後にファイル名を書きます。ファイル名は自由に設定できます。
\item ファイル読込用キーワードはTable\ref{Table:Defs}により指定します。
\item 必ず指定しなければいけないキーワードはModPara, LocSpin, Orbitalです。それ以外のキーワードについては、指定がない場合はデフォルト値が採用されます(変分パラメータについては最適化されず、固定する設定となります)。詳細は各ファイルの説明を参照してください。
\item 各キーワードは順不同に記述できます。
\item 指定したキーワード、ファイルが存在しない場合はエラー終了します。
\item $\#$で始まる行は読み飛ばされます。
\end{itemize}

 \begin{table*}[h!]
\begin{center}
  \begin{tabular}{|ll|} \hline
           Keywords     & Details for corresponding files       \\   \hline\hline
           ModPara       &  Parameters for calculation.        \\ \hline 
           %%%%%%%%%%%%%%%%%%  
           LocSpin         &  Configurations of the local spins for Hamiltonian.         \\ 
           Trans       &   Transfer and chemical potential for Hamiltonian.  \\
           InterAll  &   Two-body interactions for Hamiltonian. \\  
           CoulombIntra  &   CoulombIntra interactions. \\  
           CoulombInter  &   CoulombInter  interactions. \\  
           Hund  &   Hund couplings. \\  
           PairHop  &  Pair hopping couplings. \\  
           Exchange  &  Exchange couplings. \\  \hline
           %%%%%%%%%%%%%%%%%%
           Gutzwiller & Gutzwiller factors.\\
           Jastrow & Charge Jastrow factors.\\
           DH2 & 2-site doublon-holon correlation factors.\\
           DH4 & 4-site doublon-holon correlation factors.\\
           Orbital & pair orbital factors.\\
           TransSym & translational and lattice symmetry operation. \\ \hline
           %%%%%%%%%%%%%%%%%%
           InGutzwiller & Initial values of Gutzwiller factors.\\
           InJastrow & Initial values of charge Jastrow factors.\\
           InDH2 & Initial values of 2-site doublon-holon correlation factors.\\
           InDH4 & Initial values of 4-site doublon-holon correlation factors.\\
           InOrbital & Initial values of pair orbital factors.\\ \hline
           %%%%%%%%%%%%%%%%%%
           OneBodyG         &   Output components for Green functions $\langle c_{i\sigma}^{\dagger}c_{j\sigma}\rangle$           \\   
           TwoBodyG &   Output components for Correlation functions $\langle c_{i\sigma}^{\dagger}c_{j\sigma}c_{k\tau}^{\dagger}c_{l\tau}\rangle$  \\   \hline
  \end{tabular}
\end{center}
\caption{List of the definition files.}
\label{Table:Defs}
\end{table*}%

\newpage
%----------------------------------
\subsection{ModParaファイル}
\label{Subsec:modpara}
計算で使用するパラメータを指定します。以下のようなフォーマットをしています。\\
\begin{minipage}{10cm}
\begin{screen}
\begin{verbatim}
--------------------
Model_Parameters   0
--------------------
VMC_Cal_Parameters
--------------------
CDataFileHead  zvo
CParaFileHead  zqp
--------------------
NVMCCalMode    0
NLanczosMode   0
--------------------
NDataIdxStart  1
NDataQtySmp    5
--------------------
Nsite          16
Nelectron      8
NSPGaussLeg    1
NSPStot        0
NMPTrans       1
NSROptItrStep  1200
NSROptItrSmp   100
NSROptFixSmp   1
DSROptRedCut   0.001
DSROptStaDel   0.02
DSROptStepDt   0.02
NVMCWarmUp     10
NVMCIniterval  1
NVMCSample     1000
NExUpdatePath  0
RndSeed        11272
NSplitSize     1
NStore         1  
\end{verbatim}
\end{screen}
\end{minipage}

\subsubsection{ファイル形式}
以下のように行数に応じ異なる形式をとります。
 \begin{itemize}
   \item  1 - 5行:  ヘッダ(何が書かれても問題ありません)。
   \item  6行:  [string01]~[string02]
   \item  7 - 8行: ヘッダ(何が書かれても問題ありません)
   \item  9行以降: [string01]~[int01]
  \end{itemize}
各項目の対応関係は以下の通りです。
\begin{itemize}
   \item  $[$string01$]$
   
   {\bf 形式 :} string型 (固定)

  {\bf 説明 :} キーワードの指定を行います。
   
   \item  $[$string02$]$
   
   {\bf 形式 :} string型 (空白不可)

  {\bf 説明 :} アウトプットファイルのヘッダを指定します。

   \item  $[$int01$]$
   
   {\bf 形式 :} int型 (空白不可)

  {\bf 説明 :} キーワードでひも付けられるパラメータを指定します。

  \end{itemize}

\subsubsection{使用ルール}
本ファイルを使用するにあたってのルールは以下の通りです。
\begin{itemize}
\item 9行目以降ではキーワードを記載後、半角空白を開けた後に整数値を書きます。
\item 行数固定で読み込みを行う為、パラメータの省略はできません。
\end{itemize}

~\subsubsection{キーワード}
 \begin{itemize}
  \item  \verb|CDataFileHead|

 {\bf 形式 :} string型 (空白不可、必須)

{\bf 説明 :} アウトプットファイルのヘッダ。例えば、一体のGreen関数の出力ファイル名が{\bf xxx\_cisajs.dat}として出力されます(xxxに\verb|CDataFileHead|で指定した文字が記載)。

 \item  \verb|CParaFileHead|

 {\bf 形式 :} string型 (空白不可、必須)

{\bf 説明 :} 最適化された変分パラメータの出力ファイル名のヘッダ。最適化された変分パラメータが{\bf xxx\_opt.dat}ファイルとして出力されます(xxxに\verb|CParaFileHead|で指定した文字が記載)。
 
 
 \item  \verb|NVMCCalMode|

 {\bf 形式 :} int型 (\tr{デフォルト値 = 0})

{\bf 説明 :} [0] 変分パラメータの最適化、[1] 1 体・2 体のグリーン関数の計算。
 
 \item  \verb|NLanczosMode|

 {\bf 形式 :} int型 (\tr{デフォルト値 = 0})

{\bf 説明 :} [0] 何もしない、[1] Single Lanczos Step でエネルギーまで計算、[2] Single Lanczos Step で1 体・2 体のグリーン関数まで計算(条件: 1, 2 は\verb|NVMCCalMode| = 1のみ使用可能. また, pair hopping項, exchange 項がハミルトニアンに含まれる場合は使用できません)。
 
 \item  \verb|NDataIdxStart|

 {\bf 形式 :} int型 (\tr{デフォルト値 = 0})

{\bf 説明 :} 出力ファイルの付加番号。\verb|NVMCCalMode|= 0 の場合は\verb|NDataIdxStart|が出力され、 \verb|NVMCCalMode| = 1 の場合は、\verb|NDataIdxStart|から連番で\verb|NDataQtySmp|個のファイルを出力します。
   
 \item  \verb|NDataQtySmp|

 {\bf 形式 :} int型 (\tr{デフォルト値 = 1})

{\bf 説明 :} 出力ファイルのセット数。 \verb|NVMCCalMode| = 1 の場合に使用します。

 \item  \verb|Nsite|

{\bf 形式 :} int型 (1以上、必須)

{\bf 説明 :} サイト数を指定する整数。  

\item  \verb|Nelectron|

{\bf 形式 :} {int型 (1以上、必須)}

{\bf 説明 :} \tr{$\uparrow$電子の個数。$S_z=0$部分空間で計算するので、$\uparrow$電子と$\downarrow$電子の個数は等しい。}

 \item  \verb|NSPGaussLeg|

{\bf 形式 :} {int型 (1以上、\tr{デフォルト値 = 1})}

{\bf 説明 :} スピン量子数射影の$\beta$積分($S^y$回転)のGauss-Legendre求積法の分点数。

 \item  \verb|NSPStot|

{\bf 形式 :} int型 (0以上、\tr{デフォルト値 = 0})

{\bf 説明 :}  スピン量子数。

 \item  \verb|NMPTrans|

{\bf 形式 :} int型 (1以上、\tr{デフォルト値 = 0})

{\bf 説明 :} 
運動量・格子対称性の量子数射影の個数。TransSymファイルで指定した重みで上から\verb|NMPTrans|個まで使用する。\tr{射影を行わない場合は1に設定する。}

 \item  \verb|NSROptItrStep|

{\bf 形式 :} int型 (1以上、\tr{デフォルト値 = 1000})

{\bf 説明 :} 
SR 法で最適化する場合の全ステップ数。\verb|NVMCCalMode|=0の場合のみ使用されます。
 
 \item  \verb|NSROptItrSmp|

{\bf 形式 :} int型 (1以上数、\tr{デフォルト値 = }\verb|NSROptItrStep|/10)

{\bf 説明 :} \verb|NSROptItrStep|ステップ中、最後の\verb|NSROptItrSmp|ステップでの各変分パラメータの平均値を最適値とする。\verb|NVMCCalMode|=0の場合のみ使用されます。

 \item  \verb|NSROptFixSmp|

{\bf 形式 :} int型 (1以上、\tr{デフォルト値 = 1})

{\bf 説明 :} 固定サンプルの下で何回のSR 法のステップを行うか指定。

\item   \verb|DSROptRedCut|
   
{\bf 形式 :} double型 (\tr{デフォルト値 = 0.001})

{\bf 説明 :} SR 法安定化因子。手法論文の$\varepsilon_{\rm wf}$に対応。

 \item  \verb|DSROptStaDel| 
   
 {\bf 形式 :} double型 (\tr{デフォルト値 = 0.02})

  {\bf 説明 :} SR 法安定化因子。手法論文の$\varepsilon$に対応。
     
\item \verb|DSROptStepDt|

{\bf 形式 :} double型 

{\bf 説明 :} \tr{要確認(マニュアルに説明なし)}
 
\item \verb|NVMCWarmUp|

{\bf 形式 :} int型 (1以上、\tr{デフォルト値=10})

{\bf 説明 :}マルコフ連鎖の空回し回数。

\item \verb|NVMCIniterval|

{\bf 形式 :} int型 (1以上、\tr{デフォルト値=1})

{\bf 説明 :} サンプル間のステップ間隔。ローカル更新が\verb|Nsite|× \verb|NVMCIniterval| 回行わます。

\item \verb|NVMCSample|

{\bf 形式 :} int型 (1以上、\tr{デフォルト値=1000})

{\bf 説明 :} 期待値計算に使用するサンプル数。

\item \verb|NExUpdatePath|

{\bf 形式 :} int型 (1以上)

{\bf 説明 :} ローカル更新で2 電子交換を[0] 認めない、[1] 認める。\tr{デフォルト設定は局在スピンが一つでもある場合は1、それ以外は0となります}。

\item \verb|RndSeed|

{\bf 形式 :} int型 

{\bf 説明 :} 乱数の初期seed。\tr{MPI 並列では各計算機に}\verb|RndSeed|\tr{+my rank+1 で初期seed が与えられます。}

 \item \verb|NSplitSize|

{\bf 形式 :} int型 (1以上、\tr{デフォルト値=1})

{\bf 説明 :} mpi内部並列を行う場合の並列数。

\item \verb|NStoreOO|

{\bf 形式 :} int型 (1以上、\tr{デフォルト値=1})

{\bf 説明 :} 期待値$\langle O_k O_l \rangle$を計算するとき行列-行列積にするオプション(1で機能On)。
 
 \end{itemize}


\newpage
%----------------------------------
\subsection{LocSpin指定ファイル}
\label{Subsec:locspn}
局在スピンを指定します。以下のようなフォーマットをしています。\\
\begin{minipage}{10cm}
\begin{screen}
\begin{verbatim}
================================ 
NlocalSpin     6  
================================ 
========i_0LocSpn_1IteElc ====== 
================================ 
    0      1
    1      0
    2      1
    3      0
    4      1
    5      0
    6      1
    7      0
    8      1
    9      0
   10      1
   11      0
\end{verbatim}
\end{screen}
\end{minipage}


\subsubsection{ファイル形式}
以下のように行数に応じ異なる形式をとります。
 \begin{itemize}
   \item  1行:  ヘッダ(何が書かれても問題ありません)。
   \item  2行:   [string01]~[int01]
   \item  3-5行:  ヘッダ(何が書かれても問題ありません)。
   \item  6行以降:  [int02]~[int03]
  \end{itemize}
 \subsubsection{パラメータ}
 \begin{itemize}

 \item  $[$string01$]$

 {\bf 形式 :} string型 (空白不可)

{\bf 説明 :} 局在スピンの総数を示すキーワード(任意)。


  \item  $[$int01$]$

 {\bf 形式 :} int型 (空白不可)

{\bf 説明 :} 局在スピンの総数を指定する整数。

 
  \item  $[$int02$]$

 {\bf 形式 :} int型 (空白不可)

{\bf 説明 :} サイト番号を指定する整数。0以上\verb|Nsite|{未満}で指定します。

 
  \item  $[$int03$]$

 {\bf 形式 :} int型 (空白不可)

{\bf 説明 :} 局在スピンか遍歴電子かを指定する整数。\\
{
0: 遍歴電子\\
$n>0$: $2S=n$の局在スピン\\
}
を選択することが出来ます。
 \end{itemize}

\subsubsection{使用ルール}
本ファイルを使用するにあたってのルールは以下の通りです。
\begin{itemize}
\item 行数固定で読み込みを行う為、ヘッダの省略はできません。
\item $[$int01$]$と$[$int03$]$で指定される局在電子数の総数が異なる場合はエラー終了します。
\item $[$int02$]$の総数が全サイト数と異なる場合はエラー終了します。
\item $[$int02$]$が全サイト数以上もしくは負の値をとる場合はエラー終了します。
\end{itemize}


\newpage
\subsection{Trans指定ファイル}
\label{Subsec:Trans}
ここではハミルトニアン
\begin{align}
H +=-\sum_{ij\sigma_1\sigma2} t_{ij\sigma_1\sigma2}c_{i\sigma_1}^{\dag}c_{j\sigma_2}
\end{align}
に対するTransfer積分$t_{ij\sigma_1\sigma2}$を指定します。以下にファイル名を記載します。\\
\begin{minipage}{12.5cm}
\begin{screen}
\begin{verbatim}
======================== 
NTransfer      24  
======================== 
========i_j_s_tijs====== 
======================== 
    0     0     2     0   1.000000  0.000000
    2     0     0     0   1.000000  0.000000
    0     1     2     1   1.000000  0.000000
    2     1     0     1   1.000000  0.000000
    2     0     4     0   1.000000  0.000000
    4     0     2     0   1.000000  0.000000
    2     1     4     1   1.000000  0.000000
    4     1     2     1   1.000000  0.000000
    4     0     6     0   1.000000  0.000000
    6     0     4     0   1.000000  0.000000
    4     1     6     1   1.000000  0.000000
    6     1     4     1   1.000000  0.000000
    6     0     8     0   1.000000  0.000000
    8     0     6     0   1.000000  0.000000
…
\end{verbatim}
\end{screen}
\end{minipage}

\subsubsection{ファイル形式}
以下のように行数に応じ異なる形式をとります。
 \begin{itemize}
   \item  1行:  ヘッダ(何が書かれても問題ありません)。
   \item  2行:   [string01]~[int01]
   \item  3-5行:  ヘッダ(何が書かれても問題ありません)。
   \item  6行以降: [int02]~~[int03]~~[int04]~~[int05]~~[double01]~~[double02] 
  \end{itemize}
\subsubsection{パラメータ}
 \begin{itemize}

   \item  $[$string01$]$
   
    {\bf 形式 :} string型 (空白不可)

   {\bf 説明 :} Transfer総数のキーワード名を指定します(任意)。

   \item  $[$int01$]$
   
    {\bf 形式 :} int型 (空白不可)

   {\bf 説明 :} Transferの総数を指定します。

  \item  $[$int02$]$, $[$int04$]$

 {\bf 形式 :} int型 (空白不可)

{\bf 説明 :} サイト番号を指定する整数。0以上\verb|Nsite|{未満}で指定します。
 
  \item  $[$int03$]$, $[$int05$]$

 {\bf 形式 :} int型 (空白不可)

{\bf 説明 :} スピンを指定する整数。\\
0: アップスピン\\
1: ダウンスピン\\
を選択することが出来ます。


 \item  $[$double01$]$
   
   {\bf 形式 :} double型 (空白不可)

  {\bf 説明 :}  $t_{ij\sigma_1\sigma_2}$の実部を指定します。

 \item  $[$double02$]$
   
   {\bf 形式 :} double型 (空白不可)

  {\bf 説明 :}  $t_{ij\sigma_1\sigma_2}$の虚部を指定します。
\end{itemize}

\subsubsection{使用ルール}
本ファイルを使用するにあたってのルールは以下の通りです。
\begin{itemize}
\item 行数固定で読み込みを行う為、ヘッダの省略はできません。
\item Hamiltonianがエルミートという制限から$t_{ij\sigma_1\sigma_2}=t_{ji\sigma_2\sigma_1}^{\dagger}$の関係を満たす必要があります。上記の関係が成立しない場合にはエラー終了します。
\item 成分が重複して指定された場合にはエラー終了します。
\item $[$int01$]$と定義されているTrasferの総数が異なる場合はエラー終了します。
\item $[$int02$]$-$[$int05$]$を指定する際、範囲外の整数を指定した場合はエラー終了します。
\end{itemize}

\newpage
\subsection{InterAll指定ファイル}
\label{Subsec:interall}
ここでは一般二体相互作用をハミルトニアンに付け加えます。付け加える項は以下で与えられます。
\begin{equation}
H+=\sum_{i,j,k,l}\sum_{\sigma_1,\sigma_2, \sigma_3, \sigma_4}
I_{ijkl\sigma_1\sigma_2\sigma_3\sigma_4}c_{i\sigma_1}^{\dagger}c_{j\sigma_2}c_{k\sigma_3}^{\dagger}c_{l\sigma_4}
\end{equation}
なお、スピンに関して計算する場合には、$i=j, k=l$となるよう設定してください。
以下にファイル例を記載します。

\begin{minipage}{12.5cm}
\begin{screen}
\begin{verbatim}
====================== 
NInterAll      36  
====================== 
========zInterAll===== 
====================== 
0    0    0    1    1    1    1    0   0.50  0.0
0    1    0    0    1    0    1    1   0.50  0.0
0    0    0    0    1    0    1    0   0.25  0.0
0    0    0    0    1    1    1    1  -0.25  0.0
0    1    0    1    1    0    1    0  -0.25  0.0
0    1    0    1    1    1    1    1   0.25  0.0
2    0    2    1    3    1    3    0   0.50  0.0
2    1    2    0    3    0    3    1   0.50  0.0
2    0    2    0    3    0    3    0   0.25  0.0
2    0    2    0    3    1    3    1  -0.25  0.0
2    1    2    1    3    0    3    0  -0.25  0.0
2    1    2    1    3    1    3    1   0.25  0.0
4    0    4    1    5    1    5    0   0.50  0.0
4    1    4    0    5    0    5    1   0.50  0.0
4    0    4    0    5    0    5    0   0.25  0.0
4    0    4    0    5    1    5    1  -0.25  0.0
4    1    4    1    5    0    5    0  -0.25  0.0
4    1    4    1    5    1    5    1   0.25  0.0
…
\end{verbatim}
\end{screen}
\end{minipage}

\subsubsection{ファイル形式}
以下のように行数に応じ異なる形式をとります。
 \begin{itemize}
   \item  1行:  ヘッダ(何が書かれても問題ありません)。
   \item  2行:   [string01]~[int01]
   \item  3-5行:  ヘッダ(何が書かれても問題ありません)。
   \item  6行以降:
   [int02]~[int03]~[int04]~[int05]~[int06]~[int07]~[int08]~[int09]~[double01]~[double02] 
  \end{itemize}
\subsubsection{パラメータ}
 \begin{itemize}

   \item  $[$string01$]$
   
    {\bf 形式 :} string型 (空白不可)

   {\bf 説明 :} 二体相互作用の総数のキーワード名を指定します(任意)。

   \item  $[$int01$]$
   
    {\bf 形式 :} int型 (空白不可)

   {\bf 説明 :} 二体相互作用の総数を指定します。

  \item  $[$int02$]$, $[$int04$]$, $[$int06$]$, $[$int08$]$

 {\bf 形式 :} int型 (空白不可)

{\bf 説明 :} サイト番号を指定する整数。0以上\verb|Nsite|{未満}で指定します。
 
  \item  $[$int03$]$, $[$int05$]$, $[$int07$]$, $[$int09$]$

 {\bf 形式 :} int型 (空白不可)

{\bf 説明 :} スピンを指定する整数。\\
0: アップスピン\\
1: ダウンスピン\\
を選択することが出来ます。


 \item  $[$double01$]$
   
   {\bf 形式 :} double型 (空白不可)

  {\bf 説明 :}  $I_{ijkl\sigma_1\sigma_2\sigma_3\sigma_4}$の実部を指定します。

 \item  $[$double02$]$
   
   {\bf 形式 :} double型 (空白不可)

  {\bf 説明 :}  $I_{ijkl\sigma_1\sigma_2\sigma_3\sigma_4}$の虚部を指定します。
\end{itemize}


\subsubsection{使用ルール}
本ファイルを使用するにあたってのルールは以下の通りです。
\begin{itemize}
\item 行数固定で読み込みを行う為、ヘッダの省略はできません。
\item Hamiltonianがエルミートという制限から$I_{ijkl\sigma_1\sigma_2\sigma_3\sigma_4}=I_{lkji\sigma_4\sigma_3\sigma_2\sigma_1}^{\dag}$の関係を満たす必要があります。上記の関係が成立しない場合にはエラー終了します。
また、エルミート共役の形式は$I_{ijkl\sigma_1\sigma_2\sigma_3\sigma_4}c_{i\sigma_1}^{\dagger}c_{j\sigma_2}c_{k\sigma_3}^{\dagger}c_{l\sigma_4}$に対して、$I_{lkji\sigma_4\sigma_3\sigma_2\sigma_1}$
$c_{l\sigma_4}^{\dagger}c_{k\sigma_3}c_{j\sigma_2}^{\dagger}c_{i\sigma_1}$を満たすように入力してください。
\item スピンに関して計算する場合、$i=j, k=l$を満たさないペアが存在するとエラー終了します。
\item 成分が重複して指定された場合にはエラー終了します。
\item $[$int01$]$と定義されているInterAllの総数が異なる場合はエラー終了します。
\item $[$int02$]$-$[$int09$]$を指定する際、範囲外の整数を指定した場合はエラー終了します。
\end{itemize}


\newpage
\subsection{CoulombIntra指定ファイル}
\label{Subsec:coulombintra}
オンサイトクーロン相互作用をハミルトニアンに付け加えます。付け加える項は以下で与えられます。
\begin{equation}
H+=\sum_{i}U_i n_ {i \uparrow}n_{i \downarrow}
\end{equation}
以下にファイル例を記載します。

\begin{minipage}{12.5cm}
\begin{screen}
\begin{verbatim}
====================== 
NCoulombIntra 6  
====================== 
========i_0LocSpn_1IteElc ====== 
====================== 
   0  4.000000
   1  4.000000
   2  4.000000
   3  4.000000
   4  4.000000
   5  4.000000
\end{verbatim}
\end{screen}
\end{minipage}

\subsubsection{ファイル形式}
以下のように行数に応じ異なる形式をとります。
 \begin{itemize}
   \item  1行:  ヘッダ(何が書かれても問題ありません)。
   \item  2行:   [string01]~[int01]
   \item  3-5行:  ヘッダ(何が書かれても問題ありません)。
   \item  6行以降:
   [int02]~[double01] 
  \end{itemize}
\subsubsection{パラメータ}
 \begin{itemize}

   \item  $[$string01$]$
   
    {\bf 形式 :} string型 (空白不可)

   {\bf 説明 :} オンサイトクーロン相互作用の総数のキーワード名を指定します(任意)。

   \item  $[$int01$]$
   
    {\bf 形式 :} int型 (空白不可)

   {\bf 説明 :} オンサイトクーロン相互作用の総数を指定します。

  \item  $[$int02$]$
  
 {\bf 形式 :} int型 (空白不可)

{\bf 説明 :} サイト番号を指定する整数。0以上\verb|Nsite|{未満}で指定します。
 
 \item  $[$double01$]$
   
   {\bf 形式 :} double型 (空白不可)

  {\bf 説明 :}  $U_i$を指定します。
  
\end{itemize}

\subsubsection{使用ルール}
本ファイルを使用するにあたってのルールは以下の通りです。
\begin{itemize}
\item 行数固定で読み込みを行う為、ヘッダの省略はできません。
\item 成分が重複して指定された場合にはエラー終了します。
\item $[$int01$]$と定義されているオンサイトクーロン相互作用の総数が異なる場合はエラー終了します。
\item $[$int02$]$を指定する際、範囲外の整数を指定した場合はエラー終了します。
\end{itemize}


\newpage
\subsection{CoulombInter指定ファイル}
オフサイトクーロン相互作用をハミルトニアンに付け加えます。付け加える項は以下で与えられます。
\begin{equation}
H+=\sum_{i,j}V_{ij} n_ {i}n_{j}
\end{equation}
以下にファイル例を記載します。

\begin{minipage}{12.5cm}
\begin{screen}
\begin{verbatim}
====================== 
NCoulombInter 6  
====================== 
========CoulombInter ====== 
====================== 
   0     1 -0.125000
   1     2 -0.125000
   2     3 -0.125000
   3     4 -0.125000
   4     5 -0.125000
   5     0 -0.125000
\end{verbatim}
\end{screen}
\end{minipage}

\subsubsection{ファイル形式}
以下のように行数に応じ異なる形式をとります。
 \begin{itemize}
   \item  1行:  ヘッダ(何が書かれても問題ありません)。
   \item  2行:   [string01]~[int01]
   \item  3-5行:  ヘッダ(何が書かれても問題ありません)。
   \item  6行以降:
   [int02]~[int03]~[double01] 
  \end{itemize}
\subsubsection{パラメータ}
 \begin{itemize}

   \item  $[$string01$]$
   
    {\bf 形式 :} string型 (空白不可)

   {\bf 説明 :} オフサイトクーロン相互作用の総数のキーワード名を指定します(任意)。

   \item  $[$int01$]$
   
    {\bf 形式 :} int型 (空白不可)

   {\bf 説明 :} オフサイトクーロン相互作用の総数を指定します。

  \item  $[$int02$]$, $[$int03$]$
  
 {\bf 形式 :} int型 (空白不可)

{\bf 説明 :} サイト番号を指定する整数。0以上\verb|Nsite|{未満}で指定します。
 
 \item  $[$double01$]$
   
   {\bf 形式 :} double型 (空白不可)

  {\bf 説明 :}  $V_{ij}$を指定します。
  
\end{itemize}

\subsubsection{使用ルール}
本ファイルを使用するにあたってのルールは以下の通りです。
\begin{itemize}
\item 行数固定で読み込みを行う為、ヘッダの省略はできません。
\item 成分が重複して指定された場合にはエラー終了します。
\item $[$int01$]$と定義されているオフサイトクーロン相互作用の総数が異なる場合はエラー終了します。
\item $[$int02$]$-$[$int03$]$を指定する際、範囲外の整数を指定した場合はエラー終了します。
\end{itemize}

\newpage
\subsection{Hund指定ファイル}
Hundカップリングをハミルトニアンに付け加えます。付け加える項は以下で与えられます。
\begin{equation}
H+=-\sum_{i,j}J_{ij}^{\rm Hund} (n_{i\uparrow}n_{j\uparrow}+n_{i\downarrow}n_{j\downarrow})
\end{equation}
以下にファイル例を記載します。

\begin{minipage}{12.5cm}
\begin{screen}
\begin{verbatim}
====================== 
NHund 6  
====================== 
========Hund ====== 
====================== 
   0     1 -0.250000
   1     2 -0.250000
   2     3 -0.250000
   3     4 -0.250000
   4     5 -0.250000
   5     0 -0.250000
\end{verbatim}
\end{screen}
\end{minipage}

\subsubsection{ファイル形式}
以下のように行数に応じ異なる形式をとります。
 \begin{itemize}
   \item  1行:  ヘッダ(何が書かれても問題ありません)。
   \item  2行:   [string01]~[int01]
   \item  3-5行:  ヘッダ(何が書かれても問題ありません)。
   \item  6行以降:
   [int02]~[int03]~[double01] 
  \end{itemize}
\subsubsection{パラメータ}
 \begin{itemize}

   \item  $[$string01$]$
   
    {\bf 形式 :} string型 (空白不可)

   {\bf 説明 :} Hundカップリングの総数のキーワード名を指定します(任意)。

   \item  $[$int01$]$
   
    {\bf 形式 :} int型 (空白不可)

   {\bf 説明 :} Hundカップリングの総数を指定します。

  \item  $[$int02$]$, $[$int03$]$
  
 {\bf 形式 :} int型 (空白不可)

{\bf 説明 :} サイト番号を指定する整数。0以上\verb|Nsite|{未満}で指定します。
 
 \item  $[$double01$]$
   
   {\bf 形式 :} double型 (空白不可)

  {\bf 説明 :}  $J_{ij}^{\rm Hund}$を指定します。
  
\end{itemize}

\subsubsection{使用ルール}
本ファイルを使用するにあたってのルールは以下の通りです。
\begin{itemize}
\item 行数固定で読み込みを行う為、ヘッダの省略はできません。
\item 成分が重複して指定された場合にはエラー終了します。
\item $[$int01$]$と定義されているHundカップリングの総数が異なる場合はエラー終了します。
\item $[$int02$]$-$[$int03$]$を指定する際、範囲外の整数を指定した場合はエラー終了します。
\end{itemize}

\newpage
\subsection{PairHop指定ファイル}
PairHopカップリングをハミルトニアンに付け加えます。付け加える項は以下で与えられます。
\begin{equation}
H+=\sum_{i,j}J_{ij}^{\rm Pair} c_ {i \uparrow}^{\dag}c_{j\uparrow}c_{i \downarrow}^{\dag}c_{j  \downarrow}
\end{equation}
以下にファイル例を記載します。

\begin{minipage}{12.5cm}
\begin{screen}
\begin{verbatim}
====================== 
NPairhop 12  
====================== 
========Pairhop ====== 
====================== 
   0     1  0.50000
   1     0  0.50000  
   1     2  0.50000
   2     1  0.50000
   2     3  0.50000
   3     2  0.50000
   3     4  0.50000
   4     3  0.50000
   4     5  0.50000
   5     4  0.50000
   5     0  0.50000
   0     5  0.50000
\end{verbatim}
\end{screen}
\end{minipage}

\subsubsection{ファイル形式}
以下のように行数に応じ異なる形式をとります。
 \begin{itemize}
   \item  1行:  ヘッダ(何が書かれても問題ありません)。
   \item  2行:   [string01]~[int01]
   \item  3-5行:  ヘッダ(何が書かれても問題ありません)。
   \item  6行以降:
   [int02]~[int03]~[double01] 
  \end{itemize}
\subsubsection{パラメータ}
 \begin{itemize}

   \item  $[$string01$]$
   
    {\bf 形式 :} string型 (空白不可)

   {\bf 説明 :} PairHopカップリングの総数のキーワード名を指定します(任意)。

   \item  $[$int01$]$
   
    {\bf 形式 :} int型 (空白不可)

   {\bf 説明 :} PairHopカップリングの総数を指定します。

  \item  $[$int02$]$, $[$int03$]$
  
 {\bf 形式 :} int型 (空白不可)

{\bf 説明 :} サイト番号を指定する整数。0以上\verb|Nsite|{未満}で指定します。
 
 \item  $[$double01$]$
   
   {\bf 形式 :} double型 (空白不可)

  {\bf 説明 :}  $J_{ij}^{\rm Pair}$を指定します。
  
\end{itemize}

\subsubsection{使用ルール}
本ファイルを使用するにあたってのルールは以下の通りです。
\begin{itemize}
\item 行数固定で読み込みを行う為、ヘッダの省略はできません。
\item 成分が重複して指定された場合にはエラー終了します。
\item $[$int01$]$と定義されているPairHopカップリングの総数が異なる場合はエラー終了します。
\item $[$int02$]$-$[$int03$]$を指定する際、範囲外の整数を指定した場合はエラー終了します。
\end{itemize}

\newpage
\subsection{Exchange指定ファイル}
Exchangeカップリングをハミルトニアンに付け加えます。
電子系の場合には
\begin{equation}
H+=\sum_{i,j}J_{ij}^{\rm Ex} (c_ {i \uparrow}^{\dag}c_{j\uparrow}c_{j \downarrow}^{\dag}c_{i  \downarrow}+c_ {i \downarrow}^{\dag}c_{j\downarrow}c_{j \uparrow}^{\dag}c_{i  \uparrow})
\end{equation}
が付け加えられ、スピン系の場合には
\begin{equation}
H+=\sum_{i,j}J_{ij}^{\rm Ex} (S_i^+S_j^-+S_i^-S_j^+)
\end{equation}
が付け加えられます。
{\bf スピン系の$(S_i^+S_j^-+S_i^-S_j^+)$を
電子系の演算子で書き直すと、
$-(c_ {i \uparrow}^{\dag}c_{j\uparrow}c_{j \downarrow}^{\dag}c_{i  \downarrow}+c_ {i \downarrow}^{\dag}c_{j\downarrow}c_{j \uparrow}^{\dag}c_{i  \uparrow})$
となることに注意して下さい。}
以下にファイル例を記載します。

\begin{minipage}{12.5cm}
\begin{screen}
\begin{verbatim}
====================== 
NExchange 6  
====================== 
========Exchange ====== 
====================== 
   0     1  0.50000
   1     2  0.50000
   2     3  0.50000
   3     4  0.50000
   4     5  0.50000
   5     0  0.50000
\end{verbatim}
\end{screen}
\end{minipage}

\subsubsection{ファイル形式}
以下のように行数に応じ異なる形式をとります。
 \begin{itemize}
   \item  1行:  ヘッダ(何が書かれても問題ありません)。
   \item  2行:   [string01]~[int01]
   \item  3-5行:  ヘッダ(何が書かれても問題ありません)。
   \item  6行以降:
   [int02]~[int03]~[double01] 
  \end{itemize}
\subsubsection{パラメータ}
 \begin{itemize}

   \item  $[$string01$]$
   
    {\bf 形式 :} string型 (空白不可)

   {\bf 説明 :} Exchangeカップリングの総数のキーワード名を指定します(任意)。

   \item  $[$int01$]$
   
    {\bf 形式 :} int型 (空白不可)

   {\bf 説明 :} Exchangeカップリングの総数を指定します。

  \item  $[$int02$]$, $[$int03$]$
  
 {\bf 形式 :} int型 (空白不可)

{\bf 説明 :} サイト番号を指定する整数。0以上\verb|Nsite|{未満}で指定します。
 
 \item  $[$double01$]$
   
   {\bf 形式 :} double型 (空白不可)

  {\bf 説明 :}  $J_{ij}^{\rm Ex}$を指定します。
  
\end{itemize}

\subsubsection{使用ルール}
本ファイルを使用するにあたってのルールは以下の通りです。
\begin{itemize}
\item 行数固定で読み込みを行う為、ヘッダの省略はできません。
\item 成分が重複して指定された場合にはエラー終了します。
\item $[$int01$]$と定義されているExchangeカップリングの総数が異なる場合はエラー終了します。
\item $[$int02$]$-$[$int03$]$を指定する際、範囲外の整数を指定した場合はエラー終了します。
\end{itemize}

\newpage
\subsection{Gutzwiller指定ファイル}
\label{Subsec:Gutzwiller}

Gutzwiller因子
\begin{equation}
{\cal P}_G=\exp\left[ \sum_i g_i n_{i\uparrow} n_{i\downarrow} \right]
\end{equation}
の設定を行います。指定するパラメータはサイト番号$i$と$g_i$の変分パラメータの番号です。
以下にファイル例を記載します。

\begin{minipage}{12.5cm}
\begin{screen}
\begin{verbatim}
======================
NGutzwillerIdx 2  
====================== 
===== Gutzwiller ===== 
====================== 
   1
   0     0
   1     0
   2     0
   3     1
(continue...)
  12     1
  13     0
  14     0
  15     0
   0     1
   1     0
\end{verbatim}
\end{screen}
\end{minipage}

\subsubsection{ファイル形式}
以下のように行数に応じ異なる形式をとります($N_s$はサイト数、$N_g$は変分パラメータの種類の数)。
 \begin{itemize}
   \item  1行:  ヘッダ(何が書かれても問題ありません)。
   \item  2行:   [string01]~[int01]
   \item  3-5行:  ヘッダ(何が書かれても問題ありません)。
   \item  6行: [int02]
   \item  7 - 6+$N_s$行: [int03]~[int04]
   \item  7+$N_s$ - 6+$N_s$+$N_g$行:[int05]~[int06]
  \end{itemize}
\subsubsection{パラメータ}
 \begin{itemize}

   \item  $[$string01$]$
   
    {\bf 形式 :} string型 (空白不可)

   {\bf 説明 :} $g_i$の変分パラメータの種類の総数のキーワード名を指定します(任意)。

   \item  $[$int01$]$
   
    {\bf 形式 :} int型 (空白不可)

   {\bf 説明 :} $g_i$の変分パラメータの種類の総数を指定します。

  \item  $[$int02$]$
  
 {\bf 形式 :} int型 (空白不可)

{\bf 説明 :} 変分パラメータの型を指定する整数。0が実数、1が複素数に対応します。

  \item  $[$int03$]$
  
 {\bf 形式 :} int型 (空白不可)

{\bf 説明 :} サイト番号を指定する整数。0以上\verb|Nsite|{未満}で指定します。
 
 \item  $[$int04$]$
   
   {\bf 形式 :} int型 (空白不可)

  {\bf 説明 :} $g_i$の変分パラメータの種類を表します。0以上[int01]{未満}で指定します。

 \item  $[$int05$]$
   
   {\bf 形式 :} int型 (空白不可)

  {\bf 説明 :} $g_i$の変分パラメータの種類を表します(最適化有無の設定用)。0以上[int01]{未満}で指定します。

 \item  $[$int06$]$
   
   {\bf 形式 :} int型 (空白不可)

  {\bf 説明 :} [int05]で指定した$g_i$の変分パラメータの最適化有無を設定します。最適化する場合は1、最適化しない場合は0とします。
  
\end{itemize}

\subsubsection{使用ルール}
本ファイルを使用するにあたってのルールは以下の通りです。
\begin{itemize}
\item 行数固定で読み込みを行う為、ヘッダの省略はできません。
\item 成分が重複して指定された場合にはエラー終了します。
\item $[$int01$]$と定義されている変分パラメータの種類の総数が異なる場合はエラー終了します。
\item $[$int02$]$-$[$int06$]$を指定する際、範囲外の整数を指定した場合はエラー終了します。
\end{itemize}

\newpage
\subsection{Jastrow指定ファイル}
\label{Subsec:Jastrow}
Jastrow因子
\begin{equation}
{\cal P}_J=\exp\left[\frac{1}{2} \sum_{i\neq j} v_{ij} n_i n_j\right]
\end{equation}
の設定を行います。指定するパラメータはサイト番号$i, j$と$v_{ij}$の変分パラメータの番号です。以下にファイル例を記載します。

\begin{minipage}{12.5cm}
\begin{screen}
\begin{verbatim}
======================
NJastrowIdx 5  
====================== 
== i_j_JastrowIdx  ===
======================
   0	 
   0     1     0 
   0     2     1 
   0     3     0 
 (continue...)
   0    1 
   1    1 
   2    1 
   3    1 
   4    1 
\end{verbatim}
\end{screen}
\end{minipage}

\subsubsection{ファイル形式}
以下のように行数に応じ異なる形式をとります($N_s$はサイト数、$N_j$は変分パラメータの種類の数)。
 \begin{itemize}
   \item  1行:  ヘッダ(何が書かれても問題ありません)。
   \item  2行:   [string01]~[int01]
   \item  3-5行:  ヘッダ(何が書かれても問題ありません)。
   \item  6行:[int02]
   \item  7 - 6+$N_s\times (N_s-1)$行: [int03]~[int04]~[int05]
   \item  7+$N_s\times (N_s-1)$ - 6+$N_s\times (N_s-1)$+$N_j$行:[int06]~[int07]
  \end{itemize}
\subsubsection{パラメータ}
 \begin{itemize}

   \item  $[$string01$]$
   
    {\bf 形式 :} string型 (空白不可)

   {\bf 説明 :} $v_{ij}$の変分パラメータの種類の総数のキーワード名を指定します(任意)。

   \item  $[$int01$]$
   
    {\bf 形式 :} int型 (空白不可)

   {\bf 説明 :} $v_{ij}$の変分パラメータの種類の総数を指定します。

   \item  $[$int02$]$
   
    {\bf 形式 :} int型 (空白不可)

   {\bf 説明 :} $v_{ij}$の変分パラメータの型を指定します。0が実数、1が複素数に対応します。

  \item  $[$int03$]$, $[$int04$]$
  
 {\bf 形式 :} int型 (空白不可)

{\bf 説明 :} サイト番号を指定する整数。0以上\verb|Nsite|{未満}で指定します。
 
 \item  $[$int05$]$
   
   {\bf 形式 :} int型 (空白不可)

  {\bf 説明 :} $v_{ij}$の変分パラメータの種類を表します。0以上[int01]{未満}で指定します。

 \item  $[$int06$]$
   
   {\bf 形式 :} int型 (空白不可)

  {\bf 説明 :} $v_{ij}$の変分パラメータの種類を表します(最適化有無の設定用)。0以上[int01]{未満}で指定します。

 \item  $[$int07$]$
   
   {\bf 形式 :} int型 (空白不可)

  {\bf 説明 :} [int06]で指定した$v_{ij}$の変分パラメータの最適化有無を設定します。最適化する場合は1、最適化しない場合は0とします。
  
\end{itemize}

\subsubsection{使用ルール}
本ファイルを使用するにあたってのルールは以下の通りです。
\begin{itemize}
\item 行数固定で読み込みを行う為、ヘッダの省略はできません。
\item 成分が重複して指定された場合にはエラー終了します。
\item $[$int01$]$と定義されている変分パラメータの種類の総数が異なる場合はエラー終了します。
\item $[$int02$]$-$[$int07$]$を指定する際、範囲外の整数を指定した場合はエラー終了します。
\end{itemize}

\newpage
\subsection{DH2指定ファイル}
\label{Subsec:DH2}

\begin{equation}
{\cal P}_{d-h}^{(2)}= \exp \left[ \sum_t \sum_{n=0}^2 (\alpha_{2nt}^d \sum_{i}\xi_{i2nt}^d+\alpha_{2nt}^h \sum_{i}\xi_{i2nt}^h)\right]
\end{equation}
で表される2サイトのdoublon-holon相関因子の設定を行います。指定するパラメータはサイト番号$i$とその周囲2サイト、$\alpha_{2nt}^{d,h}$の変分パラメータの番号で、変分パラメータは各サイト毎に$t$種類設定します。以下にファイル例を記載します。

\begin{minipage}{12.5cm}
\begin{screen}
\begin{verbatim}
====================================
NDoublonHolon2siteIdx 2  
====================================
==  i_xi_xi_DoublonHolon2siteIdx  ==
====================================
   0
   0     5   15    0
   0    13    7    1
   1     6   12    0
   1    14    4    1
 (continue...)
  15     0   10    0
  15     8    2    1
   0     1 
   1     1 
   2     1 
(continue...)
  10     1 
  11     1 
\end{verbatim}
\end{screen}
\end{minipage}

\subsubsection{ファイル形式}
以下のように行数に応じ異なる形式をとります($N_s$はサイト数、$N_{\rm dh2}$は変分パラメータの種類の数)。
 \begin{itemize}
   \item  1行:  ヘッダ(何が書かれても問題ありません)。
   \item  2行:   [string01]~[int01]
   \item  3-5行:  ヘッダ(何が書かれても問題ありません)。
   \item  6行:[int02]
   \item  7 - 6+$N_s\times N_{\rm dh2}$行: [int03]~[int04]~[int05]~[int06]
   \item  7+$N_s\times N_{\rm dh2}$ - 6+$(N_s+6) \times N_{\rm dh2}$行:[int07]~[int08]
  \end{itemize}
\subsubsection{パラメータ}
 \begin{itemize}

   \item  $[$string01$]$
   
    {\bf 形式 :} string型 (空白不可)

   {\bf 説明 :} 変分パラメータのセット総数のキーワード名を指定します(任意)。

   \item  $[$int01$]$
   
    {\bf 形式 :} int型 (空白不可)

   {\bf 説明 :} 変分パラメータのセット総数を指定します。

   \item  $[$int02$]$
   
    {\bf 形式 :} int型 (空白不可)

   {\bf 説明 :} 変分パラメータの型を指定します。0が実数、1が複素数に対応します。

  \item  $[$int03$]$,  $[$int04$]$, $[$int05$]$
   
 {\bf 形式 :} int型 (空白不可)

{\bf 説明 :} サイト番号を指定する整数。0以上\verb|Nsite|{未満}で指定します。
 
 \item  $[$int06$]$
   
   {\bf 形式 :} int型 (空白不可)

  {\bf 説明 :} 変分パラメータの種類を表します。0以上[int01]{未満}で指定します。

 \item  $[$int07$]$
   
   {\bf 形式 :} int型 (空白不可)

  {\bf 説明 :} 変分パラメータの種類を表します(最適化有無の設定用)。値は
  \begin{itemize}
  \item{$n$}: 周囲のdoublon(holon)数 (0, 1, 2)  \\
  \item{$s$}: 中心がdoublonの場合$0$, 中心がholonの場合$1$ \\
  \item{$t$}: 変分パラメータのセット番号(0, $\cdots$ [int1]-1)
  \end{itemize}  
  として、$(2n+s)\times$[int01]$+t$を設定します。
  
 \item  $[$int08$]$
   
   {\bf 形式 :} int型 (空白不可)

  {\bf 説明 :} [int07]で指定した変分パラメータの最適化有無を設定します。最適化する場合は1、最適化しない場合は0とします。
  
  
\end{itemize}

\subsubsection{使用ルール}
本ファイルを使用するにあたってのルールは以下の通りです。
\begin{itemize}
\item 行数固定で読み込みを行う為、ヘッダの省略はできません。
\item 成分が重複して指定された場合にはエラー終了します。
\item $[$int01$]$と定義されている変分パラメータの種類の総数が異なる場合はエラー終了します。
\item \tr{$[$int02$]$-$[$int08$]$を指定する際、範囲外の整数を指定した場合はエラー終了します。}
\end{itemize}


\newpage
\subsection{DH4指定ファイル}
\label{Subsec:DH4}

\begin{equation}
{\cal P}_{d-h}^{(4)}= \exp \left[ \sum_t \sum_{n=0}^4 (\alpha_{4nt}^d \sum_{i}\xi_{i4nt}^d+\alpha_{4nt}^h \sum_{i}\xi_{i4nt}^h)\right]
\end{equation}
で表される4サイトのdoublon-holon相関因子の設定を行います。指定するパラメータはサイト番号$i$とその周囲4サイト、$\alpha_{4nt}^{d,h}$の変分パラメータの番号で、変分パラメータは各サイト毎に$t$種類設定します。以下にファイル例を記載します。

\begin{minipage}{12.5cm}
\begin{screen}
\begin{verbatim}
====================================
NDoublonHolon4siteIdx 1  
====================================
==  i_xi_xi_DoublonHolon4siteIdx  ==
====================================
   0
   0     1    3    4   12    0
   1     2    0    5   13    0
   2     3    1    6   14    0
   3     0    2    7   15    0
 (continue...)
  14    15   13    2   10    0
  15    12   14    3   11    0
   0     1 
   1     1 
(continue...)
   8     1 
   9     1 
\end{verbatim}
\end{screen}
\end{minipage}

\subsubsection{ファイル形式}
以下のように行数に応じ異なる形式をとります($N_s$はサイト数、$N_{\rm dh4}$は変分パラメータの種類の数)。
 \begin{itemize}
   \item  1行:  ヘッダ(何が書かれても問題ありません)。
   \item  2行:   [string01]~[int01]
   \item  3-5行:  ヘッダ(何が書かれても問題ありません)。
   \item  6行: [int02]
   \item  7 - 6+$N_s\times N_{\rm dh4}$行: [int03]~[int04]~[int05]~[int06]~[int07]~[int08]
   \item  7+$N_s\times N_{\rm dh4}$ - 6+$(N_s+10) \times N_{\rm dh4}$行:[int09]~[int10]
  \end{itemize}
\subsubsection{パラメータ}
 \begin{itemize}

   \item  $[$string01$]$
   
    {\bf 形式 :} string型 (空白不可)

   {\bf 説明 :} 変分パラメータのセット総数のキーワード名を指定します(任意)。

   \item  $[$int01$]$
   
    {\bf 形式 :} int型 (空白不可)

   {\bf 説明 :} 変分パラメータのセット総数を指定します。

   \item  $[$int02$]$
   
    {\bf 形式 :} int型 (空白不可)

   {\bf 説明 :} 変分パラメータの型を指定します。0が実数、1が複素数に対応します。

  \item   $[$int03$]$,  $[$int04$]$, $[$int05$]$, $[$int06$]$, $[$int07$]$
   
 {\bf 形式 :} int型 (空白不可)

{\bf 説明 :} サイト番号を指定する整数。0以上\verb|Nsite|{未満}で指定します。
 
 \item  $[$int08$]$
   
   {\bf 形式 :} int型 (空白不可)

  {\bf 説明 :} 変分パラメータの種類を表します。0以上[int01]{未満}で指定します。

 \item  $[$int09$]$
   
   {\bf 形式 :} int型 (空白不可)

  {\bf 説明 :} 変分パラメータの種類を表します(最適化有無の設定用)。値は
  \begin{itemize}
  \item{$n$}: 周囲のdoublon(holon)数 (0, 1, 2, 3, 4)  \\
  \item{$s$}: 中心がdoublonの場合$0$, 中心がholonの場合$1$ \\
  \item{$t$}: 変分パラメータのセット番号(0, $\cdots$ [int1]-1)
  \end{itemize}  
  として、$(2n+s)\times$[int01]$+t$を設定します。
  
 \item  $[$int10$]$
   
   {\bf 形式 :} int型 (空白不可)

  {\bf 説明 :} [int09]で指定した変分パラメータの最適化有無を設定します。最適化する場合は1、最適化しない場合は0とします。
  
\end{itemize}

\subsubsection{使用ルール}
本ファイルを使用するにあたってのルールは以下の通りです。
\begin{itemize}
\item 行数固定で読み込みを行う為、ヘッダの省略はできません。
\item 成分が重複して指定された場合にはエラー終了します。
\item $[$int01$]$と定義されている変分パラメータの種類の総数が異なる場合はエラー終了します。
\item \tr{$[$int02$]$-$[$int10$]$を指定する際、範囲外の整数を指定した場合はエラー終了します。}
\end{itemize}

\newpage
\subsection{Orbital指定ファイル}
\label{Subsec:Orbital}

\begin{equation}
|\phi_{\rm pair} \rangle = \left[\sum_{i, j=1}^{N_s} f_{ij}c_{i\uparrow}^{\dag}c_{j\downarrow}^{\dag} \right]^{N/2}|0 \rangle
\end{equation}
で表されるペア軌道の設定を行います。指定するパラメータはサイト番号$i, j$と変分パラメータの種類を設定します。以下にファイル例を記載します。

\begin{minipage}{12.5cm}
\begin{screen}
\begin{verbatim}
====================================
NOrbitalIdx 64  
====================================
==  i_j_OrbitalIdx  ==
====================================
   1
   0     0     0 
   0     1     1 
   0     2     2 
   0     3     3 
 (continue...)
  15     9    62 
  15    10    63 
   0    1 
   1    1 
(continue...)
  62    1 
  63    1 
\end{verbatim}
\end{screen}
\end{minipage}

\subsubsection{ファイル形式}
以下のように行数に応じ異なる形式をとります($N_s$はサイト数、$N_{\rm o}$は変分パラメータの種類の数)。
 \begin{itemize}
   \item  1行:  ヘッダ(何が書かれても問題ありません)。
   \item  2行:   [string01]~[int01]
   \item  3-5行:  ヘッダ(何が書かれても問題ありません)。
   \item  6行: [int02]
   \item  7 - 6+$N_s^2$行: [int03]~[int04]~[int05]
   \item  7+$N_s^2$ - 6+$N_s^2+N_{\rm o}$行:[int06]~[int07]
  \end{itemize}
\subsubsection{パラメータ}
 \begin{itemize}

   \item  $[$string01$]$
   
    {\bf 形式 :} string型 (空白不可)

   {\bf 説明 :} 変分パラメータのセット総数のキーワード名を指定します(任意)。

   \item  $[$int01$]$
   
    {\bf 形式 :} int型 (空白不可)

   {\bf 説明 :} 変分パラメータのセット総数を指定します。

   \item  $[$int02$]$
   
    {\bf 形式 :} int型 (空白不可)

   {\bf 説明 :} 変分パラメータの型を指定します。0が実数、1が複素数に対応します。

  \item  $[$int03$]$, $[$int04$]$
   
 {\bf 形式 :} int型 (空白不可)

{\bf 説明 :} サイト番号を指定する整数。0以上\verb|Nsite|{未満}で指定します。
 
 \item  $[$int05$]$
   
   {\bf 形式 :} int型 (空白不可)

  {\bf 説明 :} 変分パラメータの種類を表します。0以上[int01]{未満}で指定します。

 \item  $[$int06$]$
   
   {\bf 形式 :} int型 (空白不可)

  {\bf 説明 :} 変分パラメータの種類を表します(最適化有無の設定用)。0以上[int01]{未満}で指定します。
  
 \item  $[$int07$]$
   
   {\bf 形式 :} int型 (空白不可)

  {\bf 説明 :} [int06]で指定した変分パラメータの最適化有無を設定します。最適化する場合は1、最適化しない場合は0とします。
  
  
\end{itemize}

\subsubsection{使用ルール}
本ファイルを使用するにあたってのルールは以下の通りです。
\begin{itemize}
\item 行数固定で読み込みを行う為、ヘッダの省略はできません。
\item 成分が重複して指定された場合にはエラー終了します。
\item $[$int01$]$と定義されている変分パラメータの種類の総数が異なる場合はエラー終了します。
\item \tr{$[$int02$]$-$[$int07$]$を指定する際、範囲外の整数を指定した場合はエラー終了します。}
\end{itemize}



\newpage
\subsection{TransSym指定ファイル}
\label{Subsec:TransSym}

運動量射影${\cal L}_K=\frac{1}{N_s}\sum_{{\bm R}}e^{i {\bm K} \cdot{\bm R} } \hat{T}_{\bm R}$と格子対称性射影${\cal L}_P=\sum_{\alpha}p_{\alpha} \hat{G}_{\alpha}$について、重みとサイト番号に関する指定を行います。射影するパターンは$(\alpha, {\bm R})$で指定されます。\tr{射影を行わない場合も重み1.0 で“恒等演算”を指定してください。}
以下にファイル例を記載します。

\begin{minipage}{12.5cm}
\begin{screen}
\begin{verbatim}
====================================
NQPTrans 4  
====================================
== TrIdx_TrWeight_and_TrIdx_i_xi  ==
====================================
   0  1.000000
   1  1.000000
   2  1.000000
   3  1.000000
   0     0    0
 (continue...)
   3    12    1
   3    13    2 
\end{verbatim}
\end{screen}
\end{minipage}

\subsubsection{ファイル形式}
以下のように行数に応じ異なる形式をとります($N_s$はサイト数、$N_{\rm TS}$は射影演算子の種類の総数)。
 \begin{itemize}
   \item  1行:  ヘッダ(何が書かれても問題ありません)。
   \item  2行:   [string01]~[int01]
   \item  3-5行:  ヘッダ(何が書かれても問題ありません)。
   \item  6 - 5+$N_{\rm TS}$行: [int02]~[double01]
   \item  6+$N_{\rm TS}$ - 5+$(N_s+1) \times N_{\rm TS}$行:[int03]~[int04]~[int05]
  \end{itemize}
\subsubsection{パラメータ}
 \begin{itemize}

   \item  $[$string01$]$
   
    {\bf 形式 :} string型 (空白不可)

   {\bf 説明 :} 射影パターンの総数に関するキーワード名を指定します(任意)。

   \item  $[$int01$]$
   
    {\bf 形式 :} int型 (空白不可)

   {\bf 説明 :} 射影パターンの総数を指定します。

  \item  $[$int02$]$
   
 {\bf 形式 :} int型 (空白不可)

{\bf 説明 :} 射影パターン$(\alpha, {\bm R})$を指定する整数。0以上 $[$int01$]${未満}で指定します。
 
  \item  $[$double01$]$
   
 {\bf 形式 :} double型 (空白不可)

{\bf 説明 :} 射影パターン$(\alpha, {\bm R})$の重み$p_{\alpha}\cos ({\bm K}\cdot \alpha)$を指定します。
 
 \item  $[$int03$]$
   
   {\bf 形式 :} int型 (空白不可)

  {\bf 説明 :} 射影パターン$(\alpha, {\bm R})$を指定する整数。0以上 $[$int01$]${未満}で指定します。

 \item  $[$int04$]$, $[$int05$]$
   
   {\bf 形式 :} int型 (空白不可)

  {\bf 説明 :} サイト番号を指定する整数。0以上\verb|Nsite|{未満}で指定します。$[$int03$]$で指定した並進・点群移動をサイト番号$[$int04$]$に作用させた場合の行き先が、サイト番号$[$int05$]$となるように設定します。

\end{itemize}

\subsubsection{使用ルール}
本ファイルを使用するにあたってのルールは以下の通りです。
\begin{itemize}
\item 行数固定で読み込みを行う為、ヘッダの省略はできません。
\item 成分が重複して指定された場合にはエラー終了します。
\item $[$int01$]$と定義されている射影パターンの総数が異なる場合はエラー終了します。
\item \tr{$[$int02$]$-$[$int05$]$を指定する際、範囲外の整数を指定した場合はエラー終了します。}
\end{itemize}

\newpage
\subsection{InGutzwiller指定ファイル}
\label{Subsec:InGutzwiller}
\begin{equation}
{\cal P}_G=\exp\left[ \sum_i g_i n_{i\uparrow} n_{i\downarrow} \right]
\end{equation}
の$g_i$について初期値を設定します。

以下にファイル例を記載します。

\begin{minipage}{12.5cm}
\begin{screen}
\begin{verbatim}
======================
NGutzwillerIdx  16  
====================== 
===== Gutzwiller ===== 
====================== 
   0     0.0     0.0
   1     0.1     0.0
   2     0.0     0.0
   3     0.1     0.0
 (continue...)
  15     0.1     0.0
\end{verbatim}
\end{screen}
\end{minipage}

\subsubsection{ファイル形式}
以下のように行数に応じ異なる形式をとります($N_s$はサイト数)。
 \begin{itemize}
   \item  1行:  ヘッダ(何が書かれても問題ありません)。
   \item  2行:   [string01]~[int01]
   \item  3-5行:  ヘッダ(何が書かれても問題ありません)。
   \item  6 - 5+$N_s$行: [int02]~[double01]~[double02]
  \end{itemize}
\subsubsection{パラメータ}
 \begin{itemize}

   \item  $[$string01$]$
   
    {\bf 形式 :} string型 (空白不可)

   {\bf 説明 :} $g_i$の変分パラメータ総数のキーワード名を指定します(任意)。

   \item  $[$int01$]$
   
    {\bf 形式 :} int型 (空白不可)

   {\bf 説明 :} $g_i$の変分パラメータ総数を指定します。

  \item  $[$int02$]$
  
 {\bf 形式 :} int型 (空白不可)

{\bf 説明 :} サイト番号を指定する整数。0以上\verb|Nsite|{未満}で指定します。
 
 \item  $[$double01$]$, $[$double02$]$
   
   {\bf 形式 :} double型 (空白不可)

  {\bf 説明 :} $g_i$の初期値を与えます。$[$double01$]$が実部、$[$double02$]$が虚部を与えます。\verb|Gutzwiller|指定ファイルで型を実部に指定している場合は$[$double02$]$の値は破棄されます。
  
\end{itemize}

\subsubsection{使用ルール}
本ファイルを使用するにあたってのルールは以下の通りです。
\begin{itemize}
\item 行数固定で読み込みを行う為、ヘッダの省略はできません。
\item 成分が重複して指定された場合にはエラー終了します。
\item $[$int01$]$と定義されている変分パラメータの総数が異なる場合はエラー終了します。
\item \verb|Gutzwiller|指定ファイルで紐付けされるサイト番号とパラメータの種類と、入力ファイルで指定されるパラメータの値の整合性がとれない場合は警告を出します。その際、入力値としては平均された値が採用されます。例えば、\verb|Gutzwiller|指定ファイルで$i$, $j$サイトのパラメータの種類が共通の$0$に指定されているにも関わらず、入力ファイルで$i$,$j$サイトの値が異なる場合には警告が出されます。
\end{itemize}


\newpage
\subsection{InJastrow指定ファイル}
\label{Subsec:InJastrow}
Jastrow因子
\begin{equation}
{\cal P}_J=\exp\left[\frac{1}{2} \sum_{i\neq j} v_{ij} n_i n_j\right]
\end{equation}
の$v_{ij}$について初期値の設定を行います。

\begin{minipage}{12.5cm}
\begin{screen}
\begin{verbatim}
======================
NJastrowIdx 240 
====================== 
== i_j_JastrowIdx  ===
====================== 
   0     1     0.0     0.0 
   0     2     1.0     0.0
   0     3     0.0     0.0
 (continue...)
  16     14    0.0   0.0
  16     15    0.0   0.0
\end{verbatim}
\end{screen}
\end{minipage}

\subsubsection{ファイル形式}
以下のように行数に応じ異なる形式をとります($N_s$はサイト数)。
 \begin{itemize}
   \item  1行:  ヘッダ(何が書かれても問題ありません)。
   \item  2行:   [string01]~[int01]
   \item  3-5行:  ヘッダ(何が書かれても問題ありません)。
   \item  6 - 5+$N_s*(N_s-1)$行: [int02]~[int03]~[double01]~[double02]
  \end{itemize}
\subsubsection{パラメータ}
 \begin{itemize}

   \item  $[$string01$]$
   
    {\bf 形式 :} string型 (空白不可)

   {\bf 説明 :} $v_{ij}$の変分パラメータ総数のキーワード名を指定します(任意)。

   \item  $[$int01$]$
   
    {\bf 形式 :} int型 (空白不可)

   {\bf 説明 :} $v_{ij}$の変分パラメータ総数を指定します。

  \item  $[$int02$]$, $[$int03$]$
  
 {\bf 形式 :} int型 (空白不可)

{\bf 説明 :} サイト番号を指定する整数。0以上\verb|Nsite|{未満}で指定します。
 
 \item  $[$double01$]$, $[$double02$]$
   
   {\bf 形式 :} double型 (空白不可)

  {\bf 説明 :} $v_{ij}$の初期値を与えます。$[$double01$]$が実部、$[$double02$]$が虚部を与えます。\verb|Jastrow|指定ファイルで型を実部に指定している場合は$[$double02$]$の値は破棄されます。
  
\end{itemize}

\subsubsection{使用ルール}
本ファイルを使用するにあたってのルールは以下の通りです。
\begin{itemize}
\item 行数固定で読み込みを行う為、ヘッダの省略はできません。
\item 成分が重複して指定された場合にはエラー終了します。
\item $[$int01$]$と定義されている変分パラメータの総数が異なる場合はエラー終了します。
\item \verb|Jastrow|指定ファイルで紐付けされるサイト番号とパラメータの種類と、入力ファイルで指定されるパラメータの値の整合性がとれない場合は警告を出します。その際、入力値としては平均された値が採用されます。
\end{itemize}

\newpage
\subsection{InDH2指定ファイル}
\label{Subsec:InDH2}
\begin{equation}
{\cal P}_{d-h}^{(2)}= \exp \left[ \sum_t \sum_{n=0}^2 (\alpha_{2nt}^d \sum_{i}\xi_{i2nt}^d+\alpha_{2nt}^h \sum_{i}\xi_{i2nt}^h)\right]
\end{equation}
で表される2サイトのdoublon-holon相関因子の初期値設定を行います。指定するパラメータは$(n, t, s)$(s=0は中心がdoublon、1は中心がholon)に対応する番号と、$\alpha_{2nt}^{d,h}$の初期値です。以下にファイル例を記載します。

\begin{minipage}{12.5cm}
\begin{screen}
\begin{verbatim}
====================================
NDoublonHolon2siteIdx 32 
====================================
==  i_xi_xi_DoublonHolon2siteIdx  ==
====================================
   0    0.0    0.0
   1    0.0    0.0
   2    1.0    0.0
 (continue...)
  10   0.0    0.0
  11   1.0    0.0
\end{verbatim}
\end{screen}
\end{minipage}

\subsubsection{ファイル形式}
以下のように行数に応じ異なる形式をとります($N_s$はサイト数、$N_{\rm dh2}$は変分パラメータの種類の数)。
 \begin{itemize}
   \item  1行:  ヘッダ(何が書かれても問題ありません)。
   \item  2行:   [string01]~[int01]
   \item  3-5行:  ヘッダ(何が書かれても問題ありません)。
   \item  6 - 5+$N_{\rm dh2}$行: [int02]~[double01]~[double02]
  \end{itemize}
\subsubsection{パラメータ}
 \begin{itemize}

   \item  $[$string01$]$
   
    {\bf 形式 :} string型 (空白不可)

   {\bf 説明 :} 変分パラメータのセット総数のキーワード名を指定します(任意)。

   \item  $[$int01$]$
   
    {\bf 形式 :} int型 (空白不可)

   {\bf 説明 :} 変分パラメータのセット総数を指定します。

 \item  $[$int02$]$
   
   {\bf 形式 :} int型 (空白不可)

  {\bf 説明 :} \verb|DH2|指定ファイル内の変分プラメータの種類([int07])を指定します。値は0以上[int01]{未満}です。

 \item  $[$double01$]$, $[$double02$]$
   
   {\bf 形式 :} double型 (空白不可)

  {\bf 説明 :} $[$double01$]$が実部、$[$double02$]$が虚部を与えます。\verb|DH2|指定ファイルで型を実部に指定している場合は$[$double02$]$の値は破棄されます。
  
\end{itemize}

\subsubsection{使用ルール}
本ファイルを使用するにあたってのルールは以下の通りです。
\begin{itemize}
\item 行数固定で読み込みを行う為、ヘッダの省略はできません。
\item 成分が重複して指定された場合にはエラー終了します。
\item $[$int01$]$と定義されている変分パラメータの種類の総数が異なる場合はエラー終了します。
\end{itemize}


\newpage
\subsection{InDH4指定ファイル}
\label{Subsec:InDH4}
\begin{equation}
{\cal P}_{d-h}^{(4)}= \exp \left[ \sum_t \sum_{n=0}^4 (\alpha_{4nt}^d \sum_{i}\xi_{i4nt}^d+\alpha_{4nt}^h \sum_{i}\xi_{i4nt}^h)\right]
\end{equation}
で表される4サイトのdoublon-holon相関因子の設定を行います。。指定するパラメータは$(n, t, s)$(s=0は中心がdoublon、1は中心がholon)に対応する番号と、$\alpha_{4nt}^{d,h}$の変分パラメータの初期値です。以下にファイル例を記載します。

\begin{minipage}{12.5cm}
\begin{screen}
\begin{verbatim}
====================================
NDoublonHolon4siteIdx 10
====================================
==  i_xi_xi_DoublonHolon4siteIdx  ==
====================================
   0     1.0     0.0
   1     0.0     0.0 
(continue...)
   8     1.0     0.0 
   9     0.0     0.0 
\end{verbatim}
\end{screen}
\end{minipage}

\subsubsection{ファイル形式}
以下のように行数に応じ異なる形式をとります($N_{\rm dh4}$は変分パラメータの総数)。
 \begin{itemize}
   \item  1行:  ヘッダ(何が書かれても問題ありません)。
   \item  2行:   [string01]~[int01]
   \item  3-5行:  ヘッダ(何が書かれても問題ありません)。
   \item  6 - 6+$N_{\rm dh4}$行:[int02]~[double01]~[double02]
  \end{itemize}
\subsubsection{パラメータ}
 \begin{itemize}

   \item  $[$string01$]$
   
    {\bf 形式 :} string型 (空白不可)

   {\bf 説明 :} 変分パラメータのセット総数のキーワード名を指定します(任意)。

   \item  $[$int01$]$
   
    {\bf 形式 :} int型 (空白不可)

   {\bf 説明 :} 変分パラメータのセット総数を指定します。

 \item  $[$int02$]$
   
   {\bf 形式 :} int型 (空白不可)

  {\bf 説明 :} \verb|DH4|指定ファイル内の変分プラメータの種類([int09])を指定します。値は0以上[int01]{未満}です。

 \item  $[$double01$]$, $[$double02$]$
   
   {\bf 形式 :} double型 (空白不可)

  {\bf 説明 :} $[$double01$]$が実部、$[$double02$]$が虚部を与えます。\verb|DH4|指定ファイルで型を実部に指定している場合は$[$double02$]$の値は破棄されます。
  
\end{itemize}

\subsubsection{使用ルール}
本ファイルを使用するにあたってのルールは以下の通りです。
\begin{itemize}
\item 行数固定で読み込みを行う為、ヘッダの省略はできません。
\item 成分が重複して指定された場合にはエラー終了します。
\item $[$int01$]$と定義されている変分パラメータの種類の総数が異なる場合はエラー終了します。
\end{itemize}

\newpage
\subsection{InOrbital指定ファイル}
\label{Subsec:InOrbital}
\begin{equation}
|\phi_{\rm pair} \rangle = \left[\sum_{i, j=1}^{N_s} f_{ij}c_{i\uparrow}^{\dag}c_{j\downarrow}^{\dag} \right]^{N/2}|0 \rangle
\end{equation}
で表されるペア軌道の設定を行います。指定するパラメータはサイト番号$i, j$と変分パラメータ$f_{ij}$の初期値を設定します。以下にファイル例を記載します。

\begin{minipage}{12.5cm}
\begin{screen}
\begin{verbatim}
====================================
NOrbitalIdx 64  
====================================
==  i_j_OrbitalIdx  ==
====================================
   0     0     0.1    0.0 
   0     1     0.1    0.0     
   0     2     0.1    0.0    
   0     3     0.1    0.0    
 (continue...)
  15     9     0.2    0.0 
  15    10    0.2    0.0 
\end{verbatim}
\end{screen}
\end{minipage}

\subsubsection{ファイル形式}
以下のように行数に応じ異なる形式をとります($N_s$はサイト数)。
 \begin{itemize}
   \item  1行:  ヘッダ(何が書かれても問題ありません)。
   \item  2行:   [string01]~[int01]
   \item  3-5行:  ヘッダ(何が書かれても問題ありません)。
   \item  6 - 5+$N_s^2$行: [int02]~[int03]~[double01]~[double02]
  \end{itemize}
\subsubsection{パラメータ}
 \begin{itemize}

   \item  $[$string01$]$
   
    {\bf 形式 :} string型 (空白不可)

   {\bf 説明 :} 変分パラメータのセット総数のキーワード名を指定します(任意)。

   \item  $[$int01$]$
   
    {\bf 形式 :} int型 (空白不可)

   {\bf 説明 :} 変分パラメータのセット総数を指定します。

  \item  $[$int02$]$, $[$int03$]$
   
 {\bf 形式 :} int型 (空白不可)

{\bf 説明 :} サイト番号を指定する整数。0以上\verb|Nsite|{未満}で指定します。
 
   \item  $[$double01$]$, $[$double02$]$
   
   {\bf 形式 :} double型 (空白不可)

  {\bf 説明 :} $[$double01$]$が実部、$[$double02$]$が虚部を与えます。\verb|DH4|指定ファイルで型を実部に指定している場合は$[$double02$]$の値は破棄されます。

  
\end{itemize}

\subsubsection{使用ルール}
本ファイルを使用するにあたってのルールは以下の通りです。
\begin{itemize}
\item 行数固定で読み込みを行う為、ヘッダの省略はできません。
\item 成分が重複して指定された場合にはエラー終了します。
\item $[$int01$]$と定義されている変分パラメータの種類の総数が異なる場合はエラー終了します。
\item $[$int02$]$, $[$int03$]$を指定する際、範囲外の整数を指定した場合はエラー終了します。
\item \verb|Orbital|指定ファイルで紐付けされるサイト番号とパラメータの種類と、入力ファイルで指定されるパラメータの値の整合性がとれない場合は警告を出します。その際、入力値としては平均された値が採用されます。

\end{itemize}



\newpage
\subsection{OneBodyG指定ファイル}
\label{Subsec:onebodyg}
一体グリーン関数$\langle c_{i\sigma_1}^{\dagger}c_{j\sigma_2}\rangle$を計算します。以下にファイル例を記載します。

\begin{minipage}{12.5cm}
\begin{screen}
\begin{verbatim}
===============================
NCisAjs         24
===============================
======== Green functions ======
===============================
    0     0     0     0
    0     1     0     1
    1     0     1     0
    1     1     1     1
    2     0     2     0
    2     1     2     1
    3     0     3     0
    3     1     3     1
    4     0     4     0
    4     1     4     1
    5     0     5     0
    5     1     5     1
    6     0     6     0
    6     1     6     1
    7     0     7     0
    7     1     7     1
    8     0     8     0
    8     1     8     1
    9     0     9     0
    9     1     9     1
   10     0    10     0
   10     1    10     1
   11     0    11     0
   11     1    11     1
\end{verbatim}
\end{screen}
\end{minipage}

\subsubsection{ファイル形式}
以下のように行数に応じ異なる形式をとります。
 \begin{itemize}
   \item  1行:  ヘッダ(何が書かれても問題ありません)。
   \item  2行:   [string01]~[int01]
   \item  3-5行:  ヘッダ(何が書かれても問題ありません)。
   \item  6行以降: [int02]~~[int03]~~[int04]~~[int05]
  \end{itemize}
\subsubsection{パラメータ}
 \begin{itemize}

   \item  $[$string01$]$
   
    {\bf 形式 :} string型 (空白不可)

   {\bf 説明 :} 一体グリーン関数成分総数のキーワード名を指定します(任意)。

   \item  $[$int01$]$
   
    {\bf 形式 :} int型 (空白不可)

   {\bf 説明 :} 一体グリーン関数成分の総数を指定します。

  \item  $[$int02$]$, $[$int04$]$

 {\bf 形式 :} int型 (空白不可)

{\bf 説明 :} サイト番号を指定する整数。0以上\verb|Nsite|{未満}で指定します。
 
  \item  $[$int03$]$, $[$int05$]$

 {\bf 形式 :} int型 (空白不可)

{\bf 説明 :} スピンを指定する整数。\\
0: アップスピン\\
1: ダウンスピン\\
を選択することが出来ます。

\end{itemize}

\subsubsection{使用ルール}
本ファイルを使用するにあたってのルールは以下の通りです。
\begin{itemize}
\item 行数固定で読み込みを行う為、ヘッダの省略はできません。
\item 成分が重複して指定された場合にはエラー終了します。
\item $[$int01$]$と定義されている一体グリーン関数成分の総数が異なる場合はエラー終了します。
\item $[$int02$]$-$[$int05$]$を指定する際、範囲外の整数を指定した場合はエラー終了します。
\end{itemize}

\newpage
\subsection{TwoBodyG指定ファイル}
\label{Subsec:twobodyg}
二体グリーン関数$\langle c_{i\sigma_1}^{\dagger}c_{j\sigma_2}c_{k\sigma_3}^{\dagger}c_{l\sigma_4}\rangle$を計算します。
{なお、スピンに関して計算する場合には、$i=j, k=l$となるよう設定してください。}
以下にファイル例を記載します。

\begin{minipage}{12.5cm}
\begin{screen}
\begin{verbatim}
=============================================
NCisAjsCktAltDC        576
=============================================
======== Green functions for Sq AND Nq ======
=============================================
    0     0     0     0     0     0     0     0
    0     0     0     0     0     1     0     1
    0     0     0     0     1     0     1     0
    0     0     0     0     1     1     1     1
    0     0     0     0     2     0     2     0
    0     0     0     0     2     1     2     1
    0     0     0     0     3     0     3     0
    0     0     0     0     3     1     3     1
    0     0     0     0     4     0     4     0
    0     0     0     0     4     1     4     1
    0     0     0     0     5     0     5     0
    0     0     0     0     5     1     5     1
    0     0     0     0     6     0     6     0
    0     0     0     0     6     1     6     1
    0     0     0     0     7     0     7     0
    0     0     0     0     7     1     7     1
    0     0     0     0     8     0     8     0
    0     0     0     0     8     1     8     1
    0     0     0     0     9     0     9     0
    0     0     0     0     9     1     9     1
    0     0     0     0    10     0    10     0
    0     0     0     0    10     1    10     1
    0     0     0     0    11     0    11     0
    0     0     0     0    11     1    11     1
    0     1     0     1     0     0     0     0
    …
\end{verbatim}
\end{screen}
\end{minipage}

\subsubsection{ファイル形式}
以下のように行数に応じ異なる形式をとります。
 \begin{itemize}
   \item  1行:  ヘッダ(何が書かれても問題ありません)。
   \item  2行:   [string01]~[int01]
   \item  3-5行:  ヘッダ(何が書かれても問題ありません)。
   \item  6行以降: [int02]~~[int03]~~[int04]~~[int05]~~[int06]~~[int07]~~[int08]~~[int09]
  \end{itemize}
\subsubsection{パラメータ}
 \begin{itemize}

   \item  $[$string01$]$
   
    {\bf 形式 :} string型 (空白不可)

   {\bf 説明 :} 二体グリーン関数成分総数のキーワード名を指定します(任意)。

   \item  $[$int01$]$
   
    {\bf 形式 :} int型 (空白不可)

   {\bf 説明 :} 二体グリーン関数成分の総数を指定します。

  \item  $[$int02$]$, $[$int04$]$,$[$int06$]$, $[$int08$]$

 {\bf 形式 :} int型 (空白不可)

{\bf 説明 :} サイト番号を指定する整数。0以上\verb|Nsite|{未満}で指定します。
 
  \item  $[$int03$]$, $[$int05$]$,$[$int07$]$, $[$int09$]$

 {\bf 形式 :} int型 (空白不可)

{\bf 説明 :} スピンを指定する整数。\\
0: アップスピン\\
1: ダウンスピン\\
を選択することが出来ます。

\end{itemize}

\subsubsection{使用ルール}
本ファイルを使用するにあたってのルールは以下の通りです。
\begin{itemize}
\item 行数固定で読み込みを行う為、ヘッダの省略はできません。
\item 成分が重複して指定された場合にはエラー終了します。
\item {スピンに関して計算する場合、$i=j, k=l$を満たさない場合ペアが存在するとエラー終了します。}
\item $[$int01$]$と定義されているニ体グリーン関数成分の総数が異なる場合はエラー終了します。
\item $[$int02$]$-$[$int09$]$を指定する際、範囲外の整数を指定した場合はエラー終了します。
\end{itemize}

\newpage
\section{出力ファイル}
\label{Sec:outputfile}
T.B.D
\newpage
\section{エラーメッセージ一覧}
T.B.D
%----------------------------------------------------------
%----------------------------------------------------------
%----------------------------------------------------------
