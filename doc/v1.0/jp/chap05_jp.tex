% !TEX root = userguide_jp.tex
%----------------------------------------------------------
\chapter{アルゴリズム}
\label{Ch:algorithm}

\section{Bogoliubov表現}\label{sec_bogoliubov_rep}

スピン系の計算において一体項(\verb|transfer|)、\verb|InterAll|形式での相互作用、
相関関数のインデックスの指定にはBogoliubov表現が使われています。
スピンの演算子は次のように生成$\cdot$消滅演算子で書き換えられます。
\begin{align}
  S_{i z} &= \sum_{\sigma = -S}^{S} \sigma c_{i \sigma}^\dagger c_{i \sigma}
  \\
  S_{i}^+ &= \sum_{\sigma = -S}^{S-1} 
  \sqrt{S(S+1) - \sigma(\sigma+1)} 
  c_{i \sigma+1}^\dagger c_{i \sigma}
  \\
  S_{i}^- &= \sum_{\sigma = -S}^{S-1} 
  \sqrt{S(S+1) - \sigma(\sigma+1)} 
  c_{i \sigma}^\dagger c_{i \sigma+1}
\end{align}

%----------------------------------------------------------
